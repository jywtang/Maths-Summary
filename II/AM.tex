\documentclass[12pt, a4paper]{article}
\usepackage[T1]{fontenc}
\usepackage{amsmath, amssymb, amsthm}
\usepackage{babel}
\usepackage{siunitx}
\usepackage{tikz}
\usepackage{centernot}
\usepackage{tcolorbox}
\usepackage{cancel}
\usepackage{enumitem}
\usepackage{xparse}

\usetikzlibrary{arrows}

\setlength{\parindent}{0pt}


% Matrix groups
\newcommand{\GL}{\mathrm{GL}}
\newcommand{\M}{\mathrm{M}}
\newcommand{\Or}{\mathrm{O}}
\newcommand{\PGL}{\mathrm{PGL}}
\newcommand{\PSL}{\mathrm{PSL}}
\newcommand{\PSO}{\mathrm{PSO}}
\newcommand{\PSU}{\mathrm{PSU}}
\newcommand{\SL}{\mathrm{SL}}
\newcommand{\SO}{\mathrm{SO}}
\newcommand{\Spin}{\mathrm{Spin}}
\newcommand{\Sp}{\mathrm{Sp}}
\newcommand{\SU}{\mathrm{SU}}
\newcommand{\U}{\mathrm{U}}
\newcommand{\Mat}{\mathrm{Mat}}


% Special sets
\newcommand{\C}{\mathbb{C}}
\newcommand{\CP}{\mathbb{CP}}
\newcommand{\F}{\mathbb{F}}
\newcommand{\GG}{\mathbb{G}}
\newcommand{\N}{\mathbb{N}}
% \newcommand{\P}{\mathbb{P}}
\newcommand{\Q}{\mathbb{Q}}
\newcommand{\R}{\mathbb{R}}
\newcommand{\RP}{\mathbb{RP}}
\newcommand{\T}{\mathbb{T}}
\newcommand{\Z}{\mathbb{Z}}
\renewcommand{\H}{\mathbb{H}}

% Brackets
\renewcommand{\vec}[1]{\boldsymbol{\mathbf{#1}}}
\newcommand{\cis}[1]{ \cos\left( #1 \right) + i \sin \left( #1 \right)}

% Algebra
\DeclareMathOperator{\adj}{adj}
\DeclareMathOperator{\Ann}{Ann}
\DeclareMathOperator{\Aut}{Aut}
\DeclareMathOperator{\Char}{char}
\DeclareMathOperator{\disc}{disc}
\DeclareMathOperator{\dom}{dom}
\DeclareMathOperator{\fix}{fix}
\DeclareMathOperator{\Hom}{Hom}
\DeclareMathOperator{\id}{id}
\DeclareMathOperator{\image}{image}
\DeclareMathOperator{\im}{im}
\DeclareMathOperator{\tr}{tr}
\newcommand{\Bilin}{\mathrm{Bilin}}
\newcommand{\Frob}{\mathrm{Frob}}



\let\Im\relax
\let\Re\relax


\DeclareMathOperator{\hcf}{hcf}
\DeclareMathOperator{\Isom}{Isom}
\DeclareMathOperator{\lcm}{lcm}
\DeclareMathOperator{\sgn}{sgn}
\DeclareMathOperator{\supp}{supp}
\DeclareMathOperator{\Sym}{Sym}
\DeclareMathOperator{\Syl}{Syl}
\DeclareMathOperator{\Im}{Im}
\DeclareMathOperator{\Re}{Re}
\DeclareMathOperator{\Ker}{Ker}


% Theorems
\theoremstyle{definition}
\newtheorem*{aim}{Aim}
\newtheorem*{axiom}{Axiom}
\newtheorem*{claim}{Claim}
\newtheorem*{cor}{Corollary}
\newtheorem*{conjecture}{Conjecture}
\newtheorem*{defi}{Definition}
\newtheorem*{eg}{Example}
\newtheorem*{ex}{Exercise}
\newtheorem*{fact}{Fact}
\newtheorem*{law}{Law}
\newtheorem*{lemma}{Lemma}
\newtheorem*{notation}{Notation}
\newtheorem*{prop}{Proposition}
\newtheorem*{question}{Question}
\newtheorem*{rrule}{Rule}
\newtheorem*{thm}{Theorem}
\newtheorem*{assumption}{Assumption}

\newtheorem*{remark}{Remark}
\newtheorem*{warning}{Warning}
\newtheorem*{exercise}{Exercise}

\newtheorem{nthm}{Theorem}[section]
\newtheorem{nlemma}[nthm]{Lemma}
\newtheorem{nprop}[nthm]{Proposition}
\newtheorem{ncor}[nthm]{Corollary}


\newcommand{\abs}[1]{\left| #1 \right|} % for absolute value
\newcommand{\grad}[1]{\mathbf{\nabla} #1} % for gradient
\let\divsymb=\div % rename builtin command \div to \divsymb
\renewcommand{\div}[1]{\mathbf{\nabla} \cdot #1} % for divergence
\newcommand{\curl}[1]{\mathbf{\nabla} \times #1} % for curl




% Derivatives

\newcommand{\dd}[1][]{\mathrm{d} #1}
\newcommand{\odiff}[1]{\frac{\dd}{\dd{#1}}}
\newcommand{\odv}[2]{\frac{\dd{#1}}{\dd{#2}}}
\newcommand{\pdiff}[1]{\frac{\partial}{\partial{#1}}}
\newcommand{\pdv}[2]{\frac{\partial{#1}}{\partial{#2}}}


% \def\diffd{\mathrm{d}}
% \DeclareDocumentCommand\differential{ o g d() }{ % Differential 'd'
%     % o: optional n for nth differential
%     % g: optional argument for readability and to control spacing
%     % d: long-form as in d(cos x)
%     \IfNoValueTF{#2}{
%         \IfNoValueTF{#3}
%         {\diffd\IfNoValueTF{#1}{}{^{#1}}}
%         {\mathinner{\diffd\IfNoValueTF{#1}{}{^{#1}}\argopen(#3\argclose)}}
%     }
%     {\mathinner{\diffd\IfNoValueTF{#1}{}{^{#1}}#2} \IfNoValueTF{#3}{}{(#3)}}
% }
% \DeclareDocumentCommand\dd{}{\differential} % Shorthand for \differential

% \DeclareDocumentCommand\derivative{ s o m g d() }
% { % Total derivative
%     % s: star for \flatfrac flat derivative
%     % o: optional n for nth derivative
%     % m: mandatory (x in df/dx)
%     % g: optional (f in df/dx)
%     % d: long-form d/dx(...)
%     \IfBooleanTF{#1}
%     {\let\fractype\flatfrac}
%     {\let\fractype\frac}
%     \IfNoValueTF{#4}
%     {
%         \IfNoValueTF{#5}
%         {\fractype{\diffd \IfNoValueTF{#2}{}{^{#2}}}{\diffd #3\IfNoValueTF{#2}{}{^{#2}}}}
%         {\fractype{\diffd \IfNoValueTF{#2}{}{^{#2}}}{\diffd #3\IfNoValueTF{#2}{}{^{#2}}} \argopen(#5\argclose)}
%     }
%     {\fractype{\diffd \IfNoValueTF{#2}{}{^{#2}} #3}{\diffd #4\IfNoValueTF{#2}{}{^{#2}}}}
% }
% \DeclareDocumentCommand\dv{}{\derivative} % Shorthand for \derivative

% \DeclareDocumentCommand\partialderivative{ s o m g g d() }
% { % Partial derivative
%     % s: star for \flatfrac flat derivative
%     % o: optional n for nth derivative
%     % m: mandatory (x in df/dx)
%     % g: optional (f in df/dx)
%     % g: optional (y in d^2f/dxdy)
%     % d: long-form d/dx(...)
%     \IfBooleanTF{#1}
%     {\let\fractype\flatfrac}
%     {\let\fractype\frac}
%     \IfNoValueTF{#4}
%     {
%         \IfNoValueTF{#6}
%         {\fractype{\partial \IfNoValueTF{#2}{}{^{#2}}}{\partial #3\IfNoValueTF{#2}{}{^{#2}}}}
%         {\fractype{\partial \IfNoValueTF{#2}{}{^{#2}}}{\partial #3\IfNoValueTF{#2}{}{^{#2}}} \argopen(#6\argclose)}
%     }
%     {
%         \IfNoValueTF{#5}
%         {\fractype{\partial \IfNoValueTF{#2}{}{^{#2}} #3}{\partial #4\IfNoValueTF{#2}{}{^{#2}}}}
%         {\fractype{\partial^2 #3}{\partial #4 \partial #5}}
%     }
% }
% \DeclareDocumentCommand\pderivative{}{\partialderivative} % Shorthand for \partialderivative
% \DeclareDocumentCommand\pdv{}{\partialderivative} % Shorthand for \partialderivative

\DeclareDocumentCommand\variation{ o g d() }{ % Functional variation
    % o: optional n for nth differential
    % g: optional argument for readability and to control spacing
    % d: long-form as in d(F(g))
    \IfNoValueTF{#2}{
        \IfNoValueTF{#3}
        {\delta \IfNoValueTF{#1}{}{^{#1}}}
        {\mathinner{\delta \IfNoValueTF{#1}{}{^{#1}}\argopen(#3\argclose)}}
    }
    {\mathinner{\delta \IfNoValueTF{#1}{}{^{#1}}#2} \IfNoValueTF{#3}{}{(#3)}}
}
\DeclareDocumentCommand\var{}{\variation} % Shorthand for \variation

\DeclareDocumentCommand\functionalderivative{ s o m g d() }
{ % Functional derivative
    % s: star for \flatfrac flat derivative
    % o: optional n for nth derivative
    % m: mandatory (g in dF/dg)
    % g: optional (F in dF/dg)
    % d: long-form d/dx(...)
    \IfBooleanTF{#1}
    {\let\fractype\flatfrac}
    {\let\fractype\frac}
    \IfNoValueTF{#4}
    {
        \IfNoValueTF{#5}
        {\fractype{\variation \IfNoValueTF{#2}{}{^{#2}}}{\variation #3\IfNoValueTF{#2}{}{^{#2}}}}
        {\fractype{\variation \IfNoValueTF{#2}{}{^{#2}}}{\variation #3\IfNoValueTF{#2}{}{^{#2}}} \argopen(#5\argclose)}
    }
    {\fractype{\variation \IfNoValueTF{#2}{}{^{#2}} #3}{\variation #4\IfNoValueTF{#2}{}{^{#2}}}}
}
\DeclareDocumentCommand\fderivative{}{\functionalderivative} % Shorthand for \functionalderivative
\DeclareDocumentCommand\fdv{}{\functionalderivative} % Shorthand for \functionalderivative


\begin{document}
\section*{Asymptotic Methods \hfill II}

\section{Basic Definitions}
\begin{itemize}
    \item Big/little O notation, asymptotic equality, asymptotic sequence, asymptotic expansion, uniqueness
\end{itemize}
\section{Approximation of Integrals}
\subsection*{Watson's Lemma}
Suppose 
\[f(t) \sim \sum_{n=0}^\infty a_n t^{\alpha + \beta n} \text{ as } t \to 0^+\] 
$\alpha>-1$
\begin{enumerate}
    \item $\abs{f(t)} < Ke^{bt}$ for all $t>0$, or
    \item $\int_0^T \abs{f(t)} \odif{t} < \infty $ 
\end{enumerate}
Then \[\int_0^T e^{-xt} f(t) \odif{t} \sim \sum_{n=0}^\infty a_n \frac{\Gamma(\alpha+\beta 
n+1)}{x^{\alpha + \beta n + 1}} \text{ for } x \to +\infty\]

\subsection*{Laplace's Method} 
\[F(x) = \int_a^b f(t)e^{x\phi(t)} \odif{t} \text{ as } x \to \infty\]
    
\begin{itemize}
    \item Expand $\phi$ about global maximum, then evaluate $\int e^{-s^p} \odif{s}$: become gamma function
\end{itemize}

\subsection*{Method of Stationary Phase}
\[F(x) = \int_a^b f(t)e^{ix\phi(t)} \odif{t} \text{ as } x \to \infty \]

\begin{itemize}
    \item Riemann Lebesgue Lemma: the above integral tends to $0$ if $f$ integrable 
    \item If $\phi$ monotonic, only have contributions from two ends, decay as $O(1/x)$ (by parts)
    \item If $\phi$ has stationary points, have slower decay than $O(1/x)$: expand near stationary point, get $\int e^{-is^2} \odif{s}$
    \item All stationary points contribute to asymptotic behaviour
\end{itemize}
\subsection*{Method of Steepest Descent}
\begin{itemize}
    \item Deform contour in complex plane, can approximate better (e.g. by Watson/ Laplace) on lines of constant $\Im(\phi)$ (lines of steepest descent contours)
    \item On saddle points $\phi'(c)=0$, choose contour where $\Re(\phi(z))$ decreases away from $c$.
    \item Evaluate integrals with constant phase using Laplace (approximating contour as straight line locally) 
\end{itemize}
\section{Airy function}
Airy equation $y''=xy$: solutions are \[\int_C \exp\left(\frac{1}{3}t^3 + xt \right) \odif{t} \]
for contour $C$ starting and ending where integral is defined
\begin{align*}
    \text{Ai}(z) &= \frac{1}{\pi}\int_0^\infty \cos\left(\frac{1}{3}t^3 + xt \right) \odif{t} \\
     & \sim \frac{1}{2\sqrt{\pi}}x^{-1/4}e^{-2x^{3/2}/3} &\text{, as } x\to +\infty \\
     &= \frac{1}{\sqrt{\pi}} (-x)^{-1/4} \cos\left(\frac{2}{3}(-x)^{3/2} -\frac{\pi}{4}\right)  &\text{, as } x\to -\infty   
\end{align*}
\section{Solution to 2nd order ODE}
\subsection{Liouville Green Method}
\begin{itemize}
    \item Solve $y''=Q(x)y$ about an irregular singular point
    \item Let $ y = e^{S(X)}$, then \[S'' +(S')^2 = Q(x)\]
    Suppose $Q$ slowly varying, then first approximation is $S_0' = \pm \sqrt{Q}$, i.e. $S_0 = \pm \int \sqrt{Q} \odif{x}$.
    \item Adding another order: $S_1 \sim \frac{-1}{4}\log Q$
    \item Can write recurrent relation for higher orders
    \item Liouville Green approximation is sum of $\pm$ solutions for $S$
\end{itemize}
\subsection*{WKBJ Method}
\begin{itemize}
    \item Solve $\varepsilon^2 y''=q(x)y$ with small $\varepsilon$
    \item Asymptotic solution \[y = \sum_{\pm} A_{\pm} q^{-1/4}\exp\left( \pm \frac{1}{\varepsilon} \int \sqrt{q(x)} \odif{x}\right)\]
\end{itemize}
\subsection*{Turning point}
\begin{itemize}
\item Method fails near $z=a$ if $q(a)=0$, called a turning point

\item Use local approximation: let $q'(a) = \mu$ (wlog $\mu > 0$,so $q>0$ for $x>a$), then near $a$, \[\varepsilon^2 y'' \approx \mu (x-a) y,\]
Substitute $z=\left(\frac{\mu}{\varepsilon^2}\right)^{1/3}(x-a)$ gives Airy equation
\[\odv[order=2]{y}{z} = zy\]

\item The Airy function gives the approximate local solution that decays for large $x$: 

\begin{align*}
    y_0 & \approx B\text{Ai}(z) \\
        & \approx \frac{B}{2\sqrt{\pi}}z^{-1/4}\exp\left({-\frac{2}{3}z^{3/2}}\right) \\
        & = \frac{B}{2\sqrt{\pi} \left(\left(\mu/\varepsilon^2\right)^{1/3}(x-a)\right)^{1/4}} \exp\left({-\frac{2}{3}\frac{\sqrt{\mu}}{\varepsilon}(x-a)^{3/2}}\right) 
        &\text{, as } x\to +\infty \\
    y_0 & \approx \frac{B}{\sqrt{\pi}} (-z)^{-1/4} \cos\left(\frac{2}{3}(-z)^{3/2} -\frac{\pi}{4}\right)  \\
        &= \frac{B}{ \sqrt{\pi} \left(\left(\mu/\varepsilon^2 \right)^{1/3} (a-x) \right)^{1/4}} 
        \cos\left(\frac{2}{3}\frac{\sqrt{\mu}}{\varepsilon}(a-x))^{3/2} -\frac{\pi}{4}\right)  
        &\text{, as } x\to -\infty   
\end{align*}

\item WKBJ solution is valid far away from $x=a$, and by bringing this close to $x=a$ where linear approximation of $q$ is valid, have asymptotic behaviour: 
\begin{align*}
    y_+ &\approx Aq^{-1/4} \exp\left(-\frac{1}{\varepsilon} \int_a^x \sqrt{q(t)} \odif{t} \right) \\
    &\approx \frac{A}{(\mu (x-a))^{1/4}} \exp\left(-\frac{1}{\varepsilon} \int_a^x \sqrt{\mu (t-a) } \odif{t} \right) \\
    &= \frac{A}{(\mu (x-a))^{1/4}} \exp\left(-\frac{2}{3}\frac{\sqrt{\mu}}{\varepsilon}(x-a)^{3/2} \right)
    &\text{, for } x>a, \\
    y_- &\approx C(-q)^{-1/4} \cos\left(\frac{1}{\varepsilon} \int_x^a \sqrt{-q(t)} \odif{t} - \gamma \right) \\
    &\approx \frac{C}{(\mu (a-x))^{1/4}} \cos\left(\frac{1}{\varepsilon} \int_x^a \sqrt{\mu (a-t) } \odif{t} -\gamma \right) \\
    &= \frac{C}{(\mu (a-x))^{1/4}} \cos\left(\frac{2}{3}\frac{\sqrt{\mu}}{\varepsilon}(a-x)^{3/2} -\gamma \right) 
    &\text{, for } x<a,
\end{align*}

Matching asymptotics for both gives:
\[    \gamma = \frac{\pi}{4}, \quad  B = 2\sqrt{\pi} (\mu\varepsilon)^{-1/6}, \quad  C = 2A\]
Connection formula:
\begin{align*}
    y_+ &= Aq^{-1/4} \exp\left(-\frac{1}{\varepsilon} \int_a^x \sqrt{q(t)} \odif{t} \right)
    &\text{, for } x>a, x-a \gg \varepsilon^{2/3} \\
    y_0 &= 2\sqrt{\pi} (\mu\varepsilon)^{-1/6} \text{Ai}(z) 
    &\text{, for } \abs{x-a} \ll 1\\
    y_- &= 2A(-q)^{-1/4} \cos\left(\frac{1}{\varepsilon} \int_a^x \sqrt{-q(t)} \odif{t} - \frac{\pi}{4} \right)  
    &\text{, for } x<a, a-x \gg \varepsilon^{2/3} 
\end{align*}
\end{itemize}
\subsection*{Two turning points in $q$ and bound state}
\begin{itemize}
    \item Suppose $a<b$ are the turning points and $q>0$ outside them
    \item Want a solution that decays exponentially on either sides, so 
    \begin{align*}
    y_1 &\sim Aq^{-1/4} \exp\left(-\frac{1}{\varepsilon} \int_x^a \sqrt{q(t)} \odif{t} \right) 
    &\text{, for } x<a \\
    y_3 &\sim Bq^{-1/4} \exp\left(-\frac{1}{\varepsilon} \int_b^x \sqrt{q(t)} \odif{t} \right) 
    &\text{, for } x>b \\
    \end{align*} 
    Then by connection formula, the region in the middle ($a<x<b$) has two asymptotic expansions that need to match
    \begin{align*}
        y_2 &\approx 2A(-q)^{-1/4} \cos\left(\frac{1}{\varepsilon} \int_a^x \sqrt{-q(t)} \odif{t} - \frac{\pi}{4} \right)  \\
        y_2 &\approx 2B(-q)^{-1/4} \cos\left(\frac{1}{\varepsilon} \int_x^b \sqrt{-q(t)} \odif{t} - \frac{\pi}{4} \right)  
    \end{align*}
    So we need (by compound angle formula)
    \[\frac{1}{\varepsilon} \int_a^b \sqrt{-q(t)} \odif{t} = \left(n+\frac{1}{2}\right)\pi,\] 
    where $n = 0,1,2,\dots$, and $A=(-1)^nB$.
\end{itemize}
\end{document}