\documentclass[12pt, a4paper]{article}
\usepackage[T1]{fontenc}
\usepackage{amsmath, amssymb, amsthm}
\usepackage{babel}
\usepackage{siunitx}
\usepackage{tikz}
\usepackage{centernot}
\usepackage{tcolorbox}
\usepackage{cancel}
\usepackage{enumitem}
\usepackage{xparse}

\usetikzlibrary{arrows}

\setlength{\parindent}{0pt}


% Matrix groups
\newcommand{\GL}{\mathrm{GL}}
\newcommand{\M}{\mathrm{M}}
\newcommand{\Or}{\mathrm{O}}
\newcommand{\PGL}{\mathrm{PGL}}
\newcommand{\PSL}{\mathrm{PSL}}
\newcommand{\PSO}{\mathrm{PSO}}
\newcommand{\PSU}{\mathrm{PSU}}
\newcommand{\SL}{\mathrm{SL}}
\newcommand{\SO}{\mathrm{SO}}
\newcommand{\Spin}{\mathrm{Spin}}
\newcommand{\Sp}{\mathrm{Sp}}
\newcommand{\SU}{\mathrm{SU}}
\newcommand{\U}{\mathrm{U}}
\newcommand{\Mat}{\mathrm{Mat}}


% Special sets
\newcommand{\C}{\mathbb{C}}
\newcommand{\CP}{\mathbb{CP}}
\newcommand{\F}{\mathbb{F}}
\newcommand{\GG}{\mathbb{G}}
\newcommand{\N}{\mathbb{N}}
% \newcommand{\P}{\mathbb{P}}
\newcommand{\Q}{\mathbb{Q}}
\newcommand{\R}{\mathbb{R}}
\newcommand{\RP}{\mathbb{RP}}
\newcommand{\T}{\mathbb{T}}
\newcommand{\Z}{\mathbb{Z}}
\renewcommand{\H}{\mathbb{H}}

% Brackets
\renewcommand{\vec}[1]{\boldsymbol{\mathbf{#1}}}
\newcommand{\cis}[1]{ \cos\left( #1 \right) + i \sin \left( #1 \right)}

% Algebra
\DeclareMathOperator{\adj}{adj}
\DeclareMathOperator{\Ann}{Ann}
\DeclareMathOperator{\Aut}{Aut}
\DeclareMathOperator{\Char}{char}
\DeclareMathOperator{\disc}{disc}
\DeclareMathOperator{\dom}{dom}
\DeclareMathOperator{\fix}{fix}
\DeclareMathOperator{\Hom}{Hom}
\DeclareMathOperator{\id}{id}
\DeclareMathOperator{\image}{image}
\DeclareMathOperator{\im}{im}
\DeclareMathOperator{\tr}{tr}
\newcommand{\Bilin}{\mathrm{Bilin}}
\newcommand{\Frob}{\mathrm{Frob}}



\let\Im\relax
\let\Re\relax


\DeclareMathOperator{\hcf}{hcf}
\DeclareMathOperator{\Isom}{Isom}
\DeclareMathOperator{\lcm}{lcm}
\DeclareMathOperator{\sgn}{sgn}
\DeclareMathOperator{\supp}{supp}
\DeclareMathOperator{\Sym}{Sym}
\DeclareMathOperator{\Syl}{Syl}
\DeclareMathOperator{\Im}{Im}
\DeclareMathOperator{\Re}{Re}
\DeclareMathOperator{\Ker}{Ker}


% Theorems
\theoremstyle{definition}
\newtheorem*{aim}{Aim}
\newtheorem*{axiom}{Axiom}
\newtheorem*{claim}{Claim}
\newtheorem*{cor}{Corollary}
\newtheorem*{conjecture}{Conjecture}
\newtheorem*{defi}{Definition}
\newtheorem*{eg}{Example}
\newtheorem*{ex}{Exercise}
\newtheorem*{fact}{Fact}
\newtheorem*{law}{Law}
\newtheorem*{lemma}{Lemma}
\newtheorem*{notation}{Notation}
\newtheorem*{prop}{Proposition}
\newtheorem*{question}{Question}
\newtheorem*{rrule}{Rule}
\newtheorem*{thm}{Theorem}
\newtheorem*{assumption}{Assumption}

\newtheorem*{remark}{Remark}
\newtheorem*{warning}{Warning}
\newtheorem*{exercise}{Exercise}

\newtheorem{nthm}{Theorem}[section]
\newtheorem{nlemma}[nthm]{Lemma}
\newtheorem{nprop}[nthm]{Proposition}
\newtheorem{ncor}[nthm]{Corollary}


\newcommand{\abs}[1]{\left| #1 \right|} % for absolute value
\newcommand{\grad}[1]{\mathbf{\nabla} #1} % for gradient
\let\divsymb=\div % rename builtin command \div to \divsymb
\renewcommand{\div}[1]{\mathbf{\nabla} \cdot #1} % for divergence
\newcommand{\curl}[1]{\mathbf{\nabla} \times #1} % for curl




% Derivatives

\newcommand{\dd}[1][]{\mathrm{d} #1}
\newcommand{\odiff}[1]{\frac{\dd}{\dd{#1}}}
\newcommand{\odv}[2]{\frac{\dd{#1}}{\dd{#2}}}
\newcommand{\pdiff}[1]{\frac{\partial}{\partial{#1}}}
\newcommand{\pdv}[2]{\frac{\partial{#1}}{\partial{#2}}}


% \def\diffd{\mathrm{d}}
% \DeclareDocumentCommand\differential{ o g d() }{ % Differential 'd'
%     % o: optional n for nth differential
%     % g: optional argument for readability and to control spacing
%     % d: long-form as in d(cos x)
%     \IfNoValueTF{#2}{
%         \IfNoValueTF{#3}
%         {\diffd\IfNoValueTF{#1}{}{^{#1}}}
%         {\mathinner{\diffd\IfNoValueTF{#1}{}{^{#1}}\argopen(#3\argclose)}}
%     }
%     {\mathinner{\diffd\IfNoValueTF{#1}{}{^{#1}}#2} \IfNoValueTF{#3}{}{(#3)}}
% }
% \DeclareDocumentCommand\dd{}{\differential} % Shorthand for \differential

% \DeclareDocumentCommand\derivative{ s o m g d() }
% { % Total derivative
%     % s: star for \flatfrac flat derivative
%     % o: optional n for nth derivative
%     % m: mandatory (x in df/dx)
%     % g: optional (f in df/dx)
%     % d: long-form d/dx(...)
%     \IfBooleanTF{#1}
%     {\let\fractype\flatfrac}
%     {\let\fractype\frac}
%     \IfNoValueTF{#4}
%     {
%         \IfNoValueTF{#5}
%         {\fractype{\diffd \IfNoValueTF{#2}{}{^{#2}}}{\diffd #3\IfNoValueTF{#2}{}{^{#2}}}}
%         {\fractype{\diffd \IfNoValueTF{#2}{}{^{#2}}}{\diffd #3\IfNoValueTF{#2}{}{^{#2}}} \argopen(#5\argclose)}
%     }
%     {\fractype{\diffd \IfNoValueTF{#2}{}{^{#2}} #3}{\diffd #4\IfNoValueTF{#2}{}{^{#2}}}}
% }
% \DeclareDocumentCommand\dv{}{\derivative} % Shorthand for \derivative

% \DeclareDocumentCommand\partialderivative{ s o m g g d() }
% { % Partial derivative
%     % s: star for \flatfrac flat derivative
%     % o: optional n for nth derivative
%     % m: mandatory (x in df/dx)
%     % g: optional (f in df/dx)
%     % g: optional (y in d^2f/dxdy)
%     % d: long-form d/dx(...)
%     \IfBooleanTF{#1}
%     {\let\fractype\flatfrac}
%     {\let\fractype\frac}
%     \IfNoValueTF{#4}
%     {
%         \IfNoValueTF{#6}
%         {\fractype{\partial \IfNoValueTF{#2}{}{^{#2}}}{\partial #3\IfNoValueTF{#2}{}{^{#2}}}}
%         {\fractype{\partial \IfNoValueTF{#2}{}{^{#2}}}{\partial #3\IfNoValueTF{#2}{}{^{#2}}} \argopen(#6\argclose)}
%     }
%     {
%         \IfNoValueTF{#5}
%         {\fractype{\partial \IfNoValueTF{#2}{}{^{#2}} #3}{\partial #4\IfNoValueTF{#2}{}{^{#2}}}}
%         {\fractype{\partial^2 #3}{\partial #4 \partial #5}}
%     }
% }
% \DeclareDocumentCommand\pderivative{}{\partialderivative} % Shorthand for \partialderivative
% \DeclareDocumentCommand\pdv{}{\partialderivative} % Shorthand for \partialderivative

\DeclareDocumentCommand\variation{ o g d() }{ % Functional variation
    % o: optional n for nth differential
    % g: optional argument for readability and to control spacing
    % d: long-form as in d(F(g))
    \IfNoValueTF{#2}{
        \IfNoValueTF{#3}
        {\delta \IfNoValueTF{#1}{}{^{#1}}}
        {\mathinner{\delta \IfNoValueTF{#1}{}{^{#1}}\argopen(#3\argclose)}}
    }
    {\mathinner{\delta \IfNoValueTF{#1}{}{^{#1}}#2} \IfNoValueTF{#3}{}{(#3)}}
}
\DeclareDocumentCommand\var{}{\variation} % Shorthand for \variation

\DeclareDocumentCommand\functionalderivative{ s o m g d() }
{ % Functional derivative
    % s: star for \flatfrac flat derivative
    % o: optional n for nth derivative
    % m: mandatory (g in dF/dg)
    % g: optional (F in dF/dg)
    % d: long-form d/dx(...)
    \IfBooleanTF{#1}
    {\let\fractype\flatfrac}
    {\let\fractype\frac}
    \IfNoValueTF{#4}
    {
        \IfNoValueTF{#5}
        {\fractype{\variation \IfNoValueTF{#2}{}{^{#2}}}{\variation #3\IfNoValueTF{#2}{}{^{#2}}}}
        {\fractype{\variation \IfNoValueTF{#2}{}{^{#2}}}{\variation #3\IfNoValueTF{#2}{}{^{#2}}} \argopen(#5\argclose)}
    }
    {\fractype{\variation \IfNoValueTF{#2}{}{^{#2}} #3}{\variation #4\IfNoValueTF{#2}{}{^{#2}}}}
}
\DeclareDocumentCommand\fderivative{}{\functionalderivative} % Shorthand for \functionalderivative
\DeclareDocumentCommand\fdv{}{\functionalderivative} % Shorthand for \functionalderivative


\begin{document}
\section*{Fluid \hfill II}

\section{IB Recap}
\begin{itemize}
    \item Continuum Hypothesis, Eulerian/Lagrangian derivative
    \item Mass conservation, Momentum conservation
    \item KBC, DBC
\end{itemize}

\section{Newtonian Viscous Flow}
\begin{itemize}
    \item Viscosity definition
    \item Rate of strain and vorticity: symmetric and antisymmetric parts of $\pdv{u_i}{x_j}$, write $\Omega_{ij} = -\varepsilon_{ijk}\Omega_k$, vorticity as twice angular velocity
    \item Volume forces and surface force (traction), tensor relation between traction and stress tensor: $\tau_i = \sigma_{ij} n_j$ by force balance
    \item Symmetry of $\sigma$ by moment balance
    \item Constitutive equation for Newtonian fluid: linear, instantaneous and isotropic: \[\sigma_{ij} = -p\delta_{ij} + 2\mu e_{ij}\]
    \item Cauchy momentum equation: \[\rho \left(\pdv{u_i}{t} + u_j \pdv{u_i}{x_j}\right) = F_i + \pdv{\sigma_{ij}}{x_j} \]
    \item For Newtonian fluids, this gives N-S equations
          \[\rho \left( \pdv{\vec{u}}{t}  + (\vec{u} \cdot \nabla)\vec{u}  \right) = -\grad{p} + \mu \nabla^2 \vec{u} + \vec{F} \]
          \[\rho \left( \pdv{u_i}{t}  + u_j \pdv{u_i}{x_j} \right) = -\pdv{p}{x_i} + \mu \pdv{u_i}{x_j,x_j} + F_i \]
    \item Boundary conditions: no slip ($\vec{u}$ continuous), stress continuous ($\sigma_{ij} n_j$)
    \item Energy equation: for fixed volume $V$
          \begin{multline*}
            \frac{d}{dt} \int_V \frac12 \rho \abs{u}^2 dV = - \oint_S \frac12 \rho u^2 (u_j n_j) dS + \oint_S u_i \sigma_{ij} n_j dS  \\
             + \int_V F_i u_i dV - \int_V \pdv{u_i}{x_j}\sigma_{ij} dV  
          \end{multline*} 
          Change in KE = energy flux + work done due to surface and volume force + stress dissipation
    \item Non-dimensionalising: Reynolds number (Re $= \frac{\rho UL}{\mu}$) and Strouhal number (St $= \frac{L}{UT}$)
\end{itemize}
\section{Unidirectional Flows}
Velocity only has one component, $\odv{p}{x}$ constant
\begin{itemize}
    \item Plane Couette Flow (shear flow with $\odv{p}{x} = 0$)
    \item Plane Poiseuille Flow ($\odv{p}{x}$) consider flow rate, traction
    \item Stokes problems: Flow due to impulsively starting plane and oscillating plane
    \item Velocity diffuses (diffusive scaling $y \sim \sqrt{\nu t}$), similarity solution
    \item Viscous penetration length for oscillating case
    \item Flow between cylinders: solve using polars
\end{itemize}

\section{Stokes Flows}
\begin{itemize}
    \item Re $= 0$: Stokes equations \[\mu \nabla^2 \vec{u} =  \grad{p}- \vec{F}, \quad  \nabla \cdot \vec{u} = 0\]
    \item Solve for flow and pressure, then compute stress, traction, torque
    \item Properties: instantaneous, linear, reversible in time and space
    \item Stokes drag: $\vec{F}=6\pi\mu a\vec{U}$

\end{itemize}

\subsection*{Theoretical Results}
\begin{itemize}
    \item 'Useful Lemma': for Stokes flow $\vec{u}^S$ and admissable flow $\vec{u}$ (incompressible) in the same volume \[\int_V2\mu e_{ij}^S e_{ij} dV = \int_S\sigma_{ij}^S u_i n_j dS\]
    \item Minimum dissipation theorem: all incompressible flows with the same Dirichlet BCs, Stokes flow has minimum dissipation
    \item Stokes flow is unique
    \item Geometric bounding: use minimum dissipation to bound $ D = U_i \int_S \sigma_{ij} n_j dS = -\vec{F}\cdot \vec{U}$
    \item Reciprocal theorem for two Stokes flow in the same volume $V$ with different BCs on $S$
          \[\int_S u_i^2 \sigma_{ij}^1 n_j  dS = \int_S u_i^1\sigma_{ij}^2 n_j  dS = \int_V 2\mu e_{ij}^1 e_{ij}^2 dV\]
    \item By linearity, \[\begin{pmatrix}
                  \vec{F} \\ \vec{G} \end{pmatrix} =
              \begin{pmatrix}
                  \vec{A} & \vec{B} \\
                  \vec{C} & \vec{D}
              \end{pmatrix}
              \begin{pmatrix}
                  \vec{U} \\ \vec{\Omega}
              \end{pmatrix}\] and the general resistance matrix must be symmetric by reciprocal theorem
    \item Use symmetry/isometry of resistance tensors
\end{itemize}

\subsection*{Flow in a corner (2D)}
\begin{itemize}
    \item Use stream function
          $u_r = \frac{1}{r}\pdv{\psi}{\theta}$, $u_\theta = \pdv{\psi}{r}$
          \[\grad \cdot \vec{u} = \frac{1}{r} \pdv*{(ru_r)}{r} + \frac{1}{r}\pdv*{(u_\theta)}{\theta} = 0\]
          Vorticity $\vec{\omega} = \curl{\vec{u}} = (0,0,-\nabla^2\psi)$

          Taking curl of Stokes equation (gradient term drops out) gives $-\nabla^2\omega = \nabla^2\nabla^2\psi = 0$

    \item Source flow in a corner: some fixed volume flux $Q = \int ru_r d\theta$ with no slip boundary at constant angle

          Try $\psi = Qf(\theta)$ by scaling grounds (purely radial solution), vorticity equation gives $f'''' +4f'' =0$

          General solution \[f = A\cos2\theta + B\sin 2\theta + C + D\theta\]

          Use symmetry and normalisation (volume flux $Q$) to fix solution

    \item Flow past a corner: driven by disturbances far from corner

          Seek solutions of the form $\psi = r^\lambda f(\theta)$, general solution \[f = A\cos\lambda\theta + B\sin \lambda\theta + C\cos(\lambda-2)\theta + D\sin (\lambda-2)\theta \]
          Get Moffatt Eddies

          Use no slip BC
    \item Source flow fails when angle of corner too large: eigenvalues have Re$\lambda > 0$, so disturbances to source flow gives modes of corner flow that are dominant, do not decay far away from corner

\end{itemize}

\section{Thin-layer flow (lubrication theory)}
\begin{itemize}
    \item Almost unidirectional flow $\vec{u} = (u,v)$
    \item Vertical scale $h$, Horizontal scale $L$ over which $h$ changes and $h \ll L$, low reduced Reynolds number
    \begin{align*}
        0 &= -\pdv{p}{x} + \mu \pdv[order=2]{u}{y} \\
        0 &= -\pdv{p}{y}
    \end{align*}    
    So $p=p(x)$, and integrate to find $u$ directly, use BC
    \item Geometry of problem, solve unidirectional flow, find pressure gradient by mass conservation, and find forces
    \item Thrust-bearing flow, cylinder approaching wall, Hele-Shaw cell, gravitational spreading of a drop
\end{itemize}
\section{Vorticity}
\begin{itemize}
    \item Take curl of N-S and use vector identity to get \[\pdv{\vec{\omega}}{t} + (\vec{u}\cdot \nabla)\vec{\omega} = (\vec{\omega}\cdot \nabla)\vec{u} + \nu \nabla^2 \vec{\omega}\]
    Vortex stretching and diffusion
    \item Balance of advection and diffusion: downward velocity (e.g. by suction) will trap vorticity in a boundary layer
    \item Burgers vortex (derive again)
\end{itemize}
\section{Boundary Layer Flow}
\begin{itemize}
    \item Euler limit holds in most of the fluid except boundary layer (scaling $\sim \frac{L}{\sqrt{\text{Re}}}$), regularise discontinuity in velocity/enforce no-slip
    \item First boundary layer approx: $\pdv[order=2]{}{x} \ll \pdv[order=2]{}{y}$
    
    Viscous term and inertial term balance gives scaling for $\delta$; 
    
    Outside BL (flow tends to $(U(x,t),v)$), drop viscous term and get 
    \[\rho \left( \pdv{U}{t} + U \pdv{U}{x}\right) = -\pdv{p}{x}\]

    \item Second BL approx: Variation in pressure along and across the flow: (from $x,y$ momentum equations)
    \begin{align*}
        \delta p_x &\sim \rho U^2 \\
        \delta p_y &\sim \rho U^2 \frac{\delta^2}{L^2}
    \end{align*} 
    So variation of $p$ across flow is negligible for thin BL (when Re $\gg 1$)
    
    Approximate pressure gradient by
    \[\pdv{p}{x}_{\text{BL}} \approx \pdv{p}{x}_{\text{outside}} = - \rho \left( \pdv{U}{t} + U \pdv{U}{x}\right)\]

    \item BL equations (flow $(u,v)$, far field $U(x,y)$)    
    \begin{align*}
        &\rho \left( \pdv{u}{t} + u \pdv{u}{x} + v \pdv{u}{y}\right) = \rho \left( \pdv{U}{t} + U \pdv{U}{x}\right) + \mu \pdv[order=2]{u}{y} \\
        &\nabla\cdot\vec{u} = 0 \\
        &u = 0 \text{ on surface} \\
        &u \to U(x,t) \text{ as } y/\delta \to \infty
    \end{align*}
\end{itemize}
\subsection*{Examples}
\begin{itemize}
    \item BL near a flat plate: use stream function and scaling, get 1D ODE
    \item 2d momentum jet: balance of inertial and viscous terms, conservation of momentum flux
    \item acceleration/deceleration: acceleration keeps BL structure
    \item Wedge: have a generalisation of Stokes limit flow
    \item BL near a free surface: traction only has pressure, no tangential stress
    
    Boundary scaling: need to being velocity gradient to $0$, use geometry to find scaling $\delta \sim \sqrt{\frac{\nu L}{U}}$

    Use scaling to find dissipation dominated by external flow, not in BL (can use this to find drag force by dissipation of irrotational flow)
    \item Compare drag coefficient of rigid sphere, bubble in Stokes flow vs high Re flow
\end{itemize} 

\section{Stability of unidirectional inviscid flow}
\begin{itemize}
    \item Add perturbation to steady flow, investigate linear stability
    \item Example: Kelvin-Helmholtz instability: add irrotational flow to two layers of flow
    \item Kinemtic BC at interface: $\mdv*{y-\eta(x,t)}{t} = 0$, so \[v = \pdv{\eta}{t} + u\pdv{\eta}{x} \]
    Dynamic boundary: only have pressure continuous

    Linearise Bernoulli for unsteady potential flow 
\end{itemize}
\end{document}