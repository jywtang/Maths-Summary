\documentclass[12pt, a4paper]{article}
\usepackage[T1]{fontenc}
\usepackage{amsmath, amssymb, amsthm}
\usepackage{babel}
\usepackage{siunitx}
\usepackage{tikz}
\usepackage{centernot}
\usepackage{tcolorbox}
\usepackage{cancel}
\usepackage{enumitem}
\usepackage{xparse}

\usetikzlibrary{arrows}

\setlength{\parindent}{0pt}


% Matrix groups
\newcommand{\GL}{\mathrm{GL}}
\newcommand{\M}{\mathrm{M}}
\newcommand{\Or}{\mathrm{O}}
\newcommand{\PGL}{\mathrm{PGL}}
\newcommand{\PSL}{\mathrm{PSL}}
\newcommand{\PSO}{\mathrm{PSO}}
\newcommand{\PSU}{\mathrm{PSU}}
\newcommand{\SL}{\mathrm{SL}}
\newcommand{\SO}{\mathrm{SO}}
\newcommand{\Spin}{\mathrm{Spin}}
\newcommand{\Sp}{\mathrm{Sp}}
\newcommand{\SU}{\mathrm{SU}}
\newcommand{\U}{\mathrm{U}}
\newcommand{\Mat}{\mathrm{Mat}}


% Special sets
\newcommand{\C}{\mathbb{C}}
\newcommand{\CP}{\mathbb{CP}}
\newcommand{\F}{\mathbb{F}}
\newcommand{\GG}{\mathbb{G}}
\newcommand{\N}{\mathbb{N}}
% \newcommand{\P}{\mathbb{P}}
\newcommand{\Q}{\mathbb{Q}}
\newcommand{\R}{\mathbb{R}}
\newcommand{\RP}{\mathbb{RP}}
\newcommand{\T}{\mathbb{T}}
\newcommand{\Z}{\mathbb{Z}}
\renewcommand{\H}{\mathbb{H}}

% Brackets
\renewcommand{\vec}[1]{\boldsymbol{\mathbf{#1}}}
\newcommand{\cis}[1]{ \cos\left( #1 \right) + i \sin \left( #1 \right)}

% Algebra
\DeclareMathOperator{\adj}{adj}
\DeclareMathOperator{\Ann}{Ann}
\DeclareMathOperator{\Aut}{Aut}
\DeclareMathOperator{\Char}{char}
\DeclareMathOperator{\disc}{disc}
\DeclareMathOperator{\dom}{dom}
\DeclareMathOperator{\fix}{fix}
\DeclareMathOperator{\Hom}{Hom}
\DeclareMathOperator{\id}{id}
\DeclareMathOperator{\image}{image}
\DeclareMathOperator{\im}{im}
\DeclareMathOperator{\tr}{tr}
\newcommand{\Bilin}{\mathrm{Bilin}}
\newcommand{\Frob}{\mathrm{Frob}}



\let\Im\relax
\let\Re\relax


\DeclareMathOperator{\hcf}{hcf}
\DeclareMathOperator{\Isom}{Isom}
\DeclareMathOperator{\lcm}{lcm}
\DeclareMathOperator{\sgn}{sgn}
\DeclareMathOperator{\supp}{supp}
\DeclareMathOperator{\Sym}{Sym}
\DeclareMathOperator{\Syl}{Syl}
\DeclareMathOperator{\Im}{Im}
\DeclareMathOperator{\Re}{Re}
\DeclareMathOperator{\Ker}{Ker}


% Theorems
\theoremstyle{definition}
\newtheorem*{aim}{Aim}
\newtheorem*{axiom}{Axiom}
\newtheorem*{claim}{Claim}
\newtheorem*{cor}{Corollary}
\newtheorem*{conjecture}{Conjecture}
\newtheorem*{defi}{Definition}
\newtheorem*{eg}{Example}
\newtheorem*{ex}{Exercise}
\newtheorem*{fact}{Fact}
\newtheorem*{law}{Law}
\newtheorem*{lemma}{Lemma}
\newtheorem*{notation}{Notation}
\newtheorem*{prop}{Proposition}
\newtheorem*{question}{Question}
\newtheorem*{rrule}{Rule}
\newtheorem*{thm}{Theorem}
\newtheorem*{assumption}{Assumption}

\newtheorem*{remark}{Remark}
\newtheorem*{warning}{Warning}
\newtheorem*{exercise}{Exercise}

\newtheorem{nthm}{Theorem}[section]
\newtheorem{nlemma}[nthm]{Lemma}
\newtheorem{nprop}[nthm]{Proposition}
\newtheorem{ncor}[nthm]{Corollary}


\newcommand{\abs}[1]{\left| #1 \right|} % for absolute value
\newcommand{\grad}[1]{\mathbf{\nabla} #1} % for gradient
\let\divsymb=\div % rename builtin command \div to \divsymb
\renewcommand{\div}[1]{\mathbf{\nabla} \cdot #1} % for divergence
\newcommand{\curl}[1]{\mathbf{\nabla} \times #1} % for curl




% Derivatives

\newcommand{\dd}[1][]{\mathrm{d} #1}
\newcommand{\odiff}[1]{\frac{\dd}{\dd{#1}}}
\newcommand{\odv}[2]{\frac{\dd{#1}}{\dd{#2}}}
\newcommand{\pdiff}[1]{\frac{\partial}{\partial{#1}}}
\newcommand{\pdv}[2]{\frac{\partial{#1}}{\partial{#2}}}


% \def\diffd{\mathrm{d}}
% \DeclareDocumentCommand\differential{ o g d() }{ % Differential 'd'
%     % o: optional n for nth differential
%     % g: optional argument for readability and to control spacing
%     % d: long-form as in d(cos x)
%     \IfNoValueTF{#2}{
%         \IfNoValueTF{#3}
%         {\diffd\IfNoValueTF{#1}{}{^{#1}}}
%         {\mathinner{\diffd\IfNoValueTF{#1}{}{^{#1}}\argopen(#3\argclose)}}
%     }
%     {\mathinner{\diffd\IfNoValueTF{#1}{}{^{#1}}#2} \IfNoValueTF{#3}{}{(#3)}}
% }
% \DeclareDocumentCommand\dd{}{\differential} % Shorthand for \differential

% \DeclareDocumentCommand\derivative{ s o m g d() }
% { % Total derivative
%     % s: star for \flatfrac flat derivative
%     % o: optional n for nth derivative
%     % m: mandatory (x in df/dx)
%     % g: optional (f in df/dx)
%     % d: long-form d/dx(...)
%     \IfBooleanTF{#1}
%     {\let\fractype\flatfrac}
%     {\let\fractype\frac}
%     \IfNoValueTF{#4}
%     {
%         \IfNoValueTF{#5}
%         {\fractype{\diffd \IfNoValueTF{#2}{}{^{#2}}}{\diffd #3\IfNoValueTF{#2}{}{^{#2}}}}
%         {\fractype{\diffd \IfNoValueTF{#2}{}{^{#2}}}{\diffd #3\IfNoValueTF{#2}{}{^{#2}}} \argopen(#5\argclose)}
%     }
%     {\fractype{\diffd \IfNoValueTF{#2}{}{^{#2}} #3}{\diffd #4\IfNoValueTF{#2}{}{^{#2}}}}
% }
% \DeclareDocumentCommand\dv{}{\derivative} % Shorthand for \derivative

% \DeclareDocumentCommand\partialderivative{ s o m g g d() }
% { % Partial derivative
%     % s: star for \flatfrac flat derivative
%     % o: optional n for nth derivative
%     % m: mandatory (x in df/dx)
%     % g: optional (f in df/dx)
%     % g: optional (y in d^2f/dxdy)
%     % d: long-form d/dx(...)
%     \IfBooleanTF{#1}
%     {\let\fractype\flatfrac}
%     {\let\fractype\frac}
%     \IfNoValueTF{#4}
%     {
%         \IfNoValueTF{#6}
%         {\fractype{\partial \IfNoValueTF{#2}{}{^{#2}}}{\partial #3\IfNoValueTF{#2}{}{^{#2}}}}
%         {\fractype{\partial \IfNoValueTF{#2}{}{^{#2}}}{\partial #3\IfNoValueTF{#2}{}{^{#2}}} \argopen(#6\argclose)}
%     }
%     {
%         \IfNoValueTF{#5}
%         {\fractype{\partial \IfNoValueTF{#2}{}{^{#2}} #3}{\partial #4\IfNoValueTF{#2}{}{^{#2}}}}
%         {\fractype{\partial^2 #3}{\partial #4 \partial #5}}
%     }
% }
% \DeclareDocumentCommand\pderivative{}{\partialderivative} % Shorthand for \partialderivative
% \DeclareDocumentCommand\pdv{}{\partialderivative} % Shorthand for \partialderivative

\DeclareDocumentCommand\variation{ o g d() }{ % Functional variation
    % o: optional n for nth differential
    % g: optional argument for readability and to control spacing
    % d: long-form as in d(F(g))
    \IfNoValueTF{#2}{
        \IfNoValueTF{#3}
        {\delta \IfNoValueTF{#1}{}{^{#1}}}
        {\mathinner{\delta \IfNoValueTF{#1}{}{^{#1}}\argopen(#3\argclose)}}
    }
    {\mathinner{\delta \IfNoValueTF{#1}{}{^{#1}}#2} \IfNoValueTF{#3}{}{(#3)}}
}
\DeclareDocumentCommand\var{}{\variation} % Shorthand for \variation

\DeclareDocumentCommand\functionalderivative{ s o m g d() }
{ % Functional derivative
    % s: star for \flatfrac flat derivative
    % o: optional n for nth derivative
    % m: mandatory (g in dF/dg)
    % g: optional (F in dF/dg)
    % d: long-form d/dx(...)
    \IfBooleanTF{#1}
    {\let\fractype\flatfrac}
    {\let\fractype\frac}
    \IfNoValueTF{#4}
    {
        \IfNoValueTF{#5}
        {\fractype{\variation \IfNoValueTF{#2}{}{^{#2}}}{\variation #3\IfNoValueTF{#2}{}{^{#2}}}}
        {\fractype{\variation \IfNoValueTF{#2}{}{^{#2}}}{\variation #3\IfNoValueTF{#2}{}{^{#2}}} \argopen(#5\argclose)}
    }
    {\fractype{\variation \IfNoValueTF{#2}{}{^{#2}} #3}{\variation #4\IfNoValueTF{#2}{}{^{#2}}}}
}
\DeclareDocumentCommand\fderivative{}{\functionalderivative} % Shorthand for \functionalderivative
\DeclareDocumentCommand\fdv{}{\functionalderivative} % Shorthand for \functionalderivative


\begin{document}
\section*{fCM \hfill II}

\section{Complex Variable}
\begin{itemize}
\item Integral $F(z) = \int_C f(z,t) \odif{t}$ defines an analytic function when 
\begin{itemize}
    \item $f(z,t)$ continuous jointly in $z$ and $t$
    \item Integral converges uniformly in each compact set of the range of $z$
    \item $f(z,t)$ analytic in $z$ for each $t$
\end{itemize}
\item Analytic continuation, uniqueness, radius of convergence up to closest singularity; might have natural boundary
\item Cauchy Principal Value ($\mathcal{P}$): limit of definite integral to an undefined integral
\item Hilbert Transform: 
\[\mathcal{H}(f)(y) = \frac{1}{\pi} \mathcal{P} \left(\int_{-\infty}^\infty \frac{f(x)}{x-y} \odif{x}\right)\]
Property: \[\mathcal{H}(e^{i\omega x})(y) = \begin{cases}
    ie^{i\omega y} &, \omega> 0\\
    -ie^{i\omega y}&, \omega< 0\\
\end{cases} \]
So $\mathcal{H}^2 = -\text{Id}$.
\item Kramers Kronig Relations: for analytic $f(z) = u(x,y) + iv(x,y)$, (decaying quick enough as $z \to \infty$) \begin{align*}
    \mathcal{H}(u(x,0))&=-v(x,0) \\
    \mathcal{H}(v(x,0))&=u(x,0) \\
\end{align*}
\item Multivalued functions: branch point/cuts, defining $\arcsin$, $\ln$.
\item Elliptic function: doubly periodic meromorphic function, e.g. Weierstrass $\wp$ function

Finitely many zeroes and poles in a cell; if no poles, then is constant
\end{itemize}
\section{Special Functions}
\begin{itemize}
    \item Gamma: 
    \begin{align*}
        \Gamma(z) &= \int_0^\infty t^{z-1}e^{-t} \odif{t} &,\text{ for $\Re(z) > 0$ (Integral)} \\
        &= \lim_{n\to\infty} \frac{n! n^z}{z(z+1)\dots(z+n)} &,\text{ for $z\in \C \setminus \{0,-1,-2,\dots\}$ (Euler product)} \\
        &= \frac{1}{z}\prod_{n=1}^\infty \frac{(1+\frac{1}{n})^z}{1+\frac{z}{n}} &,\text{ for $z\in \C \setminus \{0,-1,-2,\dots\}$ (Gauss product)} \\
        &= \frac{1}{2i\sin\pi z} \int_{-\infty}^{(0^+)} t^{z-1}e^{t} \odif{t}  &,\text{ for $z\in \C \setminus \{0,-1,-2,\dots\}$ (Hankel representation)}  
    \end{align*}
    Weierstrass product (with $\gamma = \lim_{n\to\infty} \left(1+\frac{1}{2}\dots+\frac{1}{n}-\log n\right)$ )
    \[\frac{1}{\Gamma(z)} = ze^{\gamma z} \prod_{k=1}^\infty \left({1+\frac{z}{k}}\right) e^{-z/k}\]
    Reflection formula: 
    \[\Gamma(z)\Gamma(1-z) = \pi \text{cosec}(\pi z) \text{ for $z\not\in \Z$ } \]
    Generalised Factorial $\Gamma(n+1) = n!$: unique extension (with some extra properties)
    \item Beta \begin{align*}
        B(p,q) &= \int_0^1 t^{p-1}(1-t)^{q-1} \odif{t} \text{ for $\Re(p), \Re(q) > 0$} \\
        &= \frac{\Gamma(p)\Gamma(q)}{\Gamma(p+q)} \\
        &= \frac{-1}{4} e^{\pi i (p+q)} \text{cosec}(\pi p)\text{cosec}(\pi q) \int_{\text{Pochhammer}} t^{p-1}(1-t)^{q-1} \odif{t}
    \end{align*}
    \item Zeta 
    \begin{align*}
        \zeta(z) &= \sum_{n=1}^\infty n^{-z},  \text{ for $\Re(z)> 1$} \\
        &= \frac{\Gamma(1-z)}{2\pi i} \int_{-\infty}^{(0^+)} \frac{t^{z-1}}{e^{-t}-1} \odif{t} \\
        &= 2^z \pi^{z-1} \sin\left(\frac{\pi z}{2}\Gamma(1-z)\zeta(1-z) \right) 
        &= \prod_{p}\frac{1}{1-p^{-z}}, \text{ for $\Re(z)> 1$} 
    \end{align*}
    Trivial zeros $\zeta(n)=0$ at $n=-2,-4,-6,\dots$
\end{itemize}

\section{Solution of DEs by transform methods}
\begin{itemize}
    \item Integral transform $\int_\gamma K(z,t)f(t) \odif{t}$ with some kernel $K$, e.g. Laplace, Euler Mellin
    \item Choose function, then choose contour such that boundary terms vanish
    \item Laplace transform, Fourier Transform
    \item Note causality, stability (tend to zero for large $t$)
    \item Nyquist stability criterion (?)
\end{itemize}
\section{2nd order linear ODE in complex plane}
\begin{itemize}
    \item Classify ordinary points, regular singular points, irregular singular points
    \item Power series solution, indicial equation
    \item Fuchsian equation: at most 3 RSPs, transform
    \item Papperitz equation, transform rule (p.54), hypergeometric equation
    \item Hypergeometric function: 
    \[F(A,B;C;z) = \sum_{n=0}^\infty \frac{(A)_n (B)_n}{(C)_n n!}z^n\]
    where $(X)_n = (X)(X+1)(X+2)\dots (X+n-1)$
\end{itemize}
\end{document}