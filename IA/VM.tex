\documentclass[12pt, a4paper]{article}
\usepackage[T1]{fontenc}
\usepackage{amsmath, amssymb, amsthm}
\usepackage{babel}
\usepackage{siunitx}
\usepackage{tikz}
\usepackage{centernot}
\usepackage{tcolorbox}
\usepackage{cancel}
\usepackage{enumitem}
\usepackage{xparse}

\usetikzlibrary{arrows}

\setlength{\parindent}{0pt}


% Matrix groups
\newcommand{\GL}{\mathrm{GL}}
\newcommand{\M}{\mathrm{M}}
\newcommand{\Or}{\mathrm{O}}
\newcommand{\PGL}{\mathrm{PGL}}
\newcommand{\PSL}{\mathrm{PSL}}
\newcommand{\PSO}{\mathrm{PSO}}
\newcommand{\PSU}{\mathrm{PSU}}
\newcommand{\SL}{\mathrm{SL}}
\newcommand{\SO}{\mathrm{SO}}
\newcommand{\Spin}{\mathrm{Spin}}
\newcommand{\Sp}{\mathrm{Sp}}
\newcommand{\SU}{\mathrm{SU}}
\newcommand{\U}{\mathrm{U}}
\newcommand{\Mat}{\mathrm{Mat}}


% Special sets
\newcommand{\C}{\mathbb{C}}
\newcommand{\CP}{\mathbb{CP}}
\newcommand{\F}{\mathbb{F}}
\newcommand{\GG}{\mathbb{G}}
\newcommand{\N}{\mathbb{N}}
% \newcommand{\P}{\mathbb{P}}
\newcommand{\Q}{\mathbb{Q}}
\newcommand{\R}{\mathbb{R}}
\newcommand{\RP}{\mathbb{RP}}
\newcommand{\T}{\mathbb{T}}
\newcommand{\Z}{\mathbb{Z}}
\renewcommand{\H}{\mathbb{H}}

% Brackets
\renewcommand{\vec}[1]{\boldsymbol{\mathbf{#1}}}
\newcommand{\cis}[1]{ \cos\left( #1 \right) + i \sin \left( #1 \right)}

% Algebra
\DeclareMathOperator{\adj}{adj}
\DeclareMathOperator{\Ann}{Ann}
\DeclareMathOperator{\Aut}{Aut}
\DeclareMathOperator{\Char}{char}
\DeclareMathOperator{\disc}{disc}
\DeclareMathOperator{\dom}{dom}
\DeclareMathOperator{\fix}{fix}
\DeclareMathOperator{\Hom}{Hom}
\DeclareMathOperator{\id}{id}
\DeclareMathOperator{\image}{image}
\DeclareMathOperator{\im}{im}
\DeclareMathOperator{\tr}{tr}
\newcommand{\Bilin}{\mathrm{Bilin}}
\newcommand{\Frob}{\mathrm{Frob}}



\let\Im\relax
\let\Re\relax


\DeclareMathOperator{\hcf}{hcf}
\DeclareMathOperator{\Isom}{Isom}
\DeclareMathOperator{\lcm}{lcm}
\DeclareMathOperator{\sgn}{sgn}
\DeclareMathOperator{\supp}{supp}
\DeclareMathOperator{\Sym}{Sym}
\DeclareMathOperator{\Syl}{Syl}
\DeclareMathOperator{\Im}{Im}
\DeclareMathOperator{\Re}{Re}
\DeclareMathOperator{\Ker}{Ker}


% Theorems
\theoremstyle{definition}
\newtheorem*{aim}{Aim}
\newtheorem*{axiom}{Axiom}
\newtheorem*{claim}{Claim}
\newtheorem*{cor}{Corollary}
\newtheorem*{conjecture}{Conjecture}
\newtheorem*{defi}{Definition}
\newtheorem*{eg}{Example}
\newtheorem*{ex}{Exercise}
\newtheorem*{fact}{Fact}
\newtheorem*{law}{Law}
\newtheorem*{lemma}{Lemma}
\newtheorem*{notation}{Notation}
\newtheorem*{prop}{Proposition}
\newtheorem*{question}{Question}
\newtheorem*{rrule}{Rule}
\newtheorem*{thm}{Theorem}
\newtheorem*{assumption}{Assumption}

\newtheorem*{remark}{Remark}
\newtheorem*{warning}{Warning}
\newtheorem*{exercise}{Exercise}

\newtheorem{nthm}{Theorem}[section]
\newtheorem{nlemma}[nthm]{Lemma}
\newtheorem{nprop}[nthm]{Proposition}
\newtheorem{ncor}[nthm]{Corollary}


\newcommand{\abs}[1]{\left| #1 \right|} % for absolute value
\newcommand{\grad}[1]{\mathbf{\nabla} #1} % for gradient
\let\divsymb=\div % rename builtin command \div to \divsymb
\renewcommand{\div}[1]{\mathbf{\nabla} \cdot #1} % for divergence
\newcommand{\curl}[1]{\mathbf{\nabla} \times #1} % for curl




% Derivatives

\newcommand{\dd}[1][]{\mathrm{d} #1}
\newcommand{\odiff}[1]{\frac{\dd}{\dd{#1}}}
\newcommand{\odv}[2]{\frac{\dd{#1}}{\dd{#2}}}
\newcommand{\pdiff}[1]{\frac{\partial}{\partial{#1}}}
\newcommand{\pdv}[2]{\frac{\partial{#1}}{\partial{#2}}}


% \def\diffd{\mathrm{d}}
% \DeclareDocumentCommand\differential{ o g d() }{ % Differential 'd'
%     % o: optional n for nth differential
%     % g: optional argument for readability and to control spacing
%     % d: long-form as in d(cos x)
%     \IfNoValueTF{#2}{
%         \IfNoValueTF{#3}
%         {\diffd\IfNoValueTF{#1}{}{^{#1}}}
%         {\mathinner{\diffd\IfNoValueTF{#1}{}{^{#1}}\argopen(#3\argclose)}}
%     }
%     {\mathinner{\diffd\IfNoValueTF{#1}{}{^{#1}}#2} \IfNoValueTF{#3}{}{(#3)}}
% }
% \DeclareDocumentCommand\dd{}{\differential} % Shorthand for \differential

% \DeclareDocumentCommand\derivative{ s o m g d() }
% { % Total derivative
%     % s: star for \flatfrac flat derivative
%     % o: optional n for nth derivative
%     % m: mandatory (x in df/dx)
%     % g: optional (f in df/dx)
%     % d: long-form d/dx(...)
%     \IfBooleanTF{#1}
%     {\let\fractype\flatfrac}
%     {\let\fractype\frac}
%     \IfNoValueTF{#4}
%     {
%         \IfNoValueTF{#5}
%         {\fractype{\diffd \IfNoValueTF{#2}{}{^{#2}}}{\diffd #3\IfNoValueTF{#2}{}{^{#2}}}}
%         {\fractype{\diffd \IfNoValueTF{#2}{}{^{#2}}}{\diffd #3\IfNoValueTF{#2}{}{^{#2}}} \argopen(#5\argclose)}
%     }
%     {\fractype{\diffd \IfNoValueTF{#2}{}{^{#2}} #3}{\diffd #4\IfNoValueTF{#2}{}{^{#2}}}}
% }
% \DeclareDocumentCommand\dv{}{\derivative} % Shorthand for \derivative

% \DeclareDocumentCommand\partialderivative{ s o m g g d() }
% { % Partial derivative
%     % s: star for \flatfrac flat derivative
%     % o: optional n for nth derivative
%     % m: mandatory (x in df/dx)
%     % g: optional (f in df/dx)
%     % g: optional (y in d^2f/dxdy)
%     % d: long-form d/dx(...)
%     \IfBooleanTF{#1}
%     {\let\fractype\flatfrac}
%     {\let\fractype\frac}
%     \IfNoValueTF{#4}
%     {
%         \IfNoValueTF{#6}
%         {\fractype{\partial \IfNoValueTF{#2}{}{^{#2}}}{\partial #3\IfNoValueTF{#2}{}{^{#2}}}}
%         {\fractype{\partial \IfNoValueTF{#2}{}{^{#2}}}{\partial #3\IfNoValueTF{#2}{}{^{#2}}} \argopen(#6\argclose)}
%     }
%     {
%         \IfNoValueTF{#5}
%         {\fractype{\partial \IfNoValueTF{#2}{}{^{#2}} #3}{\partial #4\IfNoValueTF{#2}{}{^{#2}}}}
%         {\fractype{\partial^2 #3}{\partial #4 \partial #5}}
%     }
% }
% \DeclareDocumentCommand\pderivative{}{\partialderivative} % Shorthand for \partialderivative
% \DeclareDocumentCommand\pdv{}{\partialderivative} % Shorthand for \partialderivative

\DeclareDocumentCommand\variation{ o g d() }{ % Functional variation
    % o: optional n for nth differential
    % g: optional argument for readability and to control spacing
    % d: long-form as in d(F(g))
    \IfNoValueTF{#2}{
        \IfNoValueTF{#3}
        {\delta \IfNoValueTF{#1}{}{^{#1}}}
        {\mathinner{\delta \IfNoValueTF{#1}{}{^{#1}}\argopen(#3\argclose)}}
    }
    {\mathinner{\delta \IfNoValueTF{#1}{}{^{#1}}#2} \IfNoValueTF{#3}{}{(#3)}}
}
\DeclareDocumentCommand\var{}{\variation} % Shorthand for \variation

\DeclareDocumentCommand\functionalderivative{ s o m g d() }
{ % Functional derivative
    % s: star for \flatfrac flat derivative
    % o: optional n for nth derivative
    % m: mandatory (g in dF/dg)
    % g: optional (F in dF/dg)
    % d: long-form d/dx(...)
    \IfBooleanTF{#1}
    {\let\fractype\flatfrac}
    {\let\fractype\frac}
    \IfNoValueTF{#4}
    {
        \IfNoValueTF{#5}
        {\fractype{\variation \IfNoValueTF{#2}{}{^{#2}}}{\variation #3\IfNoValueTF{#2}{}{^{#2}}}}
        {\fractype{\variation \IfNoValueTF{#2}{}{^{#2}}}{\variation #3\IfNoValueTF{#2}{}{^{#2}}} \argopen(#5\argclose)}
    }
    {\fractype{\variation \IfNoValueTF{#2}{}{^{#2}} #3}{\variation #4\IfNoValueTF{#2}{}{^{#2}}}}
}
\DeclareDocumentCommand\fderivative{}{\functionalderivative} % Shorthand for \functionalderivative
\DeclareDocumentCommand\fdv{}{\functionalderivative} % Shorthand for \functionalderivative


\begin{document}

\section*{Vectors and Matrices \hfill IA Mich}

Read hermitian matrix diagonalisable proof, and supo notes

\section{Complex numbers}
\begin{itemize}
      \item Definitions
            \begin{itemize}
                  \item Construct $\mathbb{C}\textrm{ from \ensuremath{\mathbb{R}} with element \ensuremath{i} such that }i^{2}=-1$
                  \item Addition, multiplication, Complex conjugate $(\bar{z}\textrm{ or }z^{*})$
                  \item Modulus, argument (multi-valued, restrict using principal value),
                        Argand diagram
            \end{itemize}
      \item Properties
            \begin{itemize}
                  \item $\mathbb{C}$ is a field (with $+ - \times \div $, associative, commutative,
                        distributive)
                  \item Fundamental Theorem of Algebra
                  \item Geometric representation of addition, subtraction, complex conjugation
                  \item Composition property: $|z_{1}z_{2}|=|z_{1}||z_{2}|$
                  \item Triangle inequality: $|z_{1}+z_{2}|\leq|z_{1}|+|z_{2}|$, and alternatively
                        $|z_{1}-z_{2}|\geq\left||z_{1}|-|z_{2}|\right|$
                  \item Multiplying complex numbers: moduli multiply, arguments add
                  \item De Moivre's Theorem
            \end{itemize}
      \item Exponential and Trig functions
            \begin{itemize}
                  \item Define $\exp(z)=\sum_{n=0}^{\infty}\frac{z^{n}}{n!}$, $\sin(z),\cos(z)$
                        in terms of $e^{z}$, consistent with usual rules, thus can have exponential
                        form of complex numbers
                  \item Roots of unity: $z^{n}=1\iff z=\exp(\frac{2k\pi i}{n}),k=0,1,...,n-1$
            \end{itemize}
      \item Logarithm and Complex Powers
            \begin{itemize}
                  \item Define $\log$ as inverse of exp
                  \item $z=|z|e^{i\arg(z)}\implies\log(z)=\log|z|+i\arg(z)$ (multi-valued
                        as $\arg(z)$ is multi-valued, can restrict using principal value)
                  \item Complex power: $z^{\alpha}=e^{\alpha\log(z)}$, generally infinitely
                        many values, may be finite in special cases (integer/rational exponent)
            \end{itemize}
      \item Lines and Circles
            \begin{itemize}
                  \item Line: parametric (point and direction), and non-parametric (eliminate
                        by taking complex conjugate)
                  \item Circles: $|z-c|=\rho$ (or expanded) or $z=c+\rho e^{i\theta}$
            \end{itemize}
      \item Geometric transformations described in $\mathbb{C}$: translation,
            scaling, rotation, reflection, inversion; relation to Möbius transformation
\end{itemize}

\section{Vectors in 3D}
\begin{itemize}
      \item Euclidean space: vectors have direction and magnitudes
      \item Addition and scalar multiplication (Geometric)
            \begin{itemize}
                  \item Parallelogram construction, scalar multiplication by extending
                  \item Properties: identity, inverses, commutativity, associativity, distributivity
                  \item Linear combination and span of vectors
                  \item Parallel: one vector being a scalar multiple of the other, otherwise
                        the two vectors span a plane (allow $\mathbf{0\parallel a}$ for all
                        $\mathbf{a}$ in this definition)
            \end{itemize}
      \item Scalar/dot/inner product (Geometric)
            \begin{itemize}
                  \item Definition: $\mathbf{a\cdot b}=|\mathbf{a}||\mathbf{b}|\cos\theta$,
                        using magnitude and angle between the vectors
                  \item Properties: symmetric, positive definite, scalar multiple can 'move
                        around', distributive
                  \item Orthogonality: $\mathbf{a\perp b}\iff\mathbf{a\cdot b}=0$ (allow
                        $\mathbf{0\perp a}$ for all $\mathbf{a}$ in this definition)
                  \item Resolving a vector along another one: component of $\mathbf{b}$ along $\mathbf{a}$ is given by (in magnitude)
                        \[\mathbf{|b|}\cos\theta=\mathbf{\frac{a\cdot b}{|a|}} \]
            \end{itemize}
      \item Orthonormal Bases
            \begin{itemize}
                  \item A set of orthonormal vectors $\mathbf{e}_{1},\mathbf{e}_{2},\mathbf{e}_{3}$(or
                        $\mathbf{\hat{i},\hat{j},\hat{k}}$) are each of unit length and mutually
                        perpendicular: $\mathbf{e}_{i}\cdot\mathbf{e}_{j}=\delta_{ij}$
                  \item Equivalent to choosing Cartesian axes along the 3 directions
                  \item Basis $\{\mathbf{e}_{i}\}$: linearly independent and span: Any vector
                        can be represented uniquely by a linear combination of the three vectors:
                        ${\displaystyle \mathbf{a}=\sum_{i}a_{i}\mathbf{e}_{i}}$, components
                        determined by dotting $\mathbf{a}$ with each of the basis vectors
                  \item Notion of components: $(a_{1},a_{2},a_{3})$ or $\left(\begin{array}{c}
                                          a_{1} \\
                                          a_{2} \\
                                          a_{3}
                                    \end{array}\right)$
            \end{itemize}
      \item Vector/Cross product (Geometric)
            \begin{itemize}
                  \item Definition: $\mathbf{a\times b}=|\mathbf{a}||\mathbf{b}|\sin\theta\mathbf{\hat{n}},$where
                        $\mathbf{\hat{n}}$ (unit normal vector) and $\theta$ are defined
                        in a right-handed sense
                  \item Signed area of parallelogram, with sign giving orientation
                  \item Can be seen as rotating and scaling a vector at the same time (p.15)
                  \item Define a orthonormal right-handed set by cross product, then cross
                        products can be expressed in components
            \end{itemize}
      \item Triple Products
            \begin{itemize}
                  \item Scalar triple product: signed volume of parallelepiped, test for coplanar
                        vectors
                  \item Vector triple product: $\mathbf{a\times}(\mathbf{b}\times\mathbf{c})=(\mathbf{a\cdot c})\mathbf{b}-(\mathbf{a}\cdot\mathbf{b})\mathbf{c}$
            \end{itemize}
      \item Lines, planes and vector equations
            \begin{itemize}
                  \item Line: $\mathbf{r}=\mathbf{a}+\lambda\mathbf{u}$ (parametric form);
                        $\mathbf{u}\times\mathbf{r}=\mathbf{c}=$ constant (non-parametric)
                  \item Plane: $\mathbf{r}=\mathbf{a}+\lambda\mathbf{u}+\lambda\mathbf{v}$,
                        with $\mathbf{u}$ and $\mathbf{v}$ non-parallel in the plane (parametric
                        form); $\mathbf{n}\cdot\mathbf{r}=\mathbf{n\cdot a=\kappa}$, with
                        $\mathbf{n}$ normal vector to the plane;
                  \item Perpendicular distance of plane from origin $={\displaystyle \frac{|\kappa|}{|\mathbf{n}|}}$
                  \item Vector equations: eliminate $\mathbf{r}$ by taking products; consider
                        cases; let $\mathbf{r}$ be a linear combination of constant vectors
                        and solve for coefficients, complete square
            \end{itemize}
      \item Index notation and summation convention (Algebraic)
            \begin{itemize}
                  \item Free indices appear exactly once in every term: represent components;
                        Dummy indices appear exactly twice in each term, and are summed up;
                  \item Nothing should appear more than twice
                  \item Kronecker Delta $\delta_{ij}$: represent identity matrix $I$
                  \item Levi-Civita Epsilon $\epsilon_{ijk}$: antisymmetric, commonly used
                        in cross product, determinant
                  \item Identities: $\epsilon_{ijk}\epsilon_{pqk}=\delta_{ip}\delta_{jq}-\delta_{iq}\delta_{jp}$,
                        ``same minus different''
            \end{itemize}
\end{itemize}

\section{Vectors in general, $\mathbb{R}^{n}$ and $\mathbb{C}^{n}$ (Algebraic)}
\begin{itemize}
      \item $\mathbb{R}^{n}$: set of real n-tuples, form linear combination by
            addition and (real) scalar multiplication
            \begin{itemize}
                  \item Component expression correspond to a standard basis
                  \item Inner product on $\mathbb{R}^{n}$: symmetric, bilinear (linear in
                        both arguments), positive definite ($\mathbf{x}\cdot\mathbf{x}\geq0,\textrm{ equality holds\ensuremath{\iff}}\mathbf{x}=\mathbf{0})$
                  \item Parallel, orthogonality, and (Euclidean) norm of vectors
                  \item Cauchy-Schwarz inequality: $|\mathbf{x\cdot y}|\leq|\mathbf{x}||\mathbf{y}|$
                        for $\mathbf{x},\mathbf{y}\in\mathbb{R}^{n}$, with equality iff $\mathbf{x}=\lambda\mathbf{y}$
                        or $\mathbf{y}=\lambda\mathbf{x}$ for $\lambda\in\mathbb{R}$; thus
                        can define angle between vectors in $\mathbb{R}^{n}$ by inner product
                  \item Triangle inequality: $|\mathbf{x+y}|\leq|\mathbf{x}|+|\mathbf{y}|$
                  \item Vector inner product as matrix multiplication: $\mathbf{x\cdot y}=\mathbf{x}^{T}\mathbf{y}$
                  \item Inner product carries to higher dimensions from $\mathbb{R}^{3}$,
                        but cross product cannot (due to $\epsilon_{ij...k}$ with too many
                        free indices)
                  \item Instead, $\epsilon_{ij}x_{i}y_{j}$ defines an alternative scalar
                        product in 2D, which generalises to scalar triple product in 3D
            \end{itemize}
      \item Vector spaces
            \begin{itemize}
                  \item Defining vector spaces with axioms that enable addition, scalar multiplication
                        (e.g. distributive, associative, identity)
                  \item For a set of vectors $\mathbf{v}_{i}\in V$, a vector space, define
                        $\textrm{span}\{\mathbf{v}_{i}\}=\{\sum_{i}\lambda_{i}\mathbf{v}_{i}\}$,
                        the set of all linear combinations
                  \item The span of a set of vectors forms a subspace of $V$ (a subset and
                        itself a vector space); a subspace must contain $\mathbf{0}$
                  \item Linearly independent: $\sum_{i}\lambda_{i}\mathbf{v}_{i}=\mathbf{0}\implies\lambda_{i}=0$
                        for all $i$
                  \item Linear dependence: $\sum_{i}\lambda_{i}\mathbf{v}_{i}=\mathbf{0}$
                        has non-trivial solutions; i.e. some vector in the set can be written
                        as a linear combination of the others
                  \item Inner products obey the three properties: symmetric, bilinear, positive
                        definite
            \end{itemize}
      \item Bases and dimension
            \begin{itemize}
                  \item Basis: linearly independent and spans the vector space $V$ ; thus
                        each point in $V$ has a unique linear combination
                  \item Any set of bases of $V$ has the same number of elements, defined
                        to be $\dim V$
                  \item A spanning set with more vectors than $\dim V$ is not linearly dependent
                        (can remove one); a linearly independent set with less vectors than
                        $\dim V$ does not span (can add one)
                  \item A set of mutually orthogonal non-zero vectors is linearly independent
                        (dot with each vector, everything else vanish)
            \end{itemize}
      \item Vectors in $\mathbb{C}^{n}$
            \begin{itemize}
                  \item $\mathbb{C}^{n}$: Vector space with linear combinations (complex
                        scalar multiple), thus has $n$ basis vectors
                  \item Inner product: $(\mathbf{z},\mathbf{w})=\sum_{j}\bar{z_{j}}w_{j}$
                  \item Inner product as matrix multiplication: $(\mathbf{z},\mathbf{w})=\mathbf{z^{\dagger}}\mathbf{w}$
                  \item Properties: hermitian ($(\mathbf{z},\mathbf{w}) = \overline{(\mathbf{w},\mathbf{z})}$), anti-linear/linear (in first/second argument),
                        positive definite; scalar products in $\mathbb{R}^{2}$ recovered
                        in inner product on $\mathbb{C}$
                  \item Define length/norm, orthogonality, orthonormal basis, linear independence
                        (where $\mathbf{z^{\dagger}}=\bar{\mathbf{z}}^T$, a row vector
                        with components $\bar{z_{i}}$)
            \end{itemize}
\end{itemize}

\section{Matrices and Linear Map}
\begin{itemize}
      \item Linear transformation/map
            \begin{itemize}
                  \item A function $T:V\to W$, between vector spaces $V$ (with dim $n$)
                        and $W$ (with dim $m$): satisfying $T(\lambda\mathbf{x}+\mu\mathbf{y})=\lambda T(\mathbf{x})+\mu T(\mathbf{y})$,
                        where $\lambda$ and $\mu$ real/complex (a field in general)
                  \item Image, kernel, rank, nullity, rank-nullity theorem
                  \item Linear combination and composition of linear maps
            \end{itemize}
      \item Matrices as linear maps
            \begin{itemize}
                  \item Matrices define a transformation by $x'_{i}=M_{ij}x_{j}$ (summation)
                  \item The i-th basis vector is mapped to the i-th column; image of matrix
                        is the span of columns
                  \item $x'_{i}=(\mathbf{R}_{i})_{j}x_{j}=\mathbf{R}_{i}\cdot\mathbf{x}$:
                        dotting the i-th row with $\mathbf{x}$ gives the i-th component of
                        $\mathbf{x'}$; kernel of matrix is $\{\mathbf{x}:\mathbf{R}_{i}\cdot\mathbf{x}=0\textrm{ for all i}\}$
            \end{itemize}
      \item Geometric examples of linear map: $\mathbb{R}^{3}\to\mathbb{R}^{3}$
            \begin{itemize}
                  \item Rotation (about a unit vector $\mathbf{\hat{n}}$)
                  \item Reflection (in a plane through $\mathbf{0}$ with unit normal $\mathbf{\hat{n}}$)
                  \item Dilation (along three axes)
                  \item Shear
            \end{itemize}
      \item Matrices in general
            \begin{itemize}
                  \item Given vector spaces $V$ (dim $n$) with basis $\{\mathbf{e_{1}},\mathbf{e_{2}},...,\mathbf{e_{n}}\}$,
                        and $W$ (dim $m$) with basis $\{\mathbf{f_{1}},\mathbf{f_{2}},...,\mathbf{f_{m}}\}$,
                        the linear map $T:V\to W$ w.r.t. to the given bases can be represented
                        by an $n\times m$ matrix $M$, where $T(\mathbf{e_{i}})=\sum_{a}M_{ai}\mathbf{f_{a}}$,
                        i.e. image of $\mathbf{e_{i}}$ is the $i$-th column of $M$ with
                        respect to basis $\{\mathbf{f_{a}}\}$
                  \item $T(\mathbf{x})_{i}=\sum_{j}M_{ij}x_{j}=$ dot product of $i$-th row
                        of $M$ with $\mathbf{x}$
                  \item Given bases of each vector space, vectors and transformation become
                        components and matrices ($\mathbb{F}^{n},\mathbb{F}^{m}$ and $M_{n\times m}(\mathbb{F})$)
                  \item Matrix addition, scalar multiplication and matrix multiplication represent
                        adding, scaling and composition of linear maps
                  \item Invertible/non-singular matrices: left/right inverse, same inverse
                        for square matrices
            \end{itemize}
      \item Operation and properties of matrices
            \begin{itemize}
                  \item Transpose and hermitian conjugate: (need not be square matrix) properties
                        under adding, scaling and product of matrices
                  \item Symmetric ($A^{T}=A$), anti-symmetric ($A^{T}=-A$), hermitian ($A^{\dagger}=A$),
                        anti-hermitian ($A^{\dagger}=-A$)
                  \item Trace, decomposition of square matrices into antisymmetric + symmetric
                        traceless + isotropic (multiple of identity)
            \end{itemize}
      \item Orthogonal and Unitary matrices
            \begin{itemize}
                  \item Orthogonal matrix: $A^{T}=A^{-1}$, inner product preserved, rows
                        orthonormal, columns orthonormal
                  \item General $2\times2$ orthogonal matrices: rotation or reflection (distinguished
                        by determinant of matrix)
                  \item Unitary matrix: $A^{\dagger}=A^{-1}$, complex inner product preserved
            \end{itemize}
\end{itemize}

\section{Determinants and Inverses}
\begin{itemize}
      \item Determinant
            \begin{itemize}
                  \item Factor by which the signed (n-dim) volume is scaled by the transformation
                  \item $\det A\neq0\iff$ linear dependence
                  \item Sum over indices: ${\displaystyle \det A=\sum_{i_{1},i_{2},...,i_{n}}\varepsilon_{i_{1}i_{2}...i_{n}}A_{i_{1}1}A_{i_{2}2}...A_{i_{n}n}}$,
                        extended from 3D case
                  \item Sum over permutations in $S_{n}$:${\displaystyle \det A=\sum_{\sigma\in S_{n}}\varepsilon(\sigma)}A_{\sigma(1)1}A_{\sigma(2)2}...A_{\sigma(n)n}$,
                        sign of permutation $\varepsilon(\sigma)$
                  \item Scaling a row/column, scaling the whole matrix, row and column operations/exchanges,
                        transpose property, multiplicative property
                  \item Expand along column or row
            \end{itemize}
      \item Minors, cofactors and inverses
            \begin{itemize}
                  \item Minor: $M^{ij}=$ det of matrix $M$ with $i$-th row and $j$-th
                        col deleted
                  \item Cofactor of $\Delta_{ij}=(-1)^{i+j}M^{ij}$ (signed minor)
                  \item Adjugate: transpose of the cofactor matrix, so $M^{-1}=\frac{1}{\det M}\Delta^{T}$
            \end{itemize}
      \item System of linear equations $A\mathbf{x}=\mathbf{b}$, where $\mathbf{x},\boldsymbol{b}\in\mathbb{R}^{n}$,
            and $A\in M_{n\times n}(\mathbb{R})$
            \begin{itemize}
                  \item Existence and nature of solutions:
                        \begin{itemize}
                              \item $\det A\neq0\iff\mathtt{\textrm{Im}}A=\mathbb{R}^{n}\textrm{\ensuremath{\iff}}\textrm{Ker}A=\{\mathbf{0}\}\iff$
                                    unique solution
                              \item $\det A=0\textrm{ and }\mathbf{b}\in\mathtt{\textrm{Im}}A\iff$ infinitely
                                    many solution
                              \item $\det A=0\textrm{ and }\mathbf{b}\notin\mathtt{\textrm{Im}}A\iff$
                                    no solutions
                              \item General solution (if there exists particular solution $\mathbf{x_{0}}$)
                                    is $\mathbf{x}=\mathbf{x_{0}}+\textrm{Ker}A$
                        \end{itemize}
                  \item Can interpret each equation as a (hyper-)plane in $\mathbb{R}^{n}$,
                        intersection is the set of solutions
                        \begin{itemize}
                              \item Homogeneous equation: each plane passes through origin, so must have
                                    at least one solution: $\mathbf{0}$
                              \item Inhomogeneous: may not pass through origin, solution not guaranteed
                        \end{itemize}
                  \item Gaussian Elimination: reduce to echelon form, (variables that are
                        not pivot columns have 'degree of freedom')
            \end{itemize}
\end{itemize}

\section{Eigenvalues and eigenvectors}
\begin{itemize}
      \item Introduction
            \begin{itemize}
                  \item Given a linear map $T:V\to V$, an eigenvector $\mathbf{v}\in V\backslash\{\mathbf{0}\}$
                        with eigenvalue $\lambda\in\mathbb{F}$ (where $V$ is defined over)
                        satisfies $T(\mathbf{v})=\lambda\mathbf{v}$
                  \item Given a basis, can express the map with a matrix $A$, then have $A\mathbf{v}=\lambda\mathbf{v}\iff\chi_{A}(\lambda)=\det(A-\lambda I)=0$
                        (characteristic polynomial)
                  \item $\chi_{A}(\lambda)=0$ has $n$ roots (counted with multiplicity)
                        in $\mathbb{C}$, by FTA; deduce $\textrm{tr}A$, $\det A$ from the
                        equation
                  \item Eigenspace $E_{\lambda}$, geometric multiplicity $m_{\lambda}$,
                        algebraic multiplicity $M_{\lambda}$, defect $\Delta_{\lambda}$
                  \item Eigenvectors with distinct eigenvalues are linearly independent
            \end{itemize}
      \item Diagonalisability
            \begin{itemize}
                  \item Diagonalisable: eigenvectors form a basis $\iff$ there exists invertible
                        matrix $P$ such that $P^{-1}AP=D$, with entries being the eigenvalues
                  \item Criteria:
                        \begin{itemize}
                              \item Sufficient but not necessary: an $n\times n$ matrix has $n$ distinct
                                    eigenvalues $\implies$ form a basis
                              \item Sufficient and necessary: $M_{\lambda}=m_{\lambda}\textrm{ for each }\lambda$
                                    ($n$ linearly independent eigenvectors, form a basis)
                        \end{itemize}
            \end{itemize}
      \item Similar matrix
            \begin{itemize}
                  \item $A$ and $B$ are similar $\iff\exists$ invertible $P$ s.t. $B=P^{-1}AP$
                        (all n-dimensional), equivalence class
                  \item Similar matrices have same trace, determinant, characteristic polynomial
            \end{itemize}
      \item Hermitian and symmetric matrices
            \begin{itemize}
                  \item Hermitian matrices have real eigenvalues, eigenvectors with distinct
                        eigenvalues are orthogonal (complex inner product)
                  \item Real symmetric: can choose real eigenvectors for each real eigenvalue,
                        the eigenvectors are orthogonal (real inner product)
                  \item Gram-Schmidt Orthogonalisation: transform any basis into orthonormal
                        basis (e.g. the union of bases of eigenspaces with distinct eigenvalues
                        can be orthogonalised)
                  \item Any hermitian (real symmetric) matrix is unitarily (orthogonally)
                        diagonalisable: can choose orthonormal set of eigenvectors , i.e.
                        $P$ can be chosen to be unitary (orthogonal)
                        % (read proof on moodle)
            \end{itemize}
      \item Cayley-Hamilton Theorem: any matrix satisfies its own characteristic
            equation
\end{itemize}

\section{Change of Bases, Canonical form and Symmetries}
\begin{itemize}
      \item Change of bases:
            \begin{itemize}
                  \item Transformation matrix: the $i$-th column of P is the $i$-th vector
                        in the new basis w.r.t. the old basis: $\mathbf{e'_{i}}=\sum_{j}P_{ji}\mathbf{e_{j}}$
                  \item $A'=P^{-1}AP$: similar matrices represent same linear map w.r.t.
                        different bases
            \end{itemize}
      \item Jordan Canonical Form:
            \begin{itemize}
                  \item Classify square matrices by similarity: 2D case classify by eigenvalue
                        and multiplicities
            \end{itemize}
      \item Quadratic Forms
            \begin{itemize}
                  \item Given real-symmetric matrix $A$: the quadratic form is $\mathcal{F}(\mathbf{x})=\mathbf{x}^{T}A\mathbf{x}=x_{i}A_{ij}x_{j}$
                  \item Real-symmetric matrix can be orthogonally diagonalised (by above)
                        with $D=P^{-1}AP$, i.e. has an orthonormal basis of eigenvectors,
                        the principal axes of $\mathcal{F}$
                  \item Using coordinates w.r.t. the principal axes (i.e. letting $\mathbf{x}=P\mathbf{x'}$),
                        then $\mathcal{F}(\mathbf{x})=\mathbf{x'}^{T}D\mathbf{x'}=\sum_{i}\lambda_{i}x'^{2}_{i}$
                  \item Signs of eigenvalues and value of $\mathcal{F}(\mathbf{x})$ determine
                        shape
            \end{itemize}
      \item Quadrics and conics
            \begin{itemize}
                  \item General quadric: $Q(\mathbf{x})=\mathbf{x}^{T}A\mathbf{x}+\mathbf{b}^{T}\mathbf{x}+c=A_{ij}x_{i}x_{j}+b_{i}x_{i}+c=0$,
                        defines a surface in $\mathbb{R}^{n}$
                  \item Completing the square gives quadratic form
                  \item Conics: classify up to translation and orthogonal transformation (diagonalising)
            \end{itemize}
      \item Symmetries and Transformation Groups
            \begin{itemize}
                  \item Orthogonal and special orthogonal group $O(n),SO(n)$: length, volume
                        (and orientation) preserving
                  \item Can view as transforming vectors or changing orthonormal bases
            \end{itemize}
      \item 2D Minkowski Space and Lorentz Transformations
            \begin{itemize}
                  \item Define new inner product: Minkowski metric, $\mathbb{R}^{n}$ becomes
                        Minkowski space
                  \item Minkowski metric preserved by linear maps, subgroup of matrices with
                        $\det M>0\textrm{ and }M_{00}>0$ form the Lorentz group
                  \item General form of matrices in terms of hyperbolic trig, matrix map corresponds
                        to Lorentz transformation, speed of light being maximum possible speed
            \end{itemize}
\end{itemize}

\end{document}
