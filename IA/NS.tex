\documentclass[12pt, a4paper]{article}
\usepackage[T1]{fontenc}
\usepackage{amsmath, amssymb, amsthm}
\usepackage{babel}
\usepackage{siunitx}
\usepackage{tikz}
\usepackage{centernot}
\usepackage{tcolorbox}
\usepackage{cancel}
\usepackage{enumitem}
\usepackage{xparse}

\usetikzlibrary{arrows}

\setlength{\parindent}{0pt}


% Matrix groups
\newcommand{\GL}{\mathrm{GL}}
\newcommand{\M}{\mathrm{M}}
\newcommand{\Or}{\mathrm{O}}
\newcommand{\PGL}{\mathrm{PGL}}
\newcommand{\PSL}{\mathrm{PSL}}
\newcommand{\PSO}{\mathrm{PSO}}
\newcommand{\PSU}{\mathrm{PSU}}
\newcommand{\SL}{\mathrm{SL}}
\newcommand{\SO}{\mathrm{SO}}
\newcommand{\Spin}{\mathrm{Spin}}
\newcommand{\Sp}{\mathrm{Sp}}
\newcommand{\SU}{\mathrm{SU}}
\newcommand{\U}{\mathrm{U}}
\newcommand{\Mat}{\mathrm{Mat}}


% Special sets
\newcommand{\C}{\mathbb{C}}
\newcommand{\CP}{\mathbb{CP}}
\newcommand{\F}{\mathbb{F}}
\newcommand{\GG}{\mathbb{G}}
\newcommand{\N}{\mathbb{N}}
% \newcommand{\P}{\mathbb{P}}
\newcommand{\Q}{\mathbb{Q}}
\newcommand{\R}{\mathbb{R}}
\newcommand{\RP}{\mathbb{RP}}
\newcommand{\T}{\mathbb{T}}
\newcommand{\Z}{\mathbb{Z}}
\renewcommand{\H}{\mathbb{H}}

% Brackets
\renewcommand{\vec}[1]{\boldsymbol{\mathbf{#1}}}
\newcommand{\cis}[1]{ \cos\left( #1 \right) + i \sin \left( #1 \right)}

% Algebra
\DeclareMathOperator{\adj}{adj}
\DeclareMathOperator{\Ann}{Ann}
\DeclareMathOperator{\Aut}{Aut}
\DeclareMathOperator{\Char}{char}
\DeclareMathOperator{\disc}{disc}
\DeclareMathOperator{\dom}{dom}
\DeclareMathOperator{\fix}{fix}
\DeclareMathOperator{\Hom}{Hom}
\DeclareMathOperator{\id}{id}
\DeclareMathOperator{\image}{image}
\DeclareMathOperator{\im}{im}
\DeclareMathOperator{\tr}{tr}
\newcommand{\Bilin}{\mathrm{Bilin}}
\newcommand{\Frob}{\mathrm{Frob}}



\let\Im\relax
\let\Re\relax


\DeclareMathOperator{\hcf}{hcf}
\DeclareMathOperator{\Isom}{Isom}
\DeclareMathOperator{\lcm}{lcm}
\DeclareMathOperator{\sgn}{sgn}
\DeclareMathOperator{\supp}{supp}
\DeclareMathOperator{\Sym}{Sym}
\DeclareMathOperator{\Syl}{Syl}
\DeclareMathOperator{\Im}{Im}
\DeclareMathOperator{\Re}{Re}
\DeclareMathOperator{\Ker}{Ker}


% Theorems
\theoremstyle{definition}
\newtheorem*{aim}{Aim}
\newtheorem*{axiom}{Axiom}
\newtheorem*{claim}{Claim}
\newtheorem*{cor}{Corollary}
\newtheorem*{conjecture}{Conjecture}
\newtheorem*{defi}{Definition}
\newtheorem*{eg}{Example}
\newtheorem*{ex}{Exercise}
\newtheorem*{fact}{Fact}
\newtheorem*{law}{Law}
\newtheorem*{lemma}{Lemma}
\newtheorem*{notation}{Notation}
\newtheorem*{prop}{Proposition}
\newtheorem*{question}{Question}
\newtheorem*{rrule}{Rule}
\newtheorem*{thm}{Theorem}
\newtheorem*{assumption}{Assumption}

\newtheorem*{remark}{Remark}
\newtheorem*{warning}{Warning}
\newtheorem*{exercise}{Exercise}

\newtheorem{nthm}{Theorem}[section]
\newtheorem{nlemma}[nthm]{Lemma}
\newtheorem{nprop}[nthm]{Proposition}
\newtheorem{ncor}[nthm]{Corollary}


\newcommand{\abs}[1]{\left| #1 \right|} % for absolute value
\newcommand{\grad}[1]{\mathbf{\nabla} #1} % for gradient
\let\divsymb=\div % rename builtin command \div to \divsymb
\renewcommand{\div}[1]{\mathbf{\nabla} \cdot #1} % for divergence
\newcommand{\curl}[1]{\mathbf{\nabla} \times #1} % for curl




% Derivatives

\newcommand{\dd}[1][]{\mathrm{d} #1}
\newcommand{\odiff}[1]{\frac{\dd}{\dd{#1}}}
\newcommand{\odv}[2]{\frac{\dd{#1}}{\dd{#2}}}
\newcommand{\pdiff}[1]{\frac{\partial}{\partial{#1}}}
\newcommand{\pdv}[2]{\frac{\partial{#1}}{\partial{#2}}}


% \def\diffd{\mathrm{d}}
% \DeclareDocumentCommand\differential{ o g d() }{ % Differential 'd'
%     % o: optional n for nth differential
%     % g: optional argument for readability and to control spacing
%     % d: long-form as in d(cos x)
%     \IfNoValueTF{#2}{
%         \IfNoValueTF{#3}
%         {\diffd\IfNoValueTF{#1}{}{^{#1}}}
%         {\mathinner{\diffd\IfNoValueTF{#1}{}{^{#1}}\argopen(#3\argclose)}}
%     }
%     {\mathinner{\diffd\IfNoValueTF{#1}{}{^{#1}}#2} \IfNoValueTF{#3}{}{(#3)}}
% }
% \DeclareDocumentCommand\dd{}{\differential} % Shorthand for \differential

% \DeclareDocumentCommand\derivative{ s o m g d() }
% { % Total derivative
%     % s: star for \flatfrac flat derivative
%     % o: optional n for nth derivative
%     % m: mandatory (x in df/dx)
%     % g: optional (f in df/dx)
%     % d: long-form d/dx(...)
%     \IfBooleanTF{#1}
%     {\let\fractype\flatfrac}
%     {\let\fractype\frac}
%     \IfNoValueTF{#4}
%     {
%         \IfNoValueTF{#5}
%         {\fractype{\diffd \IfNoValueTF{#2}{}{^{#2}}}{\diffd #3\IfNoValueTF{#2}{}{^{#2}}}}
%         {\fractype{\diffd \IfNoValueTF{#2}{}{^{#2}}}{\diffd #3\IfNoValueTF{#2}{}{^{#2}}} \argopen(#5\argclose)}
%     }
%     {\fractype{\diffd \IfNoValueTF{#2}{}{^{#2}} #3}{\diffd #4\IfNoValueTF{#2}{}{^{#2}}}}
% }
% \DeclareDocumentCommand\dv{}{\derivative} % Shorthand for \derivative

% \DeclareDocumentCommand\partialderivative{ s o m g g d() }
% { % Partial derivative
%     % s: star for \flatfrac flat derivative
%     % o: optional n for nth derivative
%     % m: mandatory (x in df/dx)
%     % g: optional (f in df/dx)
%     % g: optional (y in d^2f/dxdy)
%     % d: long-form d/dx(...)
%     \IfBooleanTF{#1}
%     {\let\fractype\flatfrac}
%     {\let\fractype\frac}
%     \IfNoValueTF{#4}
%     {
%         \IfNoValueTF{#6}
%         {\fractype{\partial \IfNoValueTF{#2}{}{^{#2}}}{\partial #3\IfNoValueTF{#2}{}{^{#2}}}}
%         {\fractype{\partial \IfNoValueTF{#2}{}{^{#2}}}{\partial #3\IfNoValueTF{#2}{}{^{#2}}} \argopen(#6\argclose)}
%     }
%     {
%         \IfNoValueTF{#5}
%         {\fractype{\partial \IfNoValueTF{#2}{}{^{#2}} #3}{\partial #4\IfNoValueTF{#2}{}{^{#2}}}}
%         {\fractype{\partial^2 #3}{\partial #4 \partial #5}}
%     }
% }
% \DeclareDocumentCommand\pderivative{}{\partialderivative} % Shorthand for \partialderivative
% \DeclareDocumentCommand\pdv{}{\partialderivative} % Shorthand for \partialderivative

\DeclareDocumentCommand\variation{ o g d() }{ % Functional variation
    % o: optional n for nth differential
    % g: optional argument for readability and to control spacing
    % d: long-form as in d(F(g))
    \IfNoValueTF{#2}{
        \IfNoValueTF{#3}
        {\delta \IfNoValueTF{#1}{}{^{#1}}}
        {\mathinner{\delta \IfNoValueTF{#1}{}{^{#1}}\argopen(#3\argclose)}}
    }
    {\mathinner{\delta \IfNoValueTF{#1}{}{^{#1}}#2} \IfNoValueTF{#3}{}{(#3)}}
}
\DeclareDocumentCommand\var{}{\variation} % Shorthand for \variation

\DeclareDocumentCommand\functionalderivative{ s o m g d() }
{ % Functional derivative
    % s: star for \flatfrac flat derivative
    % o: optional n for nth derivative
    % m: mandatory (g in dF/dg)
    % g: optional (F in dF/dg)
    % d: long-form d/dx(...)
    \IfBooleanTF{#1}
    {\let\fractype\flatfrac}
    {\let\fractype\frac}
    \IfNoValueTF{#4}
    {
        \IfNoValueTF{#5}
        {\fractype{\variation \IfNoValueTF{#2}{}{^{#2}}}{\variation #3\IfNoValueTF{#2}{}{^{#2}}}}
        {\fractype{\variation \IfNoValueTF{#2}{}{^{#2}}}{\variation #3\IfNoValueTF{#2}{}{^{#2}}} \argopen(#5\argclose)}
    }
    {\fractype{\variation \IfNoValueTF{#2}{}{^{#2}} #3}{\variation #4\IfNoValueTF{#2}{}{^{#2}}}}
}
\DeclareDocumentCommand\fderivative{}{\functionalderivative} % Shorthand for \functionalderivative
\DeclareDocumentCommand\fdv{}{\functionalderivative} % Shorthand for \functionalderivative


\begin{document}

\section*{Numbers and Sets \hfill IA Mich}
\begin{itemize}
      \item Purpose of proofs, methods of proof (esp contradiction), non-valid proofs: other way round, assume something that does not exist (e.g. minimum element)
\end{itemize}

\section{Elementary Number Theory}
\begin{itemize}
      \item Natural numbers
            \begin{itemize}
                  \item Peano Axioms: define $\mathbb{N}$
                        \begin{enumerate}
                              \item Starting element (1 or 0) $\in \mathbb{N}$
                              \item Incrementation is closed in $\mathbb{N}$
                              \item Different elements increment to give different elements ($n \neq m \implies n+1 \neq m+1 $)
                              \item Axiom of induction: $P(1) \; \wedge \; \left[ P(n) \implies P(n+1) \right] \implies P(n) \; \forall n \in \mathbb{N}$
                        \end{enumerate}

                  \item Define addition and multiplication inductively
            \end{itemize}

      \item Weak and strong principles of induction
      \item Constructing $\mathbb{Z}$, and $\mathbb{Q}$ from $\mathbb{N}$
      \item Divisibility ($a \mid b \iff \exists \: c \in \mathbb{N} : ac = b$), prime ($a \mid p \implies a=p \textrm{ or } a=1$) and composites, prime factorisation, existence of infinitely
            many primes
      \item Highest common factor, division algorithm, Euclid's algorithm
      \item HCF as smallest positive linear combination of the two numbers
      \item Bezout's Theorem: Integer solutions to the equation $ax+by=c\iff\textrm{hcf}(a,b)\mid c$
      \item For all prime $p$, and $a,b\in\mathbb{Z}$, $p\mid ab\implies p\mid a\textrm{ or }p\mid b$
      \item Fundamental Theorem of Arithmetic: unique prime factorisation; does
            not hold in some other number systems, no unique factorisation into
            'primes'
      \item Modulo arithmetic
            \begin{itemize}
                  \item Two views: as integers on the number line, or as points on the clock
                        $\mathbb{Z}_{n}$
                  \item Inverses don't always exist, but if they do, they are unique, inverse
                        of $a$ in $\mathbb{Z}_{n}$ exists $\iff\textrm{hcf}(a,n)=1$, i.e.
                        coprime
            \end{itemize}
      \item Euler totient function $\phi$:
            \begin{itemize}
                  \item $\phi(n)=$ number of invertible integers (units) in $\mathbb{Z}_{n}$
                  \item $\phi(p)=p-1$; $\phi(pq)=pq-p-q+1=(p-1)(q-1)$; $\phi(p^{k})=p^{k}-p^{k-1}$
                        ($p,q$ prime)
            \end{itemize}
      \item Fermat Little Theorem: for $p$ prime and any integer $a\neq 0$, $a^{p-1}=1$ in $\mathbb{Z}_{p}$
      \item Fermat-Euler Theorem: for $a$ invertible in $\mathbb{Z}_{n}$, $a^{\phi(n)}=1$
      \item In $\mathbb{Z}_{p}$, $x^{2}=1\iff x=\pm1$
      \item Show existence of infinitely many primes in the form $4k+1$ and $4k+3$
      \item Wilson's Theorem: $(p-1)!=-1\textrm{ in }\mathbb{Z}_{p}$
      \item $x^{2}=-1\textrm{ in }\mathbb{Z}_{p}\textrm{ has a solution}\iff p=1\textrm{ (mod }4)$
      \item Linear congruences, uniqueness of solutions (each line if and only
            if)
      \item Simultaneous linear congruences: Chinese remainder theorem, existence
            and uniqueness
      \item RSA encoding
            \begin{itemize}
                  \item Have $n=pq$, product of two large distinct primes, we know $\phi(n)$
                        easily
                  \item Have message $x$, encode it by taking $x^{e}$ in $\mathbb{Z}_{n}$,
                        for some exponent $e$ (the coding exponent) coprime to $\phi(n)$
                  \item Decode: need $d$ with$\left(x^{e}\right)^{d}=x$ in $\mathbb{Z}_{n}$
                  \item Fermat-Euler: $x^{\phi(n)}=1$ (mod $n$), so need $ed=k\phi(n)+1$,
                        i.e. solving $ed\equiv1$ (mod $\phi(n)$), easy by Euclid
                  \item Hard to decode: to get $\phi(n)$, need to factorise $n$
            \end{itemize}
\end{itemize}


\section{The Reals}
\begin{itemize}
      \item Rationals are not complete: no square roots; have gaps, least upper
            bounds not guaranteed
      \item Upper bound and least upper bound definitions
      \item Real numbers: special property called 'least upper bound axiom': any
            non-empty set bounded above has a least upper bound, the supremum/lub
      \item Least upper bound of a set need not be in the set
      \item Axiom of Archimedes: $\mathbb{N}$ has no upper bound in $\mathbb{R},$i.e.
            $\forall r\in\mathbb{R},\exists n\in\mathbb{N}:n>r$
            \begin{itemize}
                  \item Corollary: $\forall t\in\mathbb{R},\exists n\in\mathbb{N}:\frac{1}{n}<t$
            \end{itemize}
      \item A set has a greatest element implies supremum of the set is in the
            set, vice versa
      \item Greatest lower bound comes 'for free'
      \item Show that the supremum of a set equals square root of something, etc.
      \item The rationals are dense in the reals (given any two real numbers,
            there must be some rational number in between); and some irrational
            between any two real number
      \item Infinite sum defined to be the limit of sequence of partial sums
      \item Limit of a sequence:${\displaystyle \lim_{n\to\infty}x_{n}=x\iff\forall\epsilon>0,\exists N\in\mathbb{\mathbb{N}},\forall n\geq N,|x_{n}-x|<\epsilon}$
      \item Convergent, divergent (limit does not exist, does not mean go to infinity)
      \item Limit of a sequence is unique, limits add and multiply
      \item Every bounded monotonic sequence converges: (say for increasing sequence)
            the set $S=\{x_{i}:i\in\mathbb{N}\}$ has a supremum, can get arbitrarily
            close to $\sup S$ (no smaller upper bound) and stay close to the
            $\sup S$ (since increasing)
      \item Bound terms of sequence by powers of two for easy evaluation
      \item Decimal expansion
      \item $e$ is irrational (proved), and transcendental (out-syl). Liouville
            number is transcendental
      \item Complex numbers as operations defined on $\mathbb{R}^{2}$
\end{itemize}


\section{Sets and Functions}
\begin{itemize}
      \item Constructing sets: subsets, unions, intersections, ordered pairs,
            power sets
      \item No universal set: Russel's paradox
      \item Finite size: can list elements
      \item Binomial coefficients, binomial theorem
      \item Inclusion exclusion formula
      \item Functions as rule of assigning elements of a set to another (formal
            definition as subset of Cartesian product)
      \item Injective, surjective, bijective
      \item Composition of functions, left/right inverse, invertible $\iff\exists$
            bijection
      \item Equivalence relation: (reflexive, symmetric, transitive) $\iff$partition a set
            the set into equivalence classes
      \item Quotient (the set of equivalence classes) and quotient/projection map
\end{itemize}

\section{Countability}
\begin{itemize}
      \item A set A is countable $\iff$ A is finite or bijects with $\mathbb{N}\iff\exists$
            injection $f:A\to\mathbb{N}$
      \item Countable union of countable sets is countable
      \item $\mathbb{N},\mathbb{Z},\mathbb{Q}$ are countable, $P(\mathbb{N}),\mathbb{R}$
            are uncountable
      \item No bijection exists from any set $X$ to $P(X)$
      \item To show a set $X$ is uncountable:
            \begin{itemize}
                  \item Copy diagonal argument
                  \item Inject uncountable set into $X$
            \end{itemize}
      \item To show a set X countable:
            \begin{itemize}
                  \item List its elements
                  \item Inject into a countable set (e.g. $\mathbb{N}$)
                  \item Use 'countable union of countable sets is countable'
                  \item If the set is related to $\mathbb{R}$, look at $\mathbb{Q}$
            \end{itemize}
      \item $A$ injects into $B\iff B$ surjects to $A$
      \item Schröder-Bernstein Theorem: $A$ injects into $B$ and $B$ injects
            into $A\iff A$ bijects with $B$
\end{itemize}

\end{document}
