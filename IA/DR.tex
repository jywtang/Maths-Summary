\documentclass[12pt, a4paper]{article}
\usepackage[T1]{fontenc}
\usepackage{amsmath, amssymb, amsthm}
\usepackage{babel}
\usepackage{siunitx}
\usepackage{tikz}
\usepackage{centernot}
\usepackage{tcolorbox}
\usepackage{cancel}
\usepackage{enumitem}
\usepackage{xparse}

\usetikzlibrary{arrows}

\setlength{\parindent}{0pt}


% Matrix groups
\newcommand{\GL}{\mathrm{GL}}
\newcommand{\M}{\mathrm{M}}
\newcommand{\Or}{\mathrm{O}}
\newcommand{\PGL}{\mathrm{PGL}}
\newcommand{\PSL}{\mathrm{PSL}}
\newcommand{\PSO}{\mathrm{PSO}}
\newcommand{\PSU}{\mathrm{PSU}}
\newcommand{\SL}{\mathrm{SL}}
\newcommand{\SO}{\mathrm{SO}}
\newcommand{\Spin}{\mathrm{Spin}}
\newcommand{\Sp}{\mathrm{Sp}}
\newcommand{\SU}{\mathrm{SU}}
\newcommand{\U}{\mathrm{U}}
\newcommand{\Mat}{\mathrm{Mat}}


% Special sets
\newcommand{\C}{\mathbb{C}}
\newcommand{\CP}{\mathbb{CP}}
\newcommand{\F}{\mathbb{F}}
\newcommand{\GG}{\mathbb{G}}
\newcommand{\N}{\mathbb{N}}
% \newcommand{\P}{\mathbb{P}}
\newcommand{\Q}{\mathbb{Q}}
\newcommand{\R}{\mathbb{R}}
\newcommand{\RP}{\mathbb{RP}}
\newcommand{\T}{\mathbb{T}}
\newcommand{\Z}{\mathbb{Z}}
\renewcommand{\H}{\mathbb{H}}

% Brackets
\renewcommand{\vec}[1]{\boldsymbol{\mathbf{#1}}}
\newcommand{\cis}[1]{ \cos\left( #1 \right) + i \sin \left( #1 \right)}

% Algebra
\DeclareMathOperator{\adj}{adj}
\DeclareMathOperator{\Ann}{Ann}
\DeclareMathOperator{\Aut}{Aut}
\DeclareMathOperator{\Char}{char}
\DeclareMathOperator{\disc}{disc}
\DeclareMathOperator{\dom}{dom}
\DeclareMathOperator{\fix}{fix}
\DeclareMathOperator{\Hom}{Hom}
\DeclareMathOperator{\id}{id}
\DeclareMathOperator{\image}{image}
\DeclareMathOperator{\im}{im}
\DeclareMathOperator{\tr}{tr}
\newcommand{\Bilin}{\mathrm{Bilin}}
\newcommand{\Frob}{\mathrm{Frob}}



\let\Im\relax
\let\Re\relax


\DeclareMathOperator{\hcf}{hcf}
\DeclareMathOperator{\Isom}{Isom}
\DeclareMathOperator{\lcm}{lcm}
\DeclareMathOperator{\sgn}{sgn}
\DeclareMathOperator{\supp}{supp}
\DeclareMathOperator{\Sym}{Sym}
\DeclareMathOperator{\Syl}{Syl}
\DeclareMathOperator{\Im}{Im}
\DeclareMathOperator{\Re}{Re}
\DeclareMathOperator{\Ker}{Ker}


% Theorems
\theoremstyle{definition}
\newtheorem*{aim}{Aim}
\newtheorem*{axiom}{Axiom}
\newtheorem*{claim}{Claim}
\newtheorem*{cor}{Corollary}
\newtheorem*{conjecture}{Conjecture}
\newtheorem*{defi}{Definition}
\newtheorem*{eg}{Example}
\newtheorem*{ex}{Exercise}
\newtheorem*{fact}{Fact}
\newtheorem*{law}{Law}
\newtheorem*{lemma}{Lemma}
\newtheorem*{notation}{Notation}
\newtheorem*{prop}{Proposition}
\newtheorem*{question}{Question}
\newtheorem*{rrule}{Rule}
\newtheorem*{thm}{Theorem}
\newtheorem*{assumption}{Assumption}

\newtheorem*{remark}{Remark}
\newtheorem*{warning}{Warning}
\newtheorem*{exercise}{Exercise}

\newtheorem{nthm}{Theorem}[section]
\newtheorem{nlemma}[nthm]{Lemma}
\newtheorem{nprop}[nthm]{Proposition}
\newtheorem{ncor}[nthm]{Corollary}


\newcommand{\abs}[1]{\left| #1 \right|} % for absolute value
\newcommand{\grad}[1]{\mathbf{\nabla} #1} % for gradient
\let\divsymb=\div % rename builtin command \div to \divsymb
\renewcommand{\div}[1]{\mathbf{\nabla} \cdot #1} % for divergence
\newcommand{\curl}[1]{\mathbf{\nabla} \times #1} % for curl




% Derivatives

\newcommand{\dd}[1][]{\mathrm{d} #1}
\newcommand{\odiff}[1]{\frac{\dd}{\dd{#1}}}
\newcommand{\odv}[2]{\frac{\dd{#1}}{\dd{#2}}}
\newcommand{\pdiff}[1]{\frac{\partial}{\partial{#1}}}
\newcommand{\pdv}[2]{\frac{\partial{#1}}{\partial{#2}}}


% \def\diffd{\mathrm{d}}
% \DeclareDocumentCommand\differential{ o g d() }{ % Differential 'd'
%     % o: optional n for nth differential
%     % g: optional argument for readability and to control spacing
%     % d: long-form as in d(cos x)
%     \IfNoValueTF{#2}{
%         \IfNoValueTF{#3}
%         {\diffd\IfNoValueTF{#1}{}{^{#1}}}
%         {\mathinner{\diffd\IfNoValueTF{#1}{}{^{#1}}\argopen(#3\argclose)}}
%     }
%     {\mathinner{\diffd\IfNoValueTF{#1}{}{^{#1}}#2} \IfNoValueTF{#3}{}{(#3)}}
% }
% \DeclareDocumentCommand\dd{}{\differential} % Shorthand for \differential

% \DeclareDocumentCommand\derivative{ s o m g d() }
% { % Total derivative
%     % s: star for \flatfrac flat derivative
%     % o: optional n for nth derivative
%     % m: mandatory (x in df/dx)
%     % g: optional (f in df/dx)
%     % d: long-form d/dx(...)
%     \IfBooleanTF{#1}
%     {\let\fractype\flatfrac}
%     {\let\fractype\frac}
%     \IfNoValueTF{#4}
%     {
%         \IfNoValueTF{#5}
%         {\fractype{\diffd \IfNoValueTF{#2}{}{^{#2}}}{\diffd #3\IfNoValueTF{#2}{}{^{#2}}}}
%         {\fractype{\diffd \IfNoValueTF{#2}{}{^{#2}}}{\diffd #3\IfNoValueTF{#2}{}{^{#2}}} \argopen(#5\argclose)}
%     }
%     {\fractype{\diffd \IfNoValueTF{#2}{}{^{#2}} #3}{\diffd #4\IfNoValueTF{#2}{}{^{#2}}}}
% }
% \DeclareDocumentCommand\dv{}{\derivative} % Shorthand for \derivative

% \DeclareDocumentCommand\partialderivative{ s o m g g d() }
% { % Partial derivative
%     % s: star for \flatfrac flat derivative
%     % o: optional n for nth derivative
%     % m: mandatory (x in df/dx)
%     % g: optional (f in df/dx)
%     % g: optional (y in d^2f/dxdy)
%     % d: long-form d/dx(...)
%     \IfBooleanTF{#1}
%     {\let\fractype\flatfrac}
%     {\let\fractype\frac}
%     \IfNoValueTF{#4}
%     {
%         \IfNoValueTF{#6}
%         {\fractype{\partial \IfNoValueTF{#2}{}{^{#2}}}{\partial #3\IfNoValueTF{#2}{}{^{#2}}}}
%         {\fractype{\partial \IfNoValueTF{#2}{}{^{#2}}}{\partial #3\IfNoValueTF{#2}{}{^{#2}}} \argopen(#6\argclose)}
%     }
%     {
%         \IfNoValueTF{#5}
%         {\fractype{\partial \IfNoValueTF{#2}{}{^{#2}} #3}{\partial #4\IfNoValueTF{#2}{}{^{#2}}}}
%         {\fractype{\partial^2 #3}{\partial #4 \partial #5}}
%     }
% }
% \DeclareDocumentCommand\pderivative{}{\partialderivative} % Shorthand for \partialderivative
% \DeclareDocumentCommand\pdv{}{\partialderivative} % Shorthand for \partialderivative

\DeclareDocumentCommand\variation{ o g d() }{ % Functional variation
    % o: optional n for nth differential
    % g: optional argument for readability and to control spacing
    % d: long-form as in d(F(g))
    \IfNoValueTF{#2}{
        \IfNoValueTF{#3}
        {\delta \IfNoValueTF{#1}{}{^{#1}}}
        {\mathinner{\delta \IfNoValueTF{#1}{}{^{#1}}\argopen(#3\argclose)}}
    }
    {\mathinner{\delta \IfNoValueTF{#1}{}{^{#1}}#2} \IfNoValueTF{#3}{}{(#3)}}
}
\DeclareDocumentCommand\var{}{\variation} % Shorthand for \variation

\DeclareDocumentCommand\functionalderivative{ s o m g d() }
{ % Functional derivative
    % s: star for \flatfrac flat derivative
    % o: optional n for nth derivative
    % m: mandatory (g in dF/dg)
    % g: optional (F in dF/dg)
    % d: long-form d/dx(...)
    \IfBooleanTF{#1}
    {\let\fractype\flatfrac}
    {\let\fractype\frac}
    \IfNoValueTF{#4}
    {
        \IfNoValueTF{#5}
        {\fractype{\variation \IfNoValueTF{#2}{}{^{#2}}}{\variation #3\IfNoValueTF{#2}{}{^{#2}}}}
        {\fractype{\variation \IfNoValueTF{#2}{}{^{#2}}}{\variation #3\IfNoValueTF{#2}{}{^{#2}}} \argopen(#5\argclose)}
    }
    {\fractype{\variation \IfNoValueTF{#2}{}{^{#2}} #3}{\variation #4\IfNoValueTF{#2}{}{^{#2}}}}
}
\DeclareDocumentCommand\fderivative{}{\functionalderivative} % Shorthand for \functionalderivative
\DeclareDocumentCommand\fdv{}{\functionalderivative} % Shorthand for \functionalderivative


\begin{document}
\section*{Dynamics and Relativity \hfill IA Lent}

\section{Newtonian Dynamics}
\begin{itemize}
      \item Particle: negligible size, specified by position, mass, charge etc
      \item Frame of reference: an origin and a set of axes to measure position etc
      \item Velocity, acceleration, momentum, etc
      \item Newton's Laws of Motion
            \begin{enumerate}
                  \item There exist inertial frames, where $\mathbf{F} = 0 \implies \mathbf{\ddot{r}} = 0$
                  \item $\mathbf{F} = \odv{t} (m\mathbf{v})$
                  \item Action-reaction
            \end{enumerate}
      \item Inertial frames are not unique, can be obtained by boost, translation (in space and time), rotation and reflection: Galilean transformations
      \item Laws of physics are Galilean invariant (in Newtonian)
      \item There is no absolute velocity
      \item Given $\mathbf{F}(t), \mathbf{r}(0), \mathbf{\dot{r}}(0)$, behaviour of particle is deterministic

\end{itemize}

\section{Dimensional Analysis}
\begin{itemize}
      \item L, M, T (and temperature perhaps), dimensions must be consistent in equation
      \item Solve relation by solving linear equations of powers of quantities
\end{itemize}

\section{Forces}
\begin{itemize}
      \item Potential: in 1D force depends on position only $\implies F = \odv{V}{x}$
      \item In 3D $\nabla \times \mathbf{F} = \mathbf{0} \implies $ F is conservative, i.e. $ \mathbf{F} = -\nabla V$
      \item Conservation of energy under conservative force
      \item Fixing energy gives first order DE
      \item Equilibrium $\iff V' = 0$, stability determined by $V''$ or gradient and hessian matrix
      \item Angular momentum $\mathbf{L} = m \mathbf{r} \times \mathbf{\dot{r}} $, conserved if and only if torque $\mathbf{\tau} = \mathbf{r} \times \mathbf{F} = 0$
      \item Central force depend only on distance $V = V(r)$, force points at radial direction, angular momentum is conserved
      \item Gravity: inverse square law; gravitational potential $\Phi_g(\mathbf{x}) = -\frac{GM}{r}$, gravitational field $\mathbf{g} = -\frac{GM}{r^2} \mathbf{\hat{r}}$
      \item Electromagnetic force: Lorentz Force Law, attraction/repulsion depends on sign of charges
      \item Friction: dry friction (static, kinetic), fluid drag (linear, quadratic), terminal velocity, projectile motion example
\end{itemize}

\section{Orbital Motion}
\begin{itemize}
      \item Use polar coordinates to describe motion in an orbit, central mass/charge at origin:

            \begin{align*}
                  \vec{r}        & = r \vec{e_r}                                                                                             \\
                  \dot{\vec{r}}  & = \dot{r} \vec{e_r} + r \dot{\theta} \vec{e_{\theta}}                                                     \\
                  \ddot{\vec{r}} & = (\ddot{r} - r \dot{\theta}^2 ) \vec{e_r} + (2 \dot{r} \dot{\theta} + r \ddot{\theta} ) \vec{e_{\theta}}
            \end{align*}

      \item Gravity/EM are central forces $\implies$ force points in radial direction, i.e. angular momentum conserved, motion is in a plane
      \item Let $h = \frac{L}{m}$ = angular momentum per unit mass (determined by initial conditions), acts as a constant for the DE
      \item Effective potential $V_{\textrm{eff}}$
            \begin{align*}
                  E & = \frac{1}{2} m \abs{\vec{\dot{r}}}^2  + V(r)                                  \\
                    & = \frac{1}{2} m \dot{r}^2 + \underbrace{\frac{1}{2} \frac{mh^2}{r^2} + V(r)}_{
                  V_{\textrm{eff}}}
            \end{align*}

      \item Treat the motion as a 1D problem in terms of $r$, the distance from origin and $V_{\textrm{eff}}$
      \item Stability of circular orbits related to nature of stationary point of $V_{\textrm{eff}}$
      \item The orbit equation: use the substitution $u = \frac{1}{r}$ and $ \odiff{t} = \frac{h}{r^2}\odiff{\theta}$, Newton's second law (for the radial motion) becomes
            \begin{align*}
                  m(\ddot{r} - r \dot{\theta}^2) & = F(r)                                                \\
                  \implies
                  \odv[2]{u}{\theta} + u         & = -\frac{1}{mh^2 u^2} \: F \left( \frac{1}{u} \right)
            \end{align*}
      \item Kepler problem: force in orbit equation given by inverse square law; this shows that orbits are circular, elliptical, parabolic or hyperbolic, characterised by eccentricity
      \item Kepler's Laws for Planetary Motion:
            \begin{enumerate}
                  \item Elliptical orbits with the sun at one focus
                  \item Equal areas: $\frac{1}{2}r^2 \dot{\theta}$ is constant (conservation of angular momentum)
                  \item $ P^2 \propto a^3$, where $P =$ period of orbital motion, $a =$ length of semi-major axis
            \end{enumerate}
      \item Rutherford Scattering: similar framework but repulsive: can find angle of deflection forces
\end{itemize}

\section{Rotating frames of Reference}
\begin{itemize}
      \item Equation of motion between rotating frame and inertial frame
            For frame of reference rotating about a fixed axis:
      \item Coriolis force: (only exists for moving object, velocity perceived in the rotating frame)
            \[-2m \vec{\omega} \times \dot{\vec{x}} \]
      \item Centrifugal force: (of order $O(\omega^2)$)
            \[-m \vec{\omega}\times (\vec{\omega} \times \vec{x}) \]
      \item Euler force: (only exists for changing angular velocity) \[-m \dot{\vec{\omega}} \times \vec{x}\]
\end{itemize}

\section{System of Particles}
\begin{itemize}

      \item A system of $N$ particles, each one ($1 \leq i \leq N$) associated with mass ($m_i$) and position ($\vec{x}_i(t)$).
      \item 2nd law on each particle: \[\dot{\vec{p}}_i = \vec{F}^{\text{ext}}_i + \sum_{j\neq i} \vec{F}_{ij} \] ($\vec{F}_{ij}$ being force acting on i due to j)
      \item 3rd law on each pair of particles: \[\vec{F}_{ij} = -\vec{F}_{ji}\]
      \item Define centre of mass \[ \vec{R} = \frac{\sum m_i \vec{x}_i}{\sum m_i} = \frac{1}{M} \sum m_i \vec{x}_i \]
            (mass-weighted average position)
      \item Total momentum: \[\vec{P} = \sum m_i \dot{\vec{x}}_i = M \dot{\vec{R}} \]
      \item 2nd law applies to a system since internal forces cancel:
            \[ \dot{\vec{P}} = \sum_i \vec{F}^{\text{ext}}_i + \cancelto{0}{\sum_i \sum_{j\neq i} \vec{F}_{ij}} \]
      \item Conservation of total momentum \[ \sum_i \vec{F}^{\text{ext}}_i = 0 \implies \dot{\vec{P}} = 0 \]
      \item Total angular momentum (about the origin):
            \[\vec{L} = \sum_i \vec{x}_i \times \vec{p}_i\]
      \item Condition for conservation of angular momentum:
            \begin{align*}
                  \dot{\vec{L}} & = \sum_i \left( \vec{x}_i \times \dot{\vec{p}}_i + \cancelto{0}{\dot{\vec{x}}_i \times \vec{p}_i} \right)                                        \\
                                & = \sum_i \left( \vec{x}_i \times \left(\vec{F}^{\text{ext}}_i + \sum_{j\neq i} \vec{F}_{ij}\right) \right)                                       \\
                                & = \underbrace{\sum_i \vec{x}_i \times \vec{F}^{\text{ext}}_i}_{\text{torque } \vec{\tau}} + \sum_{i<j} (\vec{x}_i-\vec{x}_j) \times \vec{F}_{ij}
            \end{align*}
            Angular momentum conserved if $ \vec{\tau} =0$ and $\vec{F}_{ij} \parallel \vec{x}_i-\vec{x}_j$
      \item Separate motion of particle:
            \[ \vec{x}_i = \underbrace{\vec{R}}_{\text{motion of CoM}} + \underbrace{\vec{y}_i}_{\text{relative to CoM}}\]
            \[\sum_i m_i \vec{y}_i = \vec{0}\]
      \item Total kinetic energy:
            \begin{align*}
                  T & = \frac{1}{2} \sum_i m_i \dot{\vec{x}}_i \cdot \dot{\vec{x}}_i                                                                                                                        \\
                    & = \frac{1}{2} \sum_i m_i \dot{\vec{R}}^2 + \frac{1}{2} \sum_i m_i \dot{\vec{y}}_i^2 + \cancelto{0}{\sum_i m_i \dot{\vec{R}} \cdot  \dot{\vec{y}}_i}                                   \\
                    & = \underbrace{\frac{1}{2} M \dot{\vec{R}}^2}_{\text{KE as if mass is concentrated at CoM}} + \underbrace{\frac{1}{2} \sum_i m_i \dot{\vec{y}}_i^2}_{\text{KE of particles about CoM}}
            \end{align*}
      \item Change in KE along path:
            \begin{align*}
                  \Delta T & = \sum_i \int_{C_i} \left( \vec{F}^{\text{ext}}_i + \sum_{j\neq i} \vec{F}_{ij} \right)\cdot \dd \vec{x}_i                         \\
                           & = \sum_i \int_{C_i} \vec{F}^{\text{ext}}_i \cdot \dd \vec{x}_i + \sum_i \sum_{j\neq i} \int_{C_i} \vec{F}_{ij} \cdot \dd \vec{x}_i
            \end{align*}
            Energy conserved if all external and internal forces are conservative:
            \begin{align*}
                  \vec{F}^{\text{ext}}_i & = - \nabla_i V^{\text{ext}}_i              \\
                  \vec{F}_{ij}           & = - \nabla_i V_{ij}(|\vec{x}_i-\vec{x}_j|)
            \end{align*}
            ($\nabla_i = $ directional derivative along path $C_i$) \\
            Then conservation of energy: \[E = T+ \sum_i V^{\text{ext}}_i + \frac{1}{2} \sum_{i,j}  V_{ij}(|\vec{x}_i-\vec{x}_j|) \]
      \item Two-body problem
      \item Variable mass, rocket equation


\end{itemize}

\section{Rigid bodies}
\begin{itemize}
      \item Rigid body: Distance between any two particles remain the same (no stretching, only translation of the whole object or rotation at the same angular velocity)
      \item For an object rotating about an axis through origin with angular speed $\omega$ (angular velocity $\vec{\omega} = \omega \hat{\vec{n}}$, direction given by right hand rule):
            \[\dot{\vec{x}} = \vec{\omega} \times \vec{x}\]

      \item Kinetic energy of the whole object (wrt origin, lying on axis of rotation):
            \begin{align*}
                  T & = \sum_i \frac{1}{2} m_i (\vec{\omega} \times \vec{x_i})^2                                                                        \\
                    & = \frac{1}{2} \underbrace{\left( \sum_i m_i (\hat{\vec{n}} \times \vec{x}_{i})^2\right)}_{I = \sum_i m_i x_{i\perp}^2  } \omega^2
            \end{align*}

      \item Angular momentum in terms of moment of inertia:

            \begin{align*}
                  \vec{L}                                & = \sum_i \vec{x}_i \times (m_i \dot{\vec{x}}_i)                       \\
                                                         & = \sum_i m_i \vec{x}_i \times (\vec{\omega} \times \vec{x_i})         \\
                                                         & = \omega \sum_i m_i \vec{x}_i \times (\hat{\vec{n}} \times \vec{x_i}) \\
                  \therefore \vec{L} \cdot \hat{\vec{n}} & = \omega \sum_i m_i \times (\hat{\vec{n}} \times \vec{x_i})^2         \\
                                                         & = I \omega
            \end{align*}
      \item Translating everything to integrals (continuum):
            \begin{center}
                  \renewcommand{\arraystretch}{1.5}
                  \begin{tabular}{|c|c|c|}
                        \hline
                                          & Discrete                                     & Continuum (3D version)                                       \\
                        \hline
                        Total Mass        & $M = \sum_i m_i$                             & $M = \int \rho(\vec{x}) \dd V$                               \\
                        \hline
                        Centre of Mass    & $\vec{R} = \frac{1}{M} \sum_i m_i \vec{x}_i$ & $\vec{R} = \frac{1}{M} \int  \rho(\vec{x}) \, \vec{x} \dd V$ \\
                        \hline
                        Moment of Inertia & $I = \sum_i m_i x_{i\perp}^2 $               & $I = \int \rho (\vec{x}) \, x_{\perp}^2\dd V$                \\
                        \hline
                  \end{tabular}
            \end{center}
      \item Perpendicular Axis Theorem: \\ For \emph{a lamina}, with $z$-axis pointing in normal direction, $x,y,z$-axes mutually orthogonal: \[I_z = I_x + I_y\]
      \item Parallel Axis Theorem: \\
            Given axis through centre of mass, and a parallel axis distance $d$ apart,
            \[I = I_{CoM} + Md^2\]

      \item Inertia tensor of object about origin: \[\mathcal{I}_{ij} = \int \rho(\vec{x}) (x_k x_k \delta_{ij} - x_i x_j) \dd{V} \]
            \[\text{KE} = \vec{\omega}^T \mathcal{I}\vec{\omega}, \]
            \[\text{Moment of inertia about any axis through origin: } I = \hat{\vec{n}}^T \mathcal{I}\hat{\vec{n}} \]

      \item Motion about CoM: break down to motion of CoM and rotation about CoM:
            if body rotates with angular velocity $\vec{\omega}$ about CoM:
            \[\dot{\vec{x}}_i = \dot{\vec{R}} + \vec{\omega} \times (\vec{x}_i-\vec{R}) \]
            \[\text{KE} = \frac{1}{2} M \dot{\vec{R}} \cdot \dot{\vec{R}} + \frac{1}{2}I \omega^2\]

      \item Angular velocity about any particle/point in the moving body is the same (e.g. tip of a stick): take point $\vec{Q}$ moving with body:
            \begin{align*}
                  \vec{Q} = \vec{R} + (\vec{Q}-\vec{R}) \\
                  \dot{\vec{Q}} = \dot{\vec{R}} + \vec{\omega} \times (\vec{Q}-\vec{R})
            \end{align*}
            Then any particle $\vec{r}_i$:
            \begin{align*}
                  \dot{\vec{r}}_i = \dot{\vec{R}} + \vec{\omega} \times (\vec{r}_i-\vec{R}) \\
                  \therefore \dot{\vec{r}}_i = \dot{\vec{Q}} + \vec{\omega} \times (\vec{r}_i-\vec{Q})
            \end{align*}
            In particular if $Q$ is fixed (pivot point), then total KE = rotational KE about pivot (no translational).
\end{itemize}



\section{Special Relativity}
\begin{itemize}
      \item Postulates, Lorentz transformations
      \item Simultaneity, causality, time dilation, length contraction
      \item Velocity addition
      \item Invariant interval, proper time
      \item Monkowski space, 4-vectors
\end{itemize}

\end{document}
