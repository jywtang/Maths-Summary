\documentclass[12pt, a4paper]{article}
\usepackage[T1]{fontenc}
\usepackage{amsmath, amssymb, amsthm}
\usepackage{babel}
\usepackage{siunitx}
\usepackage{tikz}
\usepackage{centernot}
\usepackage{tcolorbox}
\usepackage{cancel}
\usepackage{enumitem}
\usepackage{xparse}

\usetikzlibrary{arrows}

\setlength{\parindent}{0pt}


% Matrix groups
\newcommand{\GL}{\mathrm{GL}}
\newcommand{\M}{\mathrm{M}}
\newcommand{\Or}{\mathrm{O}}
\newcommand{\PGL}{\mathrm{PGL}}
\newcommand{\PSL}{\mathrm{PSL}}
\newcommand{\PSO}{\mathrm{PSO}}
\newcommand{\PSU}{\mathrm{PSU}}
\newcommand{\SL}{\mathrm{SL}}
\newcommand{\SO}{\mathrm{SO}}
\newcommand{\Spin}{\mathrm{Spin}}
\newcommand{\Sp}{\mathrm{Sp}}
\newcommand{\SU}{\mathrm{SU}}
\newcommand{\U}{\mathrm{U}}
\newcommand{\Mat}{\mathrm{Mat}}


% Special sets
\newcommand{\C}{\mathbb{C}}
\newcommand{\CP}{\mathbb{CP}}
\newcommand{\F}{\mathbb{F}}
\newcommand{\GG}{\mathbb{G}}
\newcommand{\N}{\mathbb{N}}
% \newcommand{\P}{\mathbb{P}}
\newcommand{\Q}{\mathbb{Q}}
\newcommand{\R}{\mathbb{R}}
\newcommand{\RP}{\mathbb{RP}}
\newcommand{\T}{\mathbb{T}}
\newcommand{\Z}{\mathbb{Z}}
\renewcommand{\H}{\mathbb{H}}

% Brackets
\renewcommand{\vec}[1]{\boldsymbol{\mathbf{#1}}}
\newcommand{\cis}[1]{ \cos\left( #1 \right) + i \sin \left( #1 \right)}

% Algebra
\DeclareMathOperator{\adj}{adj}
\DeclareMathOperator{\Ann}{Ann}
\DeclareMathOperator{\Aut}{Aut}
\DeclareMathOperator{\Char}{char}
\DeclareMathOperator{\disc}{disc}
\DeclareMathOperator{\dom}{dom}
\DeclareMathOperator{\fix}{fix}
\DeclareMathOperator{\Hom}{Hom}
\DeclareMathOperator{\id}{id}
\DeclareMathOperator{\image}{image}
\DeclareMathOperator{\im}{im}
\DeclareMathOperator{\tr}{tr}
\newcommand{\Bilin}{\mathrm{Bilin}}
\newcommand{\Frob}{\mathrm{Frob}}



\let\Im\relax
\let\Re\relax


\DeclareMathOperator{\hcf}{hcf}
\DeclareMathOperator{\Isom}{Isom}
\DeclareMathOperator{\lcm}{lcm}
\DeclareMathOperator{\sgn}{sgn}
\DeclareMathOperator{\supp}{supp}
\DeclareMathOperator{\Sym}{Sym}
\DeclareMathOperator{\Syl}{Syl}
\DeclareMathOperator{\Im}{Im}
\DeclareMathOperator{\Re}{Re}
\DeclareMathOperator{\Ker}{Ker}


% Theorems
\theoremstyle{definition}
\newtheorem*{aim}{Aim}
\newtheorem*{axiom}{Axiom}
\newtheorem*{claim}{Claim}
\newtheorem*{cor}{Corollary}
\newtheorem*{conjecture}{Conjecture}
\newtheorem*{defi}{Definition}
\newtheorem*{eg}{Example}
\newtheorem*{ex}{Exercise}
\newtheorem*{fact}{Fact}
\newtheorem*{law}{Law}
\newtheorem*{lemma}{Lemma}
\newtheorem*{notation}{Notation}
\newtheorem*{prop}{Proposition}
\newtheorem*{question}{Question}
\newtheorem*{rrule}{Rule}
\newtheorem*{thm}{Theorem}
\newtheorem*{assumption}{Assumption}

\newtheorem*{remark}{Remark}
\newtheorem*{warning}{Warning}
\newtheorem*{exercise}{Exercise}

\newtheorem{nthm}{Theorem}[section]
\newtheorem{nlemma}[nthm]{Lemma}
\newtheorem{nprop}[nthm]{Proposition}
\newtheorem{ncor}[nthm]{Corollary}


\newcommand{\abs}[1]{\left| #1 \right|} % for absolute value
\newcommand{\grad}[1]{\mathbf{\nabla} #1} % for gradient
\let\divsymb=\div % rename builtin command \div to \divsymb
\renewcommand{\div}[1]{\mathbf{\nabla} \cdot #1} % for divergence
\newcommand{\curl}[1]{\mathbf{\nabla} \times #1} % for curl




% Derivatives

\newcommand{\dd}[1][]{\mathrm{d} #1}
\newcommand{\odiff}[1]{\frac{\dd}{\dd{#1}}}
\newcommand{\odv}[2]{\frac{\dd{#1}}{\dd{#2}}}
\newcommand{\pdiff}[1]{\frac{\partial}{\partial{#1}}}
\newcommand{\pdv}[2]{\frac{\partial{#1}}{\partial{#2}}}


% \def\diffd{\mathrm{d}}
% \DeclareDocumentCommand\differential{ o g d() }{ % Differential 'd'
%     % o: optional n for nth differential
%     % g: optional argument for readability and to control spacing
%     % d: long-form as in d(cos x)
%     \IfNoValueTF{#2}{
%         \IfNoValueTF{#3}
%         {\diffd\IfNoValueTF{#1}{}{^{#1}}}
%         {\mathinner{\diffd\IfNoValueTF{#1}{}{^{#1}}\argopen(#3\argclose)}}
%     }
%     {\mathinner{\diffd\IfNoValueTF{#1}{}{^{#1}}#2} \IfNoValueTF{#3}{}{(#3)}}
% }
% \DeclareDocumentCommand\dd{}{\differential} % Shorthand for \differential

% \DeclareDocumentCommand\derivative{ s o m g d() }
% { % Total derivative
%     % s: star for \flatfrac flat derivative
%     % o: optional n for nth derivative
%     % m: mandatory (x in df/dx)
%     % g: optional (f in df/dx)
%     % d: long-form d/dx(...)
%     \IfBooleanTF{#1}
%     {\let\fractype\flatfrac}
%     {\let\fractype\frac}
%     \IfNoValueTF{#4}
%     {
%         \IfNoValueTF{#5}
%         {\fractype{\diffd \IfNoValueTF{#2}{}{^{#2}}}{\diffd #3\IfNoValueTF{#2}{}{^{#2}}}}
%         {\fractype{\diffd \IfNoValueTF{#2}{}{^{#2}}}{\diffd #3\IfNoValueTF{#2}{}{^{#2}}} \argopen(#5\argclose)}
%     }
%     {\fractype{\diffd \IfNoValueTF{#2}{}{^{#2}} #3}{\diffd #4\IfNoValueTF{#2}{}{^{#2}}}}
% }
% \DeclareDocumentCommand\dv{}{\derivative} % Shorthand for \derivative

% \DeclareDocumentCommand\partialderivative{ s o m g g d() }
% { % Partial derivative
%     % s: star for \flatfrac flat derivative
%     % o: optional n for nth derivative
%     % m: mandatory (x in df/dx)
%     % g: optional (f in df/dx)
%     % g: optional (y in d^2f/dxdy)
%     % d: long-form d/dx(...)
%     \IfBooleanTF{#1}
%     {\let\fractype\flatfrac}
%     {\let\fractype\frac}
%     \IfNoValueTF{#4}
%     {
%         \IfNoValueTF{#6}
%         {\fractype{\partial \IfNoValueTF{#2}{}{^{#2}}}{\partial #3\IfNoValueTF{#2}{}{^{#2}}}}
%         {\fractype{\partial \IfNoValueTF{#2}{}{^{#2}}}{\partial #3\IfNoValueTF{#2}{}{^{#2}}} \argopen(#6\argclose)}
%     }
%     {
%         \IfNoValueTF{#5}
%         {\fractype{\partial \IfNoValueTF{#2}{}{^{#2}} #3}{\partial #4\IfNoValueTF{#2}{}{^{#2}}}}
%         {\fractype{\partial^2 #3}{\partial #4 \partial #5}}
%     }
% }
% \DeclareDocumentCommand\pderivative{}{\partialderivative} % Shorthand for \partialderivative
% \DeclareDocumentCommand\pdv{}{\partialderivative} % Shorthand for \partialderivative

\DeclareDocumentCommand\variation{ o g d() }{ % Functional variation
    % o: optional n for nth differential
    % g: optional argument for readability and to control spacing
    % d: long-form as in d(F(g))
    \IfNoValueTF{#2}{
        \IfNoValueTF{#3}
        {\delta \IfNoValueTF{#1}{}{^{#1}}}
        {\mathinner{\delta \IfNoValueTF{#1}{}{^{#1}}\argopen(#3\argclose)}}
    }
    {\mathinner{\delta \IfNoValueTF{#1}{}{^{#1}}#2} \IfNoValueTF{#3}{}{(#3)}}
}
\DeclareDocumentCommand\var{}{\variation} % Shorthand for \variation

\DeclareDocumentCommand\functionalderivative{ s o m g d() }
{ % Functional derivative
    % s: star for \flatfrac flat derivative
    % o: optional n for nth derivative
    % m: mandatory (g in dF/dg)
    % g: optional (F in dF/dg)
    % d: long-form d/dx(...)
    \IfBooleanTF{#1}
    {\let\fractype\flatfrac}
    {\let\fractype\frac}
    \IfNoValueTF{#4}
    {
        \IfNoValueTF{#5}
        {\fractype{\variation \IfNoValueTF{#2}{}{^{#2}}}{\variation #3\IfNoValueTF{#2}{}{^{#2}}}}
        {\fractype{\variation \IfNoValueTF{#2}{}{^{#2}}}{\variation #3\IfNoValueTF{#2}{}{^{#2}}} \argopen(#5\argclose)}
    }
    {\fractype{\variation \IfNoValueTF{#2}{}{^{#2}} #3}{\variation #4\IfNoValueTF{#2}{}{^{#2}}}}
}
\DeclareDocumentCommand\fderivative{}{\functionalderivative} % Shorthand for \functionalderivative
\DeclareDocumentCommand\fdv{}{\functionalderivative} % Shorthand for \functionalderivative


\begin{document}

\section*{Differential Equations \hfill IA Mich}
\section{Basic Calculus}
\begin{itemize}
      \item Limits: $\epsilon$-$\delta$ definition,
            algebra of limits, left/right handed limits
      \item Derivative: definition, chain/product/Leibniz's rule
      \item Order of magnitude: Big O, little o notation
      \item Taylor Series/polynomial
      \item L'Hopital's rule
      \item Integration, FTC, indefinite integral, u-sub, trig-sub, by parts
      \item Multivariate functions: Partial derivative, multivariate chain rule
\end{itemize}

\section{First Order Linear ODE}
\begin{itemize}
      \item Eigenfunction, logarithm
      \item Homogeneous, heterogeneous, forcing term
      \item General solution = $y_c + y_p$ (complementary function + particular integral)
      \item \underline{Integrating factor}: $y'' + p(x)y' + q(x)y=f(x)$, use IF = $e^{\int{p(x)dx}}$
\end{itemize}

\section{First Order Non-Linear ODE}
\begin{itemize}
      \item Separable equations
      \item \underline{Exact equations}: can write $Q(x,y)\odv{f}{x} + P(x,y) =0$ as $\odv{f}{x}(x,y)=0$
      \item Quick check for exact equations over a simply connected domain: $\frac{\partial P}{\partial y} = \frac{\partial Q}{\partial x}$
      \item Isocline: lines on which derivative is the same
      \item Stability of fixed points: perturbation analysis
      \item Phase portrait: represent solution by showing directions/growth directions
\end{itemize}

\section{Higher Order Linear ODE}
\begin{itemize}
      \item Linear ODE: general solution = $y_c + y_p$ by superposition
      \item Linearly independent solutions: $n$ many for $n$-th order ODE
      \item 2nd Order ODE with constant coefficients: use characteristic equation to determine eigenfunction; detuning
      \item \underline{Reduction of order} given complementary function $y_1(x)$, find $y_2(x)$ by letting $y_2(x) = u(x)y_1(x)$
      \item Phase space: each point describes the state of a system; linearly independent solutions
      \item Wronskian; Abel's Theorem; using the Wronskian gives a first order DE for $y_2$
      \item \underline{Equidimensional equations}: scaling invariant; in the form $ax^2y'' + bxy' + cy = 0$; eigenfunction $y=x^k$; or introduce substitution $ z = \ln x$
      \item \underline{Variation of parameters}: given solution vectors for homogeneous equation, let $y_p=uy_1+vy_2$
      \item Damped oscillating systems: $\ddot{y} + 2\kappa \dot{y}(t) + y(t) = f(t)$ : types of damping; free/forced motion gives transient/long time response
      \item Resonance: $ \ddot{y} +\omega_0^2 y = \sin(\omega_0 t)$: detuning by having forcing term approach $\omega_0$: gives $y_p = \frac{-t}{2\omega_0}\cos\omega_0 t$
      \item Dirac Delta Function; Heaviside Step Function; Ramp function; DE with these as forcing: continuity/jump conditions

      \item \underline{Series solutions (Method of Frobenius)}: given \[p(x)y''+q(x)y'+r(x)y = 0,\] the point $x = x_0$ is

            \begin{itemize}
                  \item Ordinary point: if $\frac{q(x)}{p(x)}$ and $\frac{r(x)}{p(x)}$ have Taylor series about $x = x_0$, i.e. analytic
            \end{itemize}
            If $\frac{q}{p}$ and $\frac{r}{p}$ do not have Taylor series, then rewrite the DE as \[P(x)(x-x_0)^2y''+Q(x)(x-x_0)y'+R(x)y=0,\]
            \begin{itemize}
                  \item Regular singular point: if $\frac{Q}{P}$ and $\frac{R}{P}$ have Taylor series, i.e. $\frac{q}{p}(x-x_0)$ and $\frac{r}{p}(x-x_0)^2$ have Taylor series
                  \item Irregular singular point: otherwise
            \end{itemize}

            The solutions:
            \begin{itemize}
                  \item Ordinary point: Use the series $y=\sum_{n=0}^{\infty} a_n (x-x_0)^n$; solve recurrence relation
                  \item Regular Singular point: use $y=\sum_{n=0}^{\infty} a_n (x-x_0)^{n+\sigma}$, for $\sigma \in \mathbb{C} $, solve indicial equation, giving roots $\sigma_1, \sigma_2$ such that Re($\sigma_1$) $\geq$ Re($\sigma_2$):
                  \item If $\sigma_1 - \sigma_2$ is not an integer $\implies $ two independent solutions
                  \item If $\sigma_1 - \sigma_2$ is an integer $\implies$ there are two solutions in the form
                        \begin{align*}
                              y_1 & = \sum_{n=0}^{\infty}a_n (x-x_0)^{n+\sigma_1}             \\
                              y_2 & = y_1 \ln x + \sum_{n=0}^{\infty}b_n (x-x_0)^{n+\sigma_2}\end{align*}
            \end{itemize}
\end{itemize}

\section{Multivariate Functions}
\begin{itemize}
      \item Directional derivative, gradient; maximum rate of change, direction of steepest ascent
      \item Stationary points: max/min, saddle point
      \item Multivariate Taylor series, Hessian matrix; eigenvalues of Hessian matrix determines the type of stationary points
            \begin{itemize}
                  \item all $+$ve: min
                  \item all $-$ve: max
                  \item both $+$ and $-$: saddle
                  \item has eigenvalue zero: degenerate, need higher order terms
            \end{itemize}
      \item An $n$-th order ODE can be expressed as a system of $n$ first order ODEs
      \item Matrix methods: with $\mathbf{\dot{Y}}=M \mathbf{Y}+\mathbf{F}$, find eigenvalues and eigenvectors; then consider behaviour of $\mathbf{Y}$ when it is an eigenvector
      \item Behaviour around stationary points depend on eigenvalues: stable/unstable node, saddle node, stable/unstable spiral, center
      \item Autonomous system (e.g. predator prey): behaviour do not depend on time
      \item Linearise a system of DEs to analyse stability
\end{itemize}

PDEs
\begin{itemize}
      \item First order wave equation: $y(x,t)$ satisfying $\pdv{y}{t} - c\pdv{y}{x}=0$ : method of characteristics, lines on which $\odv{y}{t}$ is constant
      \item Second order wave equation: $y(x,t)$ satisfying $\pdv[2]{y}{t} - c^2\pdv[2]{y}{x}=0$ : solution is $y = f(x+ct) + g(x-ct) $
      \item Diffusion Equation: $\pdv{y}{t} =\kappa \pdv[2]{y}{x}$ : introduce similarity variable (by dimensional analysis), reduce PDE to ODE
\end{itemize}

\section{Discrete Equation}
\begin{itemize}
      \item \underline{Numerical Integration}: express derivative as fraction, find recurrence relation and then take limit
      \item \underline{Series solutions}: let power series, gather terms to obtain recurrence relation
      \item \underline{Stability of fixed points}: compare magnitude of error term in successive iterations
\end{itemize}
\end{document}