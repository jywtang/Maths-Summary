\documentclass[12pt, a4paper]{article}
\usepackage[T1]{fontenc}
\usepackage{amsmath, amssymb, amsthm}
\usepackage{babel}
\usepackage{siunitx}
\usepackage{tikz}
\usepackage{centernot}
\usepackage{tcolorbox}
\usepackage{cancel}
\usepackage{enumitem}
\usepackage{xparse}

\usetikzlibrary{arrows}

\setlength{\parindent}{0pt}


% Matrix groups
\newcommand{\GL}{\mathrm{GL}}
\newcommand{\M}{\mathrm{M}}
\newcommand{\Or}{\mathrm{O}}
\newcommand{\PGL}{\mathrm{PGL}}
\newcommand{\PSL}{\mathrm{PSL}}
\newcommand{\PSO}{\mathrm{PSO}}
\newcommand{\PSU}{\mathrm{PSU}}
\newcommand{\SL}{\mathrm{SL}}
\newcommand{\SO}{\mathrm{SO}}
\newcommand{\Spin}{\mathrm{Spin}}
\newcommand{\Sp}{\mathrm{Sp}}
\newcommand{\SU}{\mathrm{SU}}
\newcommand{\U}{\mathrm{U}}
\newcommand{\Mat}{\mathrm{Mat}}


% Special sets
\newcommand{\C}{\mathbb{C}}
\newcommand{\CP}{\mathbb{CP}}
\newcommand{\F}{\mathbb{F}}
\newcommand{\GG}{\mathbb{G}}
\newcommand{\N}{\mathbb{N}}
% \newcommand{\P}{\mathbb{P}}
\newcommand{\Q}{\mathbb{Q}}
\newcommand{\R}{\mathbb{R}}
\newcommand{\RP}{\mathbb{RP}}
\newcommand{\T}{\mathbb{T}}
\newcommand{\Z}{\mathbb{Z}}
\renewcommand{\H}{\mathbb{H}}

% Brackets
\renewcommand{\vec}[1]{\boldsymbol{\mathbf{#1}}}
\newcommand{\cis}[1]{ \cos\left( #1 \right) + i \sin \left( #1 \right)}

% Algebra
\DeclareMathOperator{\adj}{adj}
\DeclareMathOperator{\Ann}{Ann}
\DeclareMathOperator{\Aut}{Aut}
\DeclareMathOperator{\Char}{char}
\DeclareMathOperator{\disc}{disc}
\DeclareMathOperator{\dom}{dom}
\DeclareMathOperator{\fix}{fix}
\DeclareMathOperator{\Hom}{Hom}
\DeclareMathOperator{\id}{id}
\DeclareMathOperator{\image}{image}
\DeclareMathOperator{\im}{im}
\DeclareMathOperator{\tr}{tr}
\newcommand{\Bilin}{\mathrm{Bilin}}
\newcommand{\Frob}{\mathrm{Frob}}



\let\Im\relax
\let\Re\relax


\DeclareMathOperator{\hcf}{hcf}
\DeclareMathOperator{\Isom}{Isom}
\DeclareMathOperator{\lcm}{lcm}
\DeclareMathOperator{\sgn}{sgn}
\DeclareMathOperator{\supp}{supp}
\DeclareMathOperator{\Sym}{Sym}
\DeclareMathOperator{\Syl}{Syl}
\DeclareMathOperator{\Im}{Im}
\DeclareMathOperator{\Re}{Re}
\DeclareMathOperator{\Ker}{Ker}


% Theorems
\theoremstyle{definition}
\newtheorem*{aim}{Aim}
\newtheorem*{axiom}{Axiom}
\newtheorem*{claim}{Claim}
\newtheorem*{cor}{Corollary}
\newtheorem*{conjecture}{Conjecture}
\newtheorem*{defi}{Definition}
\newtheorem*{eg}{Example}
\newtheorem*{ex}{Exercise}
\newtheorem*{fact}{Fact}
\newtheorem*{law}{Law}
\newtheorem*{lemma}{Lemma}
\newtheorem*{notation}{Notation}
\newtheorem*{prop}{Proposition}
\newtheorem*{question}{Question}
\newtheorem*{rrule}{Rule}
\newtheorem*{thm}{Theorem}
\newtheorem*{assumption}{Assumption}

\newtheorem*{remark}{Remark}
\newtheorem*{warning}{Warning}
\newtheorem*{exercise}{Exercise}

\newtheorem{nthm}{Theorem}[section]
\newtheorem{nlemma}[nthm]{Lemma}
\newtheorem{nprop}[nthm]{Proposition}
\newtheorem{ncor}[nthm]{Corollary}


\newcommand{\abs}[1]{\left| #1 \right|} % for absolute value
\newcommand{\grad}[1]{\mathbf{\nabla} #1} % for gradient
\let\divsymb=\div % rename builtin command \div to \divsymb
\renewcommand{\div}[1]{\mathbf{\nabla} \cdot #1} % for divergence
\newcommand{\curl}[1]{\mathbf{\nabla} \times #1} % for curl




% Derivatives

\newcommand{\dd}[1][]{\mathrm{d} #1}
\newcommand{\odiff}[1]{\frac{\dd}{\dd{#1}}}
\newcommand{\odv}[2]{\frac{\dd{#1}}{\dd{#2}}}
\newcommand{\pdiff}[1]{\frac{\partial}{\partial{#1}}}
\newcommand{\pdv}[2]{\frac{\partial{#1}}{\partial{#2}}}


% \def\diffd{\mathrm{d}}
% \DeclareDocumentCommand\differential{ o g d() }{ % Differential 'd'
%     % o: optional n for nth differential
%     % g: optional argument for readability and to control spacing
%     % d: long-form as in d(cos x)
%     \IfNoValueTF{#2}{
%         \IfNoValueTF{#3}
%         {\diffd\IfNoValueTF{#1}{}{^{#1}}}
%         {\mathinner{\diffd\IfNoValueTF{#1}{}{^{#1}}\argopen(#3\argclose)}}
%     }
%     {\mathinner{\diffd\IfNoValueTF{#1}{}{^{#1}}#2} \IfNoValueTF{#3}{}{(#3)}}
% }
% \DeclareDocumentCommand\dd{}{\differential} % Shorthand for \differential

% \DeclareDocumentCommand\derivative{ s o m g d() }
% { % Total derivative
%     % s: star for \flatfrac flat derivative
%     % o: optional n for nth derivative
%     % m: mandatory (x in df/dx)
%     % g: optional (f in df/dx)
%     % d: long-form d/dx(...)
%     \IfBooleanTF{#1}
%     {\let\fractype\flatfrac}
%     {\let\fractype\frac}
%     \IfNoValueTF{#4}
%     {
%         \IfNoValueTF{#5}
%         {\fractype{\diffd \IfNoValueTF{#2}{}{^{#2}}}{\diffd #3\IfNoValueTF{#2}{}{^{#2}}}}
%         {\fractype{\diffd \IfNoValueTF{#2}{}{^{#2}}}{\diffd #3\IfNoValueTF{#2}{}{^{#2}}} \argopen(#5\argclose)}
%     }
%     {\fractype{\diffd \IfNoValueTF{#2}{}{^{#2}} #3}{\diffd #4\IfNoValueTF{#2}{}{^{#2}}}}
% }
% \DeclareDocumentCommand\dv{}{\derivative} % Shorthand for \derivative

% \DeclareDocumentCommand\partialderivative{ s o m g g d() }
% { % Partial derivative
%     % s: star for \flatfrac flat derivative
%     % o: optional n for nth derivative
%     % m: mandatory (x in df/dx)
%     % g: optional (f in df/dx)
%     % g: optional (y in d^2f/dxdy)
%     % d: long-form d/dx(...)
%     \IfBooleanTF{#1}
%     {\let\fractype\flatfrac}
%     {\let\fractype\frac}
%     \IfNoValueTF{#4}
%     {
%         \IfNoValueTF{#6}
%         {\fractype{\partial \IfNoValueTF{#2}{}{^{#2}}}{\partial #3\IfNoValueTF{#2}{}{^{#2}}}}
%         {\fractype{\partial \IfNoValueTF{#2}{}{^{#2}}}{\partial #3\IfNoValueTF{#2}{}{^{#2}}} \argopen(#6\argclose)}
%     }
%     {
%         \IfNoValueTF{#5}
%         {\fractype{\partial \IfNoValueTF{#2}{}{^{#2}} #3}{\partial #4\IfNoValueTF{#2}{}{^{#2}}}}
%         {\fractype{\partial^2 #3}{\partial #4 \partial #5}}
%     }
% }
% \DeclareDocumentCommand\pderivative{}{\partialderivative} % Shorthand for \partialderivative
% \DeclareDocumentCommand\pdv{}{\partialderivative} % Shorthand for \partialderivative

\DeclareDocumentCommand\variation{ o g d() }{ % Functional variation
    % o: optional n for nth differential
    % g: optional argument for readability and to control spacing
    % d: long-form as in d(F(g))
    \IfNoValueTF{#2}{
        \IfNoValueTF{#3}
        {\delta \IfNoValueTF{#1}{}{^{#1}}}
        {\mathinner{\delta \IfNoValueTF{#1}{}{^{#1}}\argopen(#3\argclose)}}
    }
    {\mathinner{\delta \IfNoValueTF{#1}{}{^{#1}}#2} \IfNoValueTF{#3}{}{(#3)}}
}
\DeclareDocumentCommand\var{}{\variation} % Shorthand for \variation

\DeclareDocumentCommand\functionalderivative{ s o m g d() }
{ % Functional derivative
    % s: star for \flatfrac flat derivative
    % o: optional n for nth derivative
    % m: mandatory (g in dF/dg)
    % g: optional (F in dF/dg)
    % d: long-form d/dx(...)
    \IfBooleanTF{#1}
    {\let\fractype\flatfrac}
    {\let\fractype\frac}
    \IfNoValueTF{#4}
    {
        \IfNoValueTF{#5}
        {\fractype{\variation \IfNoValueTF{#2}{}{^{#2}}}{\variation #3\IfNoValueTF{#2}{}{^{#2}}}}
        {\fractype{\variation \IfNoValueTF{#2}{}{^{#2}}}{\variation #3\IfNoValueTF{#2}{}{^{#2}}} \argopen(#5\argclose)}
    }
    {\fractype{\variation \IfNoValueTF{#2}{}{^{#2}} #3}{\variation #4\IfNoValueTF{#2}{}{^{#2}}}}
}
\DeclareDocumentCommand\fderivative{}{\functionalderivative} % Shorthand for \functionalderivative
\DeclareDocumentCommand\fdv{}{\functionalderivative} % Shorthand for \functionalderivative

\begin{document}
\section*{Probability \hfill IA Lent}
\section{Basic Probability}
\begin{itemize}
      \item Probability space: Axioms
            \begin{itemize}
                  \item Sample space ($\Omega$) consists of outcomes $\omega$
                  \item $\sigma$-algebra ($\mathcal{F}$) consists of events: \\
                        $\Omega \in \mathcal{F}$ \\
                        Closed under complement and countable union
                  \item Probability measure ($\mathbb{P}: \mathcal{F} \to [0,1]$): $\mathbb{P}(\Omega) = 1$ \\
                        Countable additivity: disjoint sequence $(A_n)$,
                        \[P(\bigcup_{n=1}^{\infty} A_n) = \sum_{n=1}^{\infty} P(A_n)\]
            \end{itemize}

      \item Equally likely outcomes in finite sample space: probability given by counting: \[P(A) = \frac{|A|}{|\Omega|} \]

      \item Combinatorial analysis:
            \begin{itemize}
                  \item Addition and multiplication rule
                  \item Permutation and combination
                  \item Multinomial coefficient
            \end{itemize}

      \item Stirling's formula: \[ n! \sim \sqrt{2\pi}n^{n+\frac{1}{2}}e^{-n} \]

      \item Countable subadditivity: \[P(\bigcup_{n=1}^{\infty} A_n) \leq \sum_{n=1}^{\infty} P(A_n)\]

      \item Continuity of probability measures: for non-decreasing sequence of events $A_1 \subseteq A_2 \subseteq A_3 \dots $:
            \[ P \left( \bigcup_{n=1}^{\infty}A_n \right) = \lim_{n \to \infty} P(A_n)\]

      \item Inclusion-exclusion principle: useful for problems with symmetry, i.e. probability of an intersection of any $n$ events is the same:

            \begin{align*}
                  P \left(\bigcup_{j=1}^{n} A_j \right)
                   & = \sum_{j=1}^{n} (-1)^{j+1} \sum_{1 \leq i_1 \leq \dots i_j \leq n} P(A_{i_1} \cap \dots \cap A_{i_j}) \\
                   & = \sum_{j=1}^{n} (-1)^{j+1} \binom{n}{j} P(A_1 \cap \dots \cap A_j)
            \end{align*}

      \item Bonferroni Inequalities

\end{itemize}


\section{Independence and Conditional Probability}
\begin{itemize}
      \item Independence:
            \[P(A|B) = P(A) \text{ or } \]
            The events $A_1, A_2, \dots A_n$ are independent iff for any $k\geq 2$ and $ 1 \leq i_1 < i_2 < i_3 < \dots i_k \leq n $
            \[ P(A_{i_1} \cap A_{i_2} \cap A_{i_3} \cap \dots \cap A_{i_k}) =  P(A_{i_1}) P(A_{i_2}) P(A_{i_3}) \dots P(A_{i_k}) \]

      \item Conditional Probability:
            \[P(A|B) = \frac{P(A\cap B)}{P(B)}\]
      \item Law of Total Probability: Given a partition $(B_n)_{n\geq 1}$ of $\Omega$, \[P(A) = \sum_{n=1}^{\infty} P(A|B_n) P(B_n)\]
      \item Bayes' Theorem:
            \[P(A|B) = \frac{P(B|A)P(A)}{P(B)}\]
\end{itemize}

\section{Discrete Probability Distributions}

\begin{itemize}
      \item Bernoulli: $ \text{Bern}(p) = \text{Bin}(1,p)$ \[p_1 = p, p_0 = 1-p\]
      \item Binomial: $\text{Bin}(n,p)$ \[p_k = \binom{n}{k} p^k (1-p)^{n-k}, \text{ where } k \geq 0
            \]
      \item Poisson: $\text{Poisson}(\lambda) , \lambda \in (0,\infty)$ \[p_k = \frac{e^{-\lambda}\lambda^k}{k!}, \text{ where }  k \geq 0 \]
      \item Geometric: $\text{Geom}(p)$ \[p_k = (1-p)^{k-1}p, \text{ where } k\geq 1\]
      \item Binomial approaches Poisson: \[\text{Bin}(N,p) \to \text{Poisson}(Np)\]

\end{itemize}

\section{Random Variable}
\begin{itemize}
      \item Random Variable: function that assigns (say real) values to each outcome
            \[X : \Omega \to \Omega_X (=\R \text{ or } \R^n)\]

      \item Discrete random variable if $\Omega$ is countable
            (Random variable in general probability space need $ \{\omega: X(\omega) <x \} \in \mathcal{F}$ for all $x$)

      \item Indicator function: \[1_A(\omega) = \begin{cases}
                        1 \text{, if } \omega \in A \\
                        0 \text{, if } \omega \notin A
                  \end{cases} \]

      \item (Cumulative/probability) distribution function (for real-valued rv):
            \[F_X(x) = P(X \leq x)\]
            \begin{itemize}
                  \item Piecewise constant
                  \item Non-decreasing
                  \item Right continuous
                  \item $\displaystyle{\lim_{x \to -\infty} F_X(x) = 0 } $
                  \item $\displaystyle{\lim_{x \to +\infty} F_X(x) = 1 } $
            \end{itemize}
            (This is equivalent to specifying the weights $(p_x)$ of the rv, i.e. specifying the probability mass function)

      \item Independence: \[ P(X=x,Y=y) = P(X=x)P(Y=y) \]
            Any $n$ random variables $X_1 , X_2, \dots X_n$ are independent iff
            \[ P(X_1 = x_1, X_2 = x_2,  \dots X_n = x_n) =  P(X_1 = x_1) P(X_2 = x_2) \dots P(X_n = x_n) \] (This implies any subset of the rvs are independent)

      \item Functions of random variables: new random variable $g(X)$ defined by
            \[g(X)(\omega) = g(X(\omega)) \]
\end{itemize}

\section{Expectation}
\begin{itemize}
      \item Expectation: Probability weighted sum of values
            \begin{align*}
                  \text{E}(X) & = \sum_{\omega \in \Omega} X(\omega) P(\omega) \\
                              & = \sum_{x \in \Omega_X} x P(X=x)
            \end{align*}
            Defined iff the expectations the positive X and negative X are not both infinite ($\text{E}(X_+)$, $\text{E}(X_-)$ )

      \item Integrable: rv $X$ has $\text{E}(|X|) \leq \infty$
      \item Properties of expectation: positive definite, linear, multiplicative if independent, $\text{E}(1(A)) = P(A)$

      \item Moments: $n$-th moment $=\text{E}(X^n)$
      \item Variance:
            \begin{align*}
                  \text{Var}(X) & = \text{E}((X-\text{E}(X))^2)  \\
                                & =\text{E}(X^2)-(\text{E}(X))^2
            \end{align*}

      \item Covariance:
            \[\text{Cov}(X,Y) = \text{E}[(X-\text{E}(X))(Y-\text{E}(Y))] \]

      \item Correlation:
            \[\text{Corr}(X,Y) = \frac{\text{Cov}(X,Y)}{\sqrt{\text{Var}(X)\text{Var}(Y)}}\]
\end{itemize}


\section{Inequalities}

\begin{itemize}
      \item Markov's Inequality: Given non-negative rv $X$, and $\lambda > 0$
            \[P(X \geq \lambda) \leq \frac{\text{E}(X)}{\lambda}\]

      \item Chebyshev's Inequality: Given rv $X$ with finite $\text{E}(X)$, and $\lambda > 0$
            \[ P(|X-\text{E}(X)| \geq \lambda ) \leq \frac{\text{Var}(X)}{\lambda^2}\]

      \item Cauchy-Schwarz: \[\text{E}(|XY|) \leq \sqrt{\text{E}(X^2)\text{E}(Y^2)}\]

      \item Jensen's Inequality: Given convex function $f$ on $I$, $X$ integrable on $I$
            \[f(\text{E}(X)) \leq \text{E}(f(X))\]
            AM/GM then follows
\end{itemize}

\section{Random walk (L9)}
\begin{itemize}
      \item Definition \[X_n = x + Y_1 + Y_2 + \dots + Y_n, \text{ where } Y_i \text{ are iid rvs}\]
      \item Gambler's ruin, Expected time of absorption ...
\end{itemize}


\section{Conditional Expectation}
\begin{itemize}
      \item Definition \[E(X|B) = \frac {E(X \cdot 1(B))}{P(B)} \]
      \item Law of Total Expectation: Given $(\Omega_i)$ a partition of $\Omega$, \[E(X) = \sum E(X|\Omega_i)P(\Omega_i)\]
      \item (Joint distribution, Marginal distribution, Conditional distribution)
      \item Convolution: given $X,Y$ independent, $P(X+Y=z) = \sum_y P(X=z-y)P(Y=y) $
      \item Conditional expectation of $X$ given $Y$ (a random variable in terms of $Y$): if $E(X | Y=y) = g(y)$, then $E(X|Y) = g(Y)$
      \item \[E(X) = E_Y(E(X|Y))\]
\end{itemize}

\section{Probability generating function (L9)}
\begin{itemize}
      \item Definition: \[G_X(t) = \text{E}(t^X) = \sum p_k t^k \]
      \item Can recover probability distribution by differentiation and put $0$
      \item pgf of $S = \sum_{i=1}^n X_i$, where $X_i$ independent, is
            \[F_S(t) = \prod_i F_{X_i}(t)\]
      \item Random sum ($N$) of iid random variables ($X_i$): $\sum_{i=1}^N X_i$ is $F_N(F_{X_1}(t))$
\end{itemize}

\section{Branching processes (L11)}
\begin{itemize}
      \item Galton-Watson Process
      \item Condition on first step, set number of branches
      \item Extinction probability
\end{itemize}

\section{Continuous Random Variables}
\begin{itemize}
      \item $X$ is a continuous rv if distribution function $F_X$ is also left continuous
      \item Probability distribution function continuous $\implies X$ is continuous
      \item Probability distribution function differentiable $\implies X$ is absolutely continuous (then have probability density function)
      \item Probability density function (pdf) $f(x)$ satisfies
            \begin{align*}
                   & f(x) \geq 0 \text{ for all } x \in \R, & \int_{\R} f(x) \dd x = 1
            \end{align*}
      \item Examples of distributions (L13)
            \begin{itemize}
                  \item Uniform distribution: $X \sim U[a,b]$ \[f_X(x) = \frac{1}{b-a} 1_{[a,b]}(x)\]
                  \item Exponential, memoryless property: $X \sim Exp(\lambda)$ \[f_X(x) = \lambda e^{-\lambda x} 1_{\{x>0\}}(x)\]
                  \item Gamma distribution (generalised/sum of exponential) $X \sim \Gamma(\lambda,n)$, where $\lambda >0, n\in \N$ : \[f_X(x) = e^{-\lambda x} \lambda^n \frac{x^{n-1}}{(n-1)!}, \text{ for } x\geqslant 0\]
                  \item Standard Cauchy distribution $X \sim \text{Cauchy}(0,1)$ \[f_X(x) = \frac{1}{\pi(x^2+1)}\]
                  \item Gaussian (standard normal) Normal: $X \sim N(\mu, \sigma^2)$ \[f_X(x) = \frac{1}{\sqrt{2\pi \sigma^2}} e^\frac{-(x-\mu)^2}{2\sigma^2}\] Distribution function of standard normal $N(0,1)$: $\Phi(x)$
            \end{itemize}
      \item Expectation ($E(X)= \int_\R x f_X(x) \dd x$) and variance ($Var(X)=E[(X-E(X))^2]$) of continuous rv
\end{itemize}

\section{Multivariate rv}
\begin{itemize}
      \item Joint distribution of rvs $X_1,\dots X_n$ \[F(x_1,\dots,x_n) = P(X_1 \leqslant x_1,\dots X_n \leqslant x_n)\]
      \item Joint density = mixed derivative
      \item Random variables are independent if and only if the density function factorises
      \item Marginal density = density of one rv (integrate everything else)
      \item Convolution (for independent rvs): \[f_{X+Y}(z) = f_X \ast f_Y(z) = \int_{-\infty}^{\infty} f_X(z-y) f_Y(y) \dd y\]
      \item Conditional density: \[f_{X|Y}(x|y) = \frac{f_{X,Y}(x,y)}{f_{Y}(y)}\]
      \item Conditional expectation, law of total probability
\end{itemize}

\section{Transformation of rv}
\begin{itemize}
      \item One-dimensional rv: \[ f_Y(y) = f_X(x) \abs{\odv{x}{y}} \]
      \item Multi-dimensional rv: \[ f_Y(y) = f_X(x) \abs{J} \] where $J$ is the Jacobian, \[J = \det (\pdv{x_i}{y_i})_{i,j=1}^d \]
\end{itemize}

Order Statistics of a random sample (L15)

\section{Simulation of random variables (L17)}
\begin{itemize}
      \item Want to generate rv $X$ with distribution function $F(x)$: \\
            Using $U \sim U[0,1]$, then $F^{-1}(U)$ has same distribution as $X$

      \item Rejection Sampling: \\
            Given $A \subseteq [0,1]^d$ ($d$-dim unit cube), want to generate rv $X$ in $\R^d$ with density function $f_X(x) = \frac{1_A(x)}{\abs{A}}$ ('uniform' on $A$) \\

            Use $U[0,1]$, generate a sequence of d-dimensional rv (points in unit cube) $(U_n)_{n\in \N}$, where each point is
            \[U_n = (U_{1,n} , U_{2,n} ,\dots U_{d,n}) \]
            Take $X = U_N$, where $N = \min{}$
      \item Bounded density ...
\end{itemize}

\section{Moment Generating Function (L17)}
\begin{itemize}
      \item mgf of continuous rv $X$ \[M_X(t) = E(e^{tX}) = \int_{-\infty}^{\infty} e^{tx} f_X(x) \dd x\]
      \item Given $X_1,\dots X_n$ independent, \[M_{X_1+X_2+\dots+X_n}(t) = \prod_{i=1}^n M_{X_i}(t)\]
      \item Uniquely determine distribution if mgf is finite in some neighbourhood of the origin
      \item Continuity of mgf: \\
            Suppose $X$ is a rv, and $(X_n)$ a sequence of cts rv, if $M_{X_n}(\theta) \to M_{X}(\theta) \forall \theta \in \R$ and $M_X(\theta)$ is finite for some $\theta \neq 0$, then $X_n$ converges to $X$ in distribution
      \item Multivariate mgf: \\
            Suppose $X = (X_1,\dots,X_n)$ is a rv in $\R^n$, then mgf of X is \[E(e^{\theta^T X})\]

\end{itemize}

\section{Gaussian: Multivariate}
\begin{itemize}
      \item $X$ in $\R$ is Gaussian if \[X = \mu+\sigma Z, \text{ where } Z \sim N(0,1)\]
      \item $X = \begin{pmatrix} X_1 \\ X_2 \\ \vdots \\ X_n \end{pmatrix}$ a rv in $\R^n$ is Gaussian \\
            if $ u^T X = \sum_{i=1}^n u_i X_i  $ is Gaussian for any $u  = \begin{pmatrix} u_1 \\ u_2 \\ \vdots \\ u_n \end{pmatrix} \in \R^n$
      \item Mean and variance \[\mu = E(X) = \begin{pmatrix} E(X_1) \\ E(X_2) \\ \vdots \\ E(X_n) \end{pmatrix} \]
            \begin{align*}
                  V & = \text{Var(X)}
                  \\ &= E((X-\mu)(X-\mu)^T)
                  \\ &=  E \begin{pmatrix}
                        (X_1-\mu_1)^2 & (X_1-\mu_1)(X_2-\mu_2) & \cdots & (X_1-\mu_1)(X_n-\mu_n) \\
                        \vdots        & (X_2-\mu_2)^2          & \cdots & \vdots                 \\
                        \vdots        & \vdots                 & \ddots & \vdots                 \\
                        \vdots        & \cdots                 & \cdots & (X_n-\mu_n)^2
                  \end{pmatrix}
                  \\ &= (\text{Cov}(X_i,X_j))
            \end{align*}
      \item mgf of $X$ is \[M_X(\lambda) = E(e^{\lambda^T X}) = e^{\lambda^T \mu + \frac{\lambda^T V \lambda}{2}}\] (uniquely determined by mean and variance), written $X \sim N(\mu, V)$
      \item Can use $Z = (Z_1, \dots Z_n)$, where $(Z_i)$ iid $\sim N(0,1)$ in $\R^n$ to construct any $n$-dim Gaussian vector: $X = \mu + \sigma Z \sim N(\mu, V)$  ($\sigma$ is the square root of matrix $V$)
      \item Density of X: \\
            $X = \mu + \sigma Z$, if $V$ is positive definite (invertible), have density function
            \begin{align*}
                  f_X(x) & = f_Z(z) |J|                                                                          \\
                         & = \prod_{i=1}^n \left( \frac{1}{\sqrt{2\pi}} e^{-z_i^2 /2} \right) |\det \sigma^{-1}| \\
                         & = \frac{1}{(2\pi)^{n/2}} e^{-z^T z/2} \frac{1}{(\det V)^{1/2}}                        \\
                         & = \frac{1}{(2\pi)^{n/2} (\det V)^{1/2} } e^{\frac{-(x-\mu)^T V (x-\mu)}{2}}
            \end{align*}
            In general: non-negative definite (can have zero eigenvalues): diagonalise $V$ to get \[V = \left( \begin{array}{c|c}
                              U        & 0 \\
                              \hline 0 & 0
                        \end{array} \right),\] where $U$ is an $m\times m$ positive definite matrix.
            Write $\mu = \begin{pmatrix} \lambda \\ \nu \end{pmatrix}$, where $\lambda \in \R^m$.
            Then $X = \begin{pmatrix} Y \\ \nu \end{pmatrix}$, and density of Y is given by
            \[ f_Y(y) = \frac{1}{(2\pi)^{m/2} (\det U)^{1/2} } e^{\frac{-(y-\lambda)^T U (y-\lambda)}{2}} \]

      \item Bivariate normal
      \item (Multivariate Central Limit Theorem)
\end{itemize}


\section{Limit Theorems}
\begin{itemize}
      \item Weak law of large numbers: \\
            Given $S_n = X_1 + \dots + X_n$, where $(X_i)$ iid with $E(X_1) = \mu <\infty $, then for all $\varepsilon >0,$ as $n \to \infty$ \[ P \left( \abs{\frac{S_n}{n}-\mu} > \varepsilon \right) \to 0\]
            (converge in probability)
      \item Strong law of large numbers: \\
            Given $S_n = X_1 + \dots + X_n$, where $(X_i)$ integrable iid with $E(X_1) = \mu <\infty $, then \[P \left( \lim_{n\to \infty} \frac{S_n}{n} = \mu \right) = 1 \]
            (almost surely/converge with prob 1)
      \item Central Limit Theorem: \\
            Given $S_n = X_1 + \dots + X_n$, where $(X_i)$ iid with $E(X_1) = \mu < \infty$ and $\text{Var}(X_1) = \sigma^2 < \infty$, then \[P \left( \frac{S_n - n\mu}{\sigma \sqrt{n}} \leq x \right) \to \Phi(x) \]
            \[\text{ i.e. } S_n \sim N(n\mu, n\sigma^2) \]
\end{itemize}

\section{Geometric Probs}
\begin{itemize}
      \item Box Muller Transform
      \item Buffon's needle
      \item Bertrand's paradox: defn of random
\end{itemize}

\end{document}