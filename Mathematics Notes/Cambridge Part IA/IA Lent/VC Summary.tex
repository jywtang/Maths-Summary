\documentclass[12pt, a4paper]{article}
\usepackage[T1]{fontenc}
\usepackage{amsmath, amssymb, amsthm}
\usepackage{babel}
\usepackage{siunitx}
\usepackage{tikz}
\usepackage{centernot}
\usepackage{tcolorbox}
\usepackage{cancel}
\usepackage{enumitem}
\usepackage{xparse}

\usetikzlibrary{arrows}

\setlength{\parindent}{0pt}


% Matrix groups
\newcommand{\GL}{\mathrm{GL}}
\newcommand{\M}{\mathrm{M}}
\newcommand{\Or}{\mathrm{O}}
\newcommand{\PGL}{\mathrm{PGL}}
\newcommand{\PSL}{\mathrm{PSL}}
\newcommand{\PSO}{\mathrm{PSO}}
\newcommand{\PSU}{\mathrm{PSU}}
\newcommand{\SL}{\mathrm{SL}}
\newcommand{\SO}{\mathrm{SO}}
\newcommand{\Spin}{\mathrm{Spin}}
\newcommand{\Sp}{\mathrm{Sp}}
\newcommand{\SU}{\mathrm{SU}}
\newcommand{\U}{\mathrm{U}}
\newcommand{\Mat}{\mathrm{Mat}}


% Special sets
\newcommand{\C}{\mathbb{C}}
\newcommand{\CP}{\mathbb{CP}}
\newcommand{\F}{\mathbb{F}}
\newcommand{\GG}{\mathbb{G}}
\newcommand{\N}{\mathbb{N}}
% \newcommand{\P}{\mathbb{P}}
\newcommand{\Q}{\mathbb{Q}}
\newcommand{\R}{\mathbb{R}}
\newcommand{\RP}{\mathbb{RP}}
\newcommand{\T}{\mathbb{T}}
\newcommand{\Z}{\mathbb{Z}}
\renewcommand{\H}{\mathbb{H}}

% Brackets
\renewcommand{\vec}[1]{\boldsymbol{\mathbf{#1}}}
\newcommand{\cis}[1]{ \cos\left( #1 \right) + i \sin \left( #1 \right)}

% Algebra
\DeclareMathOperator{\adj}{adj}
\DeclareMathOperator{\Ann}{Ann}
\DeclareMathOperator{\Aut}{Aut}
\DeclareMathOperator{\Char}{char}
\DeclareMathOperator{\disc}{disc}
\DeclareMathOperator{\dom}{dom}
\DeclareMathOperator{\fix}{fix}
\DeclareMathOperator{\Hom}{Hom}
\DeclareMathOperator{\id}{id}
\DeclareMathOperator{\image}{image}
\DeclareMathOperator{\im}{im}
\DeclareMathOperator{\tr}{tr}
\newcommand{\Bilin}{\mathrm{Bilin}}
\newcommand{\Frob}{\mathrm{Frob}}



\let\Im\relax
\let\Re\relax


\DeclareMathOperator{\hcf}{hcf}
\DeclareMathOperator{\Isom}{Isom}
\DeclareMathOperator{\lcm}{lcm}
\DeclareMathOperator{\sgn}{sgn}
\DeclareMathOperator{\supp}{supp}
\DeclareMathOperator{\Sym}{Sym}
\DeclareMathOperator{\Syl}{Syl}
\DeclareMathOperator{\Im}{Im}
\DeclareMathOperator{\Re}{Re}
\DeclareMathOperator{\Ker}{Ker}


% Theorems
\theoremstyle{definition}
\newtheorem*{aim}{Aim}
\newtheorem*{axiom}{Axiom}
\newtheorem*{claim}{Claim}
\newtheorem*{cor}{Corollary}
\newtheorem*{conjecture}{Conjecture}
\newtheorem*{defi}{Definition}
\newtheorem*{eg}{Example}
\newtheorem*{ex}{Exercise}
\newtheorem*{fact}{Fact}
\newtheorem*{law}{Law}
\newtheorem*{lemma}{Lemma}
\newtheorem*{notation}{Notation}
\newtheorem*{prop}{Proposition}
\newtheorem*{question}{Question}
\newtheorem*{rrule}{Rule}
\newtheorem*{thm}{Theorem}
\newtheorem*{assumption}{Assumption}

\newtheorem*{remark}{Remark}
\newtheorem*{warning}{Warning}
\newtheorem*{exercise}{Exercise}

\newtheorem{nthm}{Theorem}[section]
\newtheorem{nlemma}[nthm]{Lemma}
\newtheorem{nprop}[nthm]{Proposition}
\newtheorem{ncor}[nthm]{Corollary}


\newcommand{\abs}[1]{\left| #1 \right|} % for absolute value
\newcommand{\grad}[1]{\mathbf{\nabla} #1} % for gradient
\let\divsymb=\div % rename builtin command \div to \divsymb
\renewcommand{\div}[1]{\mathbf{\nabla} \cdot #1} % for divergence
\newcommand{\curl}[1]{\mathbf{\nabla} \times #1} % for curl




% Derivatives

\newcommand{\dd}[1][]{\mathrm{d} #1}
\newcommand{\odiff}[1]{\frac{\dd}{\dd{#1}}}
\newcommand{\odv}[2]{\frac{\dd{#1}}{\dd{#2}}}
\newcommand{\pdiff}[1]{\frac{\partial}{\partial{#1}}}
\newcommand{\pdv}[2]{\frac{\partial{#1}}{\partial{#2}}}


% \def\diffd{\mathrm{d}}
% \DeclareDocumentCommand\differential{ o g d() }{ % Differential 'd'
%     % o: optional n for nth differential
%     % g: optional argument for readability and to control spacing
%     % d: long-form as in d(cos x)
%     \IfNoValueTF{#2}{
%         \IfNoValueTF{#3}
%         {\diffd\IfNoValueTF{#1}{}{^{#1}}}
%         {\mathinner{\diffd\IfNoValueTF{#1}{}{^{#1}}\argopen(#3\argclose)}}
%     }
%     {\mathinner{\diffd\IfNoValueTF{#1}{}{^{#1}}#2} \IfNoValueTF{#3}{}{(#3)}}
% }
% \DeclareDocumentCommand\dd{}{\differential} % Shorthand for \differential

% \DeclareDocumentCommand\derivative{ s o m g d() }
% { % Total derivative
%     % s: star for \flatfrac flat derivative
%     % o: optional n for nth derivative
%     % m: mandatory (x in df/dx)
%     % g: optional (f in df/dx)
%     % d: long-form d/dx(...)
%     \IfBooleanTF{#1}
%     {\let\fractype\flatfrac}
%     {\let\fractype\frac}
%     \IfNoValueTF{#4}
%     {
%         \IfNoValueTF{#5}
%         {\fractype{\diffd \IfNoValueTF{#2}{}{^{#2}}}{\diffd #3\IfNoValueTF{#2}{}{^{#2}}}}
%         {\fractype{\diffd \IfNoValueTF{#2}{}{^{#2}}}{\diffd #3\IfNoValueTF{#2}{}{^{#2}}} \argopen(#5\argclose)}
%     }
%     {\fractype{\diffd \IfNoValueTF{#2}{}{^{#2}} #3}{\diffd #4\IfNoValueTF{#2}{}{^{#2}}}}
% }
% \DeclareDocumentCommand\dv{}{\derivative} % Shorthand for \derivative

% \DeclareDocumentCommand\partialderivative{ s o m g g d() }
% { % Partial derivative
%     % s: star for \flatfrac flat derivative
%     % o: optional n for nth derivative
%     % m: mandatory (x in df/dx)
%     % g: optional (f in df/dx)
%     % g: optional (y in d^2f/dxdy)
%     % d: long-form d/dx(...)
%     \IfBooleanTF{#1}
%     {\let\fractype\flatfrac}
%     {\let\fractype\frac}
%     \IfNoValueTF{#4}
%     {
%         \IfNoValueTF{#6}
%         {\fractype{\partial \IfNoValueTF{#2}{}{^{#2}}}{\partial #3\IfNoValueTF{#2}{}{^{#2}}}}
%         {\fractype{\partial \IfNoValueTF{#2}{}{^{#2}}}{\partial #3\IfNoValueTF{#2}{}{^{#2}}} \argopen(#6\argclose)}
%     }
%     {
%         \IfNoValueTF{#5}
%         {\fractype{\partial \IfNoValueTF{#2}{}{^{#2}} #3}{\partial #4\IfNoValueTF{#2}{}{^{#2}}}}
%         {\fractype{\partial^2 #3}{\partial #4 \partial #5}}
%     }
% }
% \DeclareDocumentCommand\pderivative{}{\partialderivative} % Shorthand for \partialderivative
% \DeclareDocumentCommand\pdv{}{\partialderivative} % Shorthand for \partialderivative

\DeclareDocumentCommand\variation{ o g d() }{ % Functional variation
    % o: optional n for nth differential
    % g: optional argument for readability and to control spacing
    % d: long-form as in d(F(g))
    \IfNoValueTF{#2}{
        \IfNoValueTF{#3}
        {\delta \IfNoValueTF{#1}{}{^{#1}}}
        {\mathinner{\delta \IfNoValueTF{#1}{}{^{#1}}\argopen(#3\argclose)}}
    }
    {\mathinner{\delta \IfNoValueTF{#1}{}{^{#1}}#2} \IfNoValueTF{#3}{}{(#3)}}
}
\DeclareDocumentCommand\var{}{\variation} % Shorthand for \variation

\DeclareDocumentCommand\functionalderivative{ s o m g d() }
{ % Functional derivative
    % s: star for \flatfrac flat derivative
    % o: optional n for nth derivative
    % m: mandatory (g in dF/dg)
    % g: optional (F in dF/dg)
    % d: long-form d/dx(...)
    \IfBooleanTF{#1}
    {\let\fractype\flatfrac}
    {\let\fractype\frac}
    \IfNoValueTF{#4}
    {
        \IfNoValueTF{#5}
        {\fractype{\variation \IfNoValueTF{#2}{}{^{#2}}}{\variation #3\IfNoValueTF{#2}{}{^{#2}}}}
        {\fractype{\variation \IfNoValueTF{#2}{}{^{#2}}}{\variation #3\IfNoValueTF{#2}{}{^{#2}}} \argopen(#5\argclose)}
    }
    {\fractype{\variation \IfNoValueTF{#2}{}{^{#2}} #3}{\variation #4\IfNoValueTF{#2}{}{^{#2}}}}
}
\DeclareDocumentCommand\fderivative{}{\functionalderivative} % Shorthand for \functionalderivative
\DeclareDocumentCommand\fdv{}{\functionalderivative} % Shorthand for \functionalderivative


\begin{document}
\section*{Vector Calculus \hfill IA Lent}
\section{Differential Geometry of Curves}
\begin{itemize}
      \item Parametrisation of a curve, under some coordinate system (e.g. Cartesian):
            a function \( \vec{x}:[a,b] \to \mathbb{R}^3\) with
            \[\vec{x}(t) = \begin{pmatrix} x(t) \\  y(t) \\ z(t) \end{pmatrix} \]
      \item In Cartesians: Curve differentiable $=$ components differentiable
      \item Curve is regular: $|\mathbf{x'}(t)| \neq 0$
      \item Arc length of curve:
            \begin{enumerate}
                  \item Partition the interval \([a,b]\)
                  \item Calculate sum of straight line lengths at partition points
                  \item Take limit as maximum $\Delta t$ approaches zero
            \end{enumerate}
      \item Piecewise smooth curve: break down into integral of each piece
      \item Line element:
            \begin{align*}
                  \dd s & = \abs{\vec{x}'(t)} \, \dd t                                                                              \\
                        & = \sqrt{ \left( \odv{x}{t} \right)^2 +\left( \odv{y}{t} \right)^2 + \left( \odv{z}{t} \right)^2} \, \dd t
            \end{align*}

      \item Tangent, normal, binormal, curvature, torsion
\end{itemize}



\section{Integration}
\begin{itemize}
      \item Line integral: parametrise curve $C$ by $[a,b] \ni t \to \vec{x}(t)$
            \begin{itemize}

                  \item
                        Scalar field $f(\vec{x})$ \\
                        Scalar arc-length element: $\dd s = \abs{\vec{x}'(t)} \, \dd t $:

                        \[
                              \int_C f(\vec{x}) \, \dd s = \int_{a}^{b} f(\vec{x}(t)) \abs{\vec{x}'(t)} \, \dd t
                        \]

                  \item
                        Vector field $\vec{F}(\vec{x})$ \\
                        Vector line element: $\dd \vec{x} = \vec{x}'(t) \, \dd t$:

                        \[
                              \int_C \vec{F}(\vec{x}) \cdot \dd \vec{x} = \int_{a}^{b} \vec{F}(\vec{x}(t)) \cdot \vec{x}'(t) \, \dd t
                        \]

                  \item If the curve $C$ is a closed loop: closed integral/ circulation about $C$
                        \[\oint_C \vec{F}\cdot \dd \vec{x} \]
            \end{itemize}


      \item Area integral: given a region $D$ \\
            Scalar field $f(\vec{x})$ \\
            Scalar area element (if $D$ is divided into rectangular elements) \\
            $\dd A = \dd x \, \dd y$

            \begin{align*}
                  \iint_D f(\vec{x}) \, \dd A & = \int_y \int_{X_y} f(x,y) \, \dd x \, \dd y \\
                                              & = \int_x \int_{Y_x} f(x,y) \, \dd y \, \dd x
            \end{align*}
            where $X_y = \{x: (x,y) \in D\}$ (visualise horizontal strips), \\
            and $Y_x = \{y: (x,y) \in D\}$ (vertical strips)
      \item Jacobian: If $x= x(u,v)$ and $y=y(u,v)$ are smooth bijection from region $D'$ in $(u,v)$ plane to region $D$ in $(x,y)$ plane, then
            \[\iint_D f(x,y)  \, \dd x \, \dd y = \iint_{D'} f(x(u,v),y(u,v)) \, \abs{\pdv{(x,y)}{(u,v)}} \, \dd u \, \dd v\]

            \[\text{i.e. } \dd x \, \dd y = \abs{J} \, \dd u \, \dd v\]
            \[\textrm{where } J = \pdv{(x,y)}{(u,v)} =
                  \det \begin{pmatrix}
                        \pdv{x}{u} & \pdv{x}{v} \\
                        \pdv{y}{u} & \pdv{y}{v}
                  \end{pmatrix}
                  = \det\left( \pdv{\vec{x}}{u} \bigg| \pdv{\vec{x}}{v} \right)\]

            ($|J|$ is the "scale factor" for area)

      \item Volume integral
            \[\iiint_V f(\vec{x}) dV
            \]
            Analogue for Jacobian: \[J = \pdv{(x,y,z)}{(u,v,w)} = \det\left( \pdv{\vec{x}}{u} \bigg| \pdv{\vec{x}}{v} \bigg| \pdv{\vec{x}}{w} \right)\]
            Examples: \\
            Cylindrical polars: \[ \dd V =  \dd x \, \dd y \, \dd z = \rho \, \dd \rho \, \dd \theta \, \dd z \]
            Spherical polars: \[ \dd V = \dd x \, \dd y \, \dd x = r^2 \sin \theta \, \dd r \, \dd \theta \, \dd \phi \]
      \item Surface integral: \\
            Given surface defined by $S = \{ \vec{x}: f(\vec{x}) = 0 \}$: $\nabla  f$ is normal to the surface \\
            Given parametrised surface $S = \{ \vec{x}: \vec{x}(u,v) : (u,v) \in D\}$ for some region $D$ in $(u,v)$ plane \\
            Normal vector (of unit length) \[\vec{n} = \frac{\pdv{\vec{x}}{u} \times \pdv{\vec{x}}{v}} {\abs{\pdv{\vec{x}}{u} \times \pdv{\vec{x}}{v}}} \]

            Scalar and vector area element:
            \[\dd S = \abs{\pdv{\vec{x}}{u} \times \pdv{\vec{x}}{v}} \, \dd u \, \dd v \]
            \[\dd \vec{S} = \pdv{\vec{x}}{u} \times \pdv{\vec{x}}{v} \, \dd u \, \dd v = \vec{n} \, \dd S
            \]
\end{itemize}

\section{Div, Grad, Curl and Laplacian: In Cartesians}
\begin{itemize}
      \item Gradient (of scalar field): $\nabla  f$
            \[ [\nabla  f]_i = \pdv{f}{x_i}\]
      \item Gradient of vector field
      \item Directional derivative: $ D_{\vec{v}} f = \vec{v} \cdot \nabla  f$
            \[ \vec{v} \cdot \nabla  f = v_i \pdv{f}{x_i}\]
      \item Divergence: $ \nabla \cdot \vec{F}$
            \[ \nabla \cdot \vec{F} = \pdv{F_i}{x_i}\]
      \item Curl:  $ \nabla \times \vec{F}$
            \[ [\nabla \times \vec{F}]_i = \epsilon_{ijk} \pdv{x_j} F_k\]
      \item Laplacian (scalar): (div grad) $\nabla ^2 f = \nabla \cdot \nabla f$
            \[ \nabla ^2 f = \pdv{f}{x_i}{x_i} \]
      \item Laplacian (vector field): $\nabla ^2 \vec{F} = \nabla (\nabla \cdot \vec{F}) - \nabla \times (\nabla \times \vec{F})$
            \[ [\nabla ^2 \vec{F}]_i = (\nabla^2 F_i){\vec{e}_i} \]
      \item div $\circ$ curl $ = \nabla \cdot \nabla \times \vec{F} = 0 $
      \item curl $\circ$ grad $ = \nabla \times \nabla f = \vec{0} $
\end{itemize}

\section{Integration Theorems}
\begin{itemize}
      \item Green's Theorem: \\
            If $P$ and $Q$ are continuously differentiable scalar fields on $A \subset R^2$ and $\partial A$ is made of collection of smooth curves, then
            \[ \oint_{\partial A} P \, \dd{x} + Q \, \dd y = \iint_A \left( \pdv{Q}{x} - \pdv{P}{y} \right) \, \dd x \, \dd y \]

      \item Stokes' Theorem: \\
            If $\vec{F}(\vec{x})$ is a continuously differentiable vector field, $S$ an orientable surface with $\partial S$ piecewise regular boundary, then
            \[\int_S ({\vec{\nabla \times F})\cdot \dd \vec{S} = \oint_{\partial S} \vec{F} \cdot \dd \vec{x}}\]

      \item Divergence/Gauss' Theorem: \\ (3D)
            If $\vec{F}(\vec{x})$ is a continuously differentiable vector field, $V$ a volume with $\partial V$ piecewise regular boundary, then
            \[\int_V \nabla \cdot\vec{F} \, \dd V = \int_{\partial V} \vec{F} \cdot \dd \vec{S}\]

      \item (2D): (normal points out of the region)
            \[ \int_D \nabla \cdot\vec{F} \, \dd A = \oint_{\partial D} \vec{F} \cdot \vec{n} \, \dd s
            \]
\end{itemize}

\section{Maxwell's Equations}
\section{Poisson's Equation}

\begin{itemize}
      \item Laplace's equation (Forcing = 0)
      \item Dirichlet and Neumann condition, with uniqueness
\end{itemize}


\section{Cartesian Tensors}
\begin{itemize}
      \item Definition: A tensor $T$ of rank $n$ has components $T_{\underbrace{ij\dots k}_{n \text{ indices}}}$ transforms from the right handed orthonormal basis ${\vec{e}_i}$ to ${\vec{e}'_i}$ under the law \[T'_{ij\dots k} = R_{ip}R_{jq}\dots R_{kr} T_{pq\dots r}\] (where $R_ij$ are components of a rotation matrix)
      \item Scalar (rank 0), vector (rank 1), linear map
      \item Scaling and adding tensors (of same order) give tensors
      \item Tensor product: $(T \otimes S)_{\underbrace{ij\dots k}_{n \text{ indices}}\underbrace{pq\dots r}_{m \text{ indices}}} = T_{ij\dots k}S_{pq\dots r}$
      \item Contraction: contracting on indices $i$ and $j$: $S_{\underbrace{p \dots q}_{n-2}} = \delta_{ij} T_{ijp \dots q}$
      \item Symmetric, antisymmetric (in components in a pair of components); totally symmetric/antisymmetric
      \item Tensor field of rank $n$: $T_{ij\dots k}(\vec{x}_0)$ gives a tensor at each point in space (say $\R^3$), e.g. scalar field, vector field
      \item Differentiating tensor fields: \[ \pdiff{x_i'} = R_{ij} \pdiff{x_j}\] (each derivative gives a factor $R$)
      \item $m$-th partial derivative of rank $n$ tensor is a tensor of rank $m+n$: \[ \left( \pdiff{x_p} \right) \left( \pdiff {x_q} \right) \dots \left( \pdiff {x_r} \right) T_{ij \dots k}(\vec{x}) \]
      \item Divergence theorem: \[ \int_V \pdv{T_{ij \dots k \dots l}}{x_k} \dd V = \int_{\pdv{V}} T_{ij \dots k \dots l} n_k \dd S \]
      \item Rank 2 tensor: decompose (uniquely) into symmetric and antisymmetric parts:
            \[T_{ij} = \underbrace{\frac{1}{2}(T_{ij} + T_{ji})}_{S_{ij}} + \underbrace{\frac{1}{2}(T_{ij} - T_{ji})}_{A_{ij}} \]

            And the antisymmetric part \[A_{ij}= \varepsilon_{ijk} \omega_k\]
            where $\omega_i = \frac{1}{2} \varepsilon_{ijk}T_{jk}$. \\
            Symmetric part: \[ S_{ij} = \underbrace{P_ij}_{\text{traceless}} + \underbrace{\frac{1}{3}\delta_{ij} S_kk}_{\text{isotropic}} \]
      \item For any symmetric second rank tensor, there exist a choice of right handed Cartesian coords where the matrix is diagonal (real symmetric matrix can be orthogonally diagonalised)
      \item Isotropic tensor: invariant under any change of bases :
            \[ T_{ij\dots k}'=R_{ip}R_{jq}\dots R_{kr} T_{pq\dots r} = T_{ij\dots k}\]
      \item Correspondence between multi-linear maps and tensors: \\
            If a multi-linear map (well-defined, independent of basis) is given by
            \begin{align*}
                  \underbrace{\R^3 \times \R^3 \times \dots \times \R^3}_n & \to \R                        \\
                  (\vec{a}, \vec{b} , \dots \vec{c})                       & \to T_{ij\dots k} a_i b_j c_k
            \end{align*}
            Then $T_{ij\dots k}$ is a rank $n$ tensor.
      \item Quotient theorem:\\
            Given array $T_{i\dots jp\dots q}$, if \[ v_{i\dots j} := T_{i\dots jp\dots q} u_{p \dots q}\] is a tensor for \emph{any} tensor $u_{p\dots q}$, then $T_{i\dots jp\dots q}$ are components of a tensor.
\end{itemize}
\end{document}
