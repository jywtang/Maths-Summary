\documentclass[12pt, a4paper]{article}
\usepackage[T1]{fontenc}
\usepackage{amsmath, amssymb, amsthm}
\usepackage{babel}
\usepackage{siunitx}
\usepackage{tikz}
\usepackage{centernot}
\usepackage{tcolorbox}
\usepackage{cancel}
\usepackage{enumitem}
\usepackage{xparse}

\usetikzlibrary{arrows}

\setlength{\parindent}{0pt}


% Matrix groups
\newcommand{\GL}{\mathrm{GL}}
\newcommand{\M}{\mathrm{M}}
\newcommand{\Or}{\mathrm{O}}
\newcommand{\PGL}{\mathrm{PGL}}
\newcommand{\PSL}{\mathrm{PSL}}
\newcommand{\PSO}{\mathrm{PSO}}
\newcommand{\PSU}{\mathrm{PSU}}
\newcommand{\SL}{\mathrm{SL}}
\newcommand{\SO}{\mathrm{SO}}
\newcommand{\Spin}{\mathrm{Spin}}
\newcommand{\Sp}{\mathrm{Sp}}
\newcommand{\SU}{\mathrm{SU}}
\newcommand{\U}{\mathrm{U}}
\newcommand{\Mat}{\mathrm{Mat}}


% Special sets
\newcommand{\C}{\mathbb{C}}
\newcommand{\CP}{\mathbb{CP}}
\newcommand{\F}{\mathbb{F}}
\newcommand{\GG}{\mathbb{G}}
\newcommand{\N}{\mathbb{N}}
% \newcommand{\P}{\mathbb{P}}
\newcommand{\Q}{\mathbb{Q}}
\newcommand{\R}{\mathbb{R}}
\newcommand{\RP}{\mathbb{RP}}
\newcommand{\T}{\mathbb{T}}
\newcommand{\Z}{\mathbb{Z}}
\renewcommand{\H}{\mathbb{H}}

% Brackets
\renewcommand{\vec}[1]{\boldsymbol{\mathbf{#1}}}
\newcommand{\cis}[1]{ \cos\left( #1 \right) + i \sin \left( #1 \right)}

% Algebra
\DeclareMathOperator{\adj}{adj}
\DeclareMathOperator{\Ann}{Ann}
\DeclareMathOperator{\Aut}{Aut}
\DeclareMathOperator{\Char}{char}
\DeclareMathOperator{\disc}{disc}
\DeclareMathOperator{\dom}{dom}
\DeclareMathOperator{\fix}{fix}
\DeclareMathOperator{\Hom}{Hom}
\DeclareMathOperator{\id}{id}
\DeclareMathOperator{\image}{image}
\DeclareMathOperator{\im}{im}
\DeclareMathOperator{\tr}{tr}
\newcommand{\Bilin}{\mathrm{Bilin}}
\newcommand{\Frob}{\mathrm{Frob}}



\let\Im\relax
\let\Re\relax


\DeclareMathOperator{\hcf}{hcf}
\DeclareMathOperator{\Isom}{Isom}
\DeclareMathOperator{\lcm}{lcm}
\DeclareMathOperator{\sgn}{sgn}
\DeclareMathOperator{\supp}{supp}
\DeclareMathOperator{\Sym}{Sym}
\DeclareMathOperator{\Syl}{Syl}
\DeclareMathOperator{\Im}{Im}
\DeclareMathOperator{\Re}{Re}
\DeclareMathOperator{\Ker}{Ker}


% Theorems
\theoremstyle{definition}
\newtheorem*{aim}{Aim}
\newtheorem*{axiom}{Axiom}
\newtheorem*{claim}{Claim}
\newtheorem*{cor}{Corollary}
\newtheorem*{conjecture}{Conjecture}
\newtheorem*{defi}{Definition}
\newtheorem*{eg}{Example}
\newtheorem*{ex}{Exercise}
\newtheorem*{fact}{Fact}
\newtheorem*{law}{Law}
\newtheorem*{lemma}{Lemma}
\newtheorem*{notation}{Notation}
\newtheorem*{prop}{Proposition}
\newtheorem*{question}{Question}
\newtheorem*{rrule}{Rule}
\newtheorem*{thm}{Theorem}
\newtheorem*{assumption}{Assumption}

\newtheorem*{remark}{Remark}
\newtheorem*{warning}{Warning}
\newtheorem*{exercise}{Exercise}

\newtheorem{nthm}{Theorem}[section]
\newtheorem{nlemma}[nthm]{Lemma}
\newtheorem{nprop}[nthm]{Proposition}
\newtheorem{ncor}[nthm]{Corollary}


\newcommand{\abs}[1]{\left| #1 \right|} % for absolute value
\newcommand{\grad}[1]{\mathbf{\nabla} #1} % for gradient
\let\divsymb=\div % rename builtin command \div to \divsymb
\renewcommand{\div}[1]{\mathbf{\nabla} \cdot #1} % for divergence
\newcommand{\curl}[1]{\mathbf{\nabla} \times #1} % for curl




% Derivatives

\newcommand{\dd}[1][]{\mathrm{d} #1}
\newcommand{\odiff}[1]{\frac{\dd}{\dd{#1}}}
\newcommand{\odv}[2]{\frac{\dd{#1}}{\dd{#2}}}
\newcommand{\pdiff}[1]{\frac{\partial}{\partial{#1}}}
\newcommand{\pdv}[2]{\frac{\partial{#1}}{\partial{#2}}}


% \def\diffd{\mathrm{d}}
% \DeclareDocumentCommand\differential{ o g d() }{ % Differential 'd'
%     % o: optional n for nth differential
%     % g: optional argument for readability and to control spacing
%     % d: long-form as in d(cos x)
%     \IfNoValueTF{#2}{
%         \IfNoValueTF{#3}
%         {\diffd\IfNoValueTF{#1}{}{^{#1}}}
%         {\mathinner{\diffd\IfNoValueTF{#1}{}{^{#1}}\argopen(#3\argclose)}}
%     }
%     {\mathinner{\diffd\IfNoValueTF{#1}{}{^{#1}}#2} \IfNoValueTF{#3}{}{(#3)}}
% }
% \DeclareDocumentCommand\dd{}{\differential} % Shorthand for \differential

% \DeclareDocumentCommand\derivative{ s o m g d() }
% { % Total derivative
%     % s: star for \flatfrac flat derivative
%     % o: optional n for nth derivative
%     % m: mandatory (x in df/dx)
%     % g: optional (f in df/dx)
%     % d: long-form d/dx(...)
%     \IfBooleanTF{#1}
%     {\let\fractype\flatfrac}
%     {\let\fractype\frac}
%     \IfNoValueTF{#4}
%     {
%         \IfNoValueTF{#5}
%         {\fractype{\diffd \IfNoValueTF{#2}{}{^{#2}}}{\diffd #3\IfNoValueTF{#2}{}{^{#2}}}}
%         {\fractype{\diffd \IfNoValueTF{#2}{}{^{#2}}}{\diffd #3\IfNoValueTF{#2}{}{^{#2}}} \argopen(#5\argclose)}
%     }
%     {\fractype{\diffd \IfNoValueTF{#2}{}{^{#2}} #3}{\diffd #4\IfNoValueTF{#2}{}{^{#2}}}}
% }
% \DeclareDocumentCommand\dv{}{\derivative} % Shorthand for \derivative

% \DeclareDocumentCommand\partialderivative{ s o m g g d() }
% { % Partial derivative
%     % s: star for \flatfrac flat derivative
%     % o: optional n for nth derivative
%     % m: mandatory (x in df/dx)
%     % g: optional (f in df/dx)
%     % g: optional (y in d^2f/dxdy)
%     % d: long-form d/dx(...)
%     \IfBooleanTF{#1}
%     {\let\fractype\flatfrac}
%     {\let\fractype\frac}
%     \IfNoValueTF{#4}
%     {
%         \IfNoValueTF{#6}
%         {\fractype{\partial \IfNoValueTF{#2}{}{^{#2}}}{\partial #3\IfNoValueTF{#2}{}{^{#2}}}}
%         {\fractype{\partial \IfNoValueTF{#2}{}{^{#2}}}{\partial #3\IfNoValueTF{#2}{}{^{#2}}} \argopen(#6\argclose)}
%     }
%     {
%         \IfNoValueTF{#5}
%         {\fractype{\partial \IfNoValueTF{#2}{}{^{#2}} #3}{\partial #4\IfNoValueTF{#2}{}{^{#2}}}}
%         {\fractype{\partial^2 #3}{\partial #4 \partial #5}}
%     }
% }
% \DeclareDocumentCommand\pderivative{}{\partialderivative} % Shorthand for \partialderivative
% \DeclareDocumentCommand\pdv{}{\partialderivative} % Shorthand for \partialderivative

\DeclareDocumentCommand\variation{ o g d() }{ % Functional variation
    % o: optional n for nth differential
    % g: optional argument for readability and to control spacing
    % d: long-form as in d(F(g))
    \IfNoValueTF{#2}{
        \IfNoValueTF{#3}
        {\delta \IfNoValueTF{#1}{}{^{#1}}}
        {\mathinner{\delta \IfNoValueTF{#1}{}{^{#1}}\argopen(#3\argclose)}}
    }
    {\mathinner{\delta \IfNoValueTF{#1}{}{^{#1}}#2} \IfNoValueTF{#3}{}{(#3)}}
}
\DeclareDocumentCommand\var{}{\variation} % Shorthand for \variation

\DeclareDocumentCommand\functionalderivative{ s o m g d() }
{ % Functional derivative
    % s: star for \flatfrac flat derivative
    % o: optional n for nth derivative
    % m: mandatory (g in dF/dg)
    % g: optional (F in dF/dg)
    % d: long-form d/dx(...)
    \IfBooleanTF{#1}
    {\let\fractype\flatfrac}
    {\let\fractype\frac}
    \IfNoValueTF{#4}
    {
        \IfNoValueTF{#5}
        {\fractype{\variation \IfNoValueTF{#2}{}{^{#2}}}{\variation #3\IfNoValueTF{#2}{}{^{#2}}}}
        {\fractype{\variation \IfNoValueTF{#2}{}{^{#2}}}{\variation #3\IfNoValueTF{#2}{}{^{#2}}} \argopen(#5\argclose)}
    }
    {\fractype{\variation \IfNoValueTF{#2}{}{^{#2}} #3}{\variation #4\IfNoValueTF{#2}{}{^{#2}}}}
}
\DeclareDocumentCommand\fderivative{}{\functionalderivative} % Shorthand for \functionalderivative
\DeclareDocumentCommand\fdv{}{\functionalderivative} % Shorthand for \functionalderivative


\begin{document}
\section*{Methods \hfill IB}
\section{Self-adjoint ODEs}
\subsection*{Fourier Series}
\begin{itemize}
    \item In the vector space of 'nice enough' periodic functions: $\R \to \R$ with inner product $\langle f,g \rangle = \int_{-L}^{L} f(x) g(x) \dd{x} $, the functions $\cos\left(\frac{n\pi x}{L}\right)$ and $\sin\left(\frac{n\pi x}{L}\right)$ form a countable complete orthonormal basis
    \item Dirichlet conditions:
          \begin{enumerate}
              \item Finitely many bouned discontinuties
              \item Finitely many extrema in one period
              \item Absolutely integrable (Bounded?)
          \end{enumerate}
    \item Fourier series: \[f(x) = \frac{a_0}{2} + \sum_{n=1}^\infty \left( a_n \cos\left(\frac{n\pi x}{L} \right) + b_n \sin\left(\frac{n\pi x}{L}\right) \right),\]
          \begin{align*}
              \text{ where } a_n & = \frac{1}{L} \int_{-L}^L f(x) \cos\left(\frac{n\pi x}{L}\right) \dd{x}, \\
              b_n                & = \frac{1}{L} \int_{-L}^L f(x) \sin\left(\frac{n\pi x}{L}\right) \dd{x}
          \end{align*}
    \item Half-range series
    \item Fourier series converge to average of left and right limits at discontinuities
    \item Order of coefficients related to differentiability of $f$ and discontinuties: \\
          $f(p)$ discts $\implies$ $a_n,b_n = O(n^{-(p+1)})$
    \item Can always integrate, but not always differentiate
\end{itemize}

Can also formulate using complex exponential as basis over periodic function $\R \to \C$ with inner product $\langle f,g \rangle = \int_{-L}^{L} f^*(x) g(x) \dd{x} $:
\[f(x) = \sum_{n=-\infty}^\infty c_n e^{\frac{i n \pi x}{L}}, \quad \text{ where } c_n = \frac{1}{2L}\int_{-L}^{L} f(x) e^{\frac{-i n \pi x}{L}} \dd{x}\]

\begin{itemize}
    \item Parseval's Theorem: \[\int_{-L}^L \abs{f}^2 \dd x = 2L \sum_{n=-\infty}^\infty \abs{c_n}^2 = L \left( \frac{a_0^2}{2} + \sum_{n=1}^\infty (a_n^2+b_n^2) \right)\]
\end{itemize}

\subsection*{Sturm-Liouville Theory}
Background: (DE, LA)
\begin{itemize}
    \item General solution to DE = Particular + Complementary ($n$ linearly indep and spanning), uniqueness by boundary/initial conditions
    \item On inner product space $V$, a self-adjoint map (i.e. hermitian matrices for finite dim) $M:V \to V$ satisfies $\langle Mu,v\rangle = \langle u, Mv \rangle \, \forall u,v, \in V$
    \item Self-adjoint maps are orthogonally diagonalisable, can use eigenvector basis to solve $Mx=b$ (get coeffs)
\end{itemize}

S-L Theory: Consider linear differential operators as linear maps between inner product function spaces:

Aim: Solve 2nd order linear ODE: \[\mathcal{D}y = \alpha y''+\beta y' + \gamma y = f, \quad a \leq x \leq b\] and two boundary conditions

\begin{itemize}
    \item If $\alpha \neq 0$, any eigenvalue problem ($\mathcal{D}y = \lambda y$) becomes $\mathcal{L}y = -(py')'+qy = \lambda wy$, weight function $w$ with countably many zeroes
    \item S-L form is self-adjoint iff $ \left[ p\left( \bar{y_1}' y_2 - \bar{y_1}y_2'\right) \right]_a^b = 0$, for all $y_1, y_2$ that satisfy BC, includes many types (e.g. homogeneous Dirichlet, Neumann, mixed)
\end{itemize}

Good properties of self-adjoint operators:
\begin{itemize}
    \item If $\mathcal{L}$ is self-adjoint, get countably infinite, complete basis of orthogonal (orthonormal) simple eigenfunctions with real eigenvalues: $\mathcal{L}y_n = \lambda_n wy_n$, orthogonal in the sense $\langle u,v \rangle_w = \int_a^b w (\bar{u} v) \dd{x}$
    \item Can use basis to solve for particular integral, if forcing has same BC as solution
\end{itemize}

\begin{itemize}
    \item Parseval's Theorem for eigenfunction expansion: if $f(x) = \sum_{n\in\N} a_n y_n$, then $\int_a^b w \left(f(x)\right)^2 \dd{x} = \sum_{n\in\N} (a_n)^2 \norm{y_n}^2_w$ (norm of $y_n$ is integral with weight)
    \item Bessel's inequality, for incomplete set of eigenfunctions
    \item Least squares approx: \\
          Using a finite set of eigenfunctions $S_N(x) = \sum_{n=1}^N b_n y_n$ to approximate $f(x) = \sum_{n\in\N} a_n y_n$: choosing $b_n = a_n$ minimises mean squared error $\varepsilon_N = \int_a^b w \left[ f - S_N\right]^2 \dd{x}$
\end{itemize}

Can solve forced problems
Taking product with Green's functions 'inverts' L


\section{PDEs, Separation of variables}

\subsection*{Wave equation}

\[\text{1D: } y_{tt} = c^2 y_{xx}\]
\[\text{2D: } \pdv[2]{\phi}{t} = c^2 \grad^2{\phi}, c=\sqrt{T/\mu}\]
\begin{itemize}

    \item SOV: if $y = X(x)T(t)$ two sides depends on different variables $\implies$ both sides constant.

    \item If given BC: homogeneous Dirichlet at end points, SOV give eigenvalue problems; solutions must be quantised sine wave, called the normal modes
          \[y(x,t) = \sum_{n=1}^{\infty} \left(a_n \cos \left(\frac{n \pi ct}{L}\right) + b_n \sin \left(\frac{n \pi ct}{L}\right)\right) \sin \left(\frac{n \pi x}{L} \right) \]

    \item IC: initial position and velocity; determine the coefficient of each separable solution using Fourier

    \item Energy = sum of squares of coeffs, indep of time (use orthogonality)

    \item Reflection at boundary of changing mass density, three waves: incident, transmitted and reflected; $y, y'$ continuous at boundary at all time; phase shift

\end{itemize}


Drum problem:

\begin{itemize}
    \item Wave equation in 2D plane polars, with fixed circular boundary conditions $u(r=1,\theta,t) = 0$\\
          Suppose solutions separable: $u = R(r) \Theta(\theta) T(t)$

    \item Equation for $R$ gives Bessel's equation \[z^2 \odv[2]{R}{z} + z \odv{R}{z} +(z^2-m^2)R = 0\]

    \item Power series solutions: Bessel/Neumann functions ($J_m(x), Y_m(x)$)

    \item Regularity (boundedness) at r = 0 forces only Bessel function; so solution for space component: \[X_{mn}(r,\theta) = J_m(j_{mn} r) (A_n \cos(m\theta) + B_n \sin(m\theta))\]

    \item General space solution = linear combination of $X_{mn}$:
    \item Solution for $u$: \begin{align*}
              u(r,\theta,t) = & \sum_{n=1}^{\infty} J_0(j_{0n} r) \left( A_{0n} \cos (j_{0n} ct) +  C_{0n} \sin (j_{0n} ct)  \right)                                \\
                              & \sum_{m=1}^{\infty} \sum_{n=1}^{\infty} J_m(j_{mn} r) (A_{mn} \cos(m\theta) + B_{mn} \sin(m\theta)) \left( \cos (j_{mn} ct) \right) \\
              +               & \sum_{m=1}^{\infty} \sum_{n=1}^{\infty} J_m(j_{mn} r) (C_{mn} \cos(m\theta) + D_{mn} \sin(m\theta)) \left( \sin (j_{mn} ct) \right)
          \end{align*}
          IC determine coefficients by orthogonality

\end{itemize}

\subsection*{Diffusion equation}
\begin{itemize}
    \item Flux proportional to diffusion gradient, and opposite direction: \[\vec{q} = -k \grad \theta\]
    \item Changes in total heat = - (net flux outwards)
    \item Diffusion equation: \[\pdv{\theta}{t} = D \grad^2 \theta\]
    \item Alternative derivation from random walk
    \item Error function is a solution without BC; good initial approximation for solution with BCs
    \item SOV: reformulate problem into homogeneous, get transient solution


\end{itemize}

\subsection*{Laplace's equation}
\begin{itemize}
    \item Cartesian, Plane polar, Cylindrical polar, Spherical polar

    \item Spherical polar, assume axisymmetric, equation for $\Theta$ give Legendre's equation by substitution $x= \cos\theta \in [-1,1]$:
          \[-\odiff{x} \left((1-x^2) \odv{\Theta}{x} \right) = \lambda \Theta\]

    \item Series solution, bounded at endpoints $\implies$ solutions are polynomials, with $\lambda_l = l(l+1)$

    \item Legendre polynomials: \\
          Scaling convention $P_l(1)=1$, orthogonal, \[\int_{-1}^1 P_n(x) \dd{x} = \frac{2}{2n+1}\]

    \item Generating function

\end{itemize}

\section{Generalised function, Inhom ODE}

\begin{itemize}
    \item Dirac Delta: $\delta(x)$: defined by its property \[\int_{-\infty}^{\infty} \delta(x) f(x) \dd{x} = f(0)\] (sampling property)
    \item Dirac Delta as a limit of discrete functions
          \[ \delta_n(x) = \begin{cases}
                  0   & , \abs{x} > 1/n    \\
                  n/2 & , \abs{x} \leq 1/n
              \end{cases} \]

    \item Dirac Delta as a limit of continuous Gaussian functions, mean 0, var $\varepsilon^2$:
          \[\delta_\varepsilon(x) = \frac{1}{\varepsilon \sqrt{\pi}} e^{-x^2/\varepsilon^2}\]


    \item Properties:  $\delta(ax)$,  $\delta(f(x))$, integral of $\delta'(x)$
    \item Write Dirac Delta as a Fourier series (Dirac comb): if $f(x) = \delta(x)$ on $-L<x<L$, then \[\delta(x) = \frac{1}{2L} \sum_{n=-\infty}^{\infty} c_n e^{in\pi x/L}\]
    \item Using Dirac Delta to define Green's function: solution to \[\mathcal{L}G(x;\xi) = \delta(x-\xi)\] with homogeneous BC
\end{itemize}



\section{Fourier Transform}
Fourier Transform:
\[\tilde{f}(k) = \mathcal{F}[f](k) = \int_{-\infty}^\infty f(x)e^{-ikx} \dd x\]
Inverse Fourier Transform:
\[f(x) = \mathcal{F}^{-1}[\tilde{f}](x) = \frac{1}{2\pi} \int_{-\infty}^\infty \tilde{f}(k)e^{ikx} \dd k\]
(Dual property)

Properties:
\begin{itemize}
    \item FT of xf(x)
    \item FT of f': $\tilde{f'}(k) = ik \tilde{f}(k)$
    \item Convolution: $h(x) = (f*g)(x) = \int_{-\infty}^\infty f(x-u)g(u) \dd{u}$ iff $\tilde{h} =\tilde{f}\tilde{g}$
          Parseval's Theorem:
          \[\int_{-\infty}^\infty \abs{f(u)}^2 \dd{u} = \frac{1}{2 \pi} \int_{\infty}^\infty \abs{\tilde{f} (k)}^2 \dd{k}\]
    \item FT of trig, Heaviside, Delta
    \item Express Dirac Delta as a FT: $delta \longleftrightarrow 1$
\end{itemize}


\section{PDEs}
Cauchy problem: include BVP/IVP as auxiliary data (Cauchy data)

Well posed problem:
\begin{itemize}
    \item Solution exists
    \item Solution is unique
    \item Solution depends continuously on auxiliary data
\end{itemize}

\subsection*{Method of characteristics}
1st order,
Change coordinates along characteristics and along boundary curve

\subsection*{Classification of 2nd order PDEs}
\begin{itemize}
    \item General form with coefficient functions, compute value of $b^2-ac$ to classify: \\
          \begin{itemize}
              \item >0: hyperbolic, two real characteristics, e.g. wave
              \item =0: parabolic, one real characteristics, e.g. heat
              \item >0: elliptic, complex characteristics, e.g. Laplace
          \end{itemize}
    \item Equations for characteristics
    \item Hyperbolic equation: use two sets of characteristics as coordinates, canonical form for the equation
    \item d'Alembert's solution for wave equation: change variable to $x+ct$ and $x-ct$, get solution \[u = f(x+ct) + g(x-ct)\]

\end{itemize}


Solving PDEs with FT
Heat equation: FT wrt x, solve, convolution of IC and source function/fundamental solution/diffusion kernel \[S_d(x,t)=\frac{1}{\sqrt{4\pi Dt}} e^\frac{-x^2}{4Dt}\]

Duhamel's principle:
\begin{itemize}
    \item (Hyperbolic/parabolic, i.e. wave/heat equation)
    \item Greens function (forcing is product of deltas); \[G=H(t-\tau) S_d(x-\xi,t-\tau)\]
    \item Built in causal relationship
\end{itemize}


Poisson's equation (forced Laplace's):
\begin{itemize}
    \item Higher dim delta function (higher dim intergation)
    \item Solve with delta forcing: \[\nabla^2 G(\vec{r};\vec{r_0}) = \delta(\vec{r-r_0})\]
    \item Divergence theorem; get formula for free space Green's function (different in 3D and 2D)
    \item Dirichlet Greens function: vanish on boundary, contains free space Green's, the remaining part harmonic in domain  
\end{itemize}

\subsection*{Method of Images}
Do something outside of domain to eliminate boundary condition (e.g. another heat/cold source) to satisfy Dirichlet/Neumann(homogeneous) boundary conditions to construct Green's function 


\end{document}