\documentclass[12pt, a4paper]{article}
\usepackage[T1]{fontenc}
\usepackage{amsmath, amssymb, amsthm}
\usepackage{babel}
\usepackage{siunitx}
\usepackage{tikz}
\usepackage{centernot}
\usepackage{tcolorbox}
\usepackage{cancel}
\usepackage{enumitem}
\usepackage{xparse}

\usetikzlibrary{arrows}

\setlength{\parindent}{0pt}


% Matrix groups
\newcommand{\GL}{\mathrm{GL}}
\newcommand{\M}{\mathrm{M}}
\newcommand{\Or}{\mathrm{O}}
\newcommand{\PGL}{\mathrm{PGL}}
\newcommand{\PSL}{\mathrm{PSL}}
\newcommand{\PSO}{\mathrm{PSO}}
\newcommand{\PSU}{\mathrm{PSU}}
\newcommand{\SL}{\mathrm{SL}}
\newcommand{\SO}{\mathrm{SO}}
\newcommand{\Spin}{\mathrm{Spin}}
\newcommand{\Sp}{\mathrm{Sp}}
\newcommand{\SU}{\mathrm{SU}}
\newcommand{\U}{\mathrm{U}}
\newcommand{\Mat}{\mathrm{Mat}}


% Special sets
\newcommand{\C}{\mathbb{C}}
\newcommand{\CP}{\mathbb{CP}}
\newcommand{\F}{\mathbb{F}}
\newcommand{\GG}{\mathbb{G}}
\newcommand{\N}{\mathbb{N}}
% \newcommand{\P}{\mathbb{P}}
\newcommand{\Q}{\mathbb{Q}}
\newcommand{\R}{\mathbb{R}}
\newcommand{\RP}{\mathbb{RP}}
\newcommand{\T}{\mathbb{T}}
\newcommand{\Z}{\mathbb{Z}}
\renewcommand{\H}{\mathbb{H}}

% Brackets
\renewcommand{\vec}[1]{\boldsymbol{\mathbf{#1}}}
\newcommand{\cis}[1]{ \cos\left( #1 \right) + i \sin \left( #1 \right)}

% Algebra
\DeclareMathOperator{\adj}{adj}
\DeclareMathOperator{\Ann}{Ann}
\DeclareMathOperator{\Aut}{Aut}
\DeclareMathOperator{\Char}{char}
\DeclareMathOperator{\disc}{disc}
\DeclareMathOperator{\dom}{dom}
\DeclareMathOperator{\fix}{fix}
\DeclareMathOperator{\Hom}{Hom}
\DeclareMathOperator{\id}{id}
\DeclareMathOperator{\image}{image}
\DeclareMathOperator{\im}{im}
\DeclareMathOperator{\tr}{tr}
\newcommand{\Bilin}{\mathrm{Bilin}}
\newcommand{\Frob}{\mathrm{Frob}}



\let\Im\relax
\let\Re\relax


\DeclareMathOperator{\hcf}{hcf}
\DeclareMathOperator{\Isom}{Isom}
\DeclareMathOperator{\lcm}{lcm}
\DeclareMathOperator{\sgn}{sgn}
\DeclareMathOperator{\supp}{supp}
\DeclareMathOperator{\Sym}{Sym}
\DeclareMathOperator{\Syl}{Syl}
\DeclareMathOperator{\Im}{Im}
\DeclareMathOperator{\Re}{Re}
\DeclareMathOperator{\Ker}{Ker}


% Theorems
\theoremstyle{definition}
\newtheorem*{aim}{Aim}
\newtheorem*{axiom}{Axiom}
\newtheorem*{claim}{Claim}
\newtheorem*{cor}{Corollary}
\newtheorem*{conjecture}{Conjecture}
\newtheorem*{defi}{Definition}
\newtheorem*{eg}{Example}
\newtheorem*{ex}{Exercise}
\newtheorem*{fact}{Fact}
\newtheorem*{law}{Law}
\newtheorem*{lemma}{Lemma}
\newtheorem*{notation}{Notation}
\newtheorem*{prop}{Proposition}
\newtheorem*{question}{Question}
\newtheorem*{rrule}{Rule}
\newtheorem*{thm}{Theorem}
\newtheorem*{assumption}{Assumption}

\newtheorem*{remark}{Remark}
\newtheorem*{warning}{Warning}
\newtheorem*{exercise}{Exercise}

\newtheorem{nthm}{Theorem}[section]
\newtheorem{nlemma}[nthm]{Lemma}
\newtheorem{nprop}[nthm]{Proposition}
\newtheorem{ncor}[nthm]{Corollary}


\newcommand{\abs}[1]{\left| #1 \right|} % for absolute value
\newcommand{\grad}[1]{\mathbf{\nabla} #1} % for gradient
\let\divsymb=\div % rename builtin command \div to \divsymb
\renewcommand{\div}[1]{\mathbf{\nabla} \cdot #1} % for divergence
\newcommand{\curl}[1]{\mathbf{\nabla} \times #1} % for curl




% Derivatives

\newcommand{\dd}[1][]{\mathrm{d} #1}
\newcommand{\odiff}[1]{\frac{\dd}{\dd{#1}}}
\newcommand{\odv}[2]{\frac{\dd{#1}}{\dd{#2}}}
\newcommand{\pdiff}[1]{\frac{\partial}{\partial{#1}}}
\newcommand{\pdv}[2]{\frac{\partial{#1}}{\partial{#2}}}


% \def\diffd{\mathrm{d}}
% \DeclareDocumentCommand\differential{ o g d() }{ % Differential 'd'
%     % o: optional n for nth differential
%     % g: optional argument for readability and to control spacing
%     % d: long-form as in d(cos x)
%     \IfNoValueTF{#2}{
%         \IfNoValueTF{#3}
%         {\diffd\IfNoValueTF{#1}{}{^{#1}}}
%         {\mathinner{\diffd\IfNoValueTF{#1}{}{^{#1}}\argopen(#3\argclose)}}
%     }
%     {\mathinner{\diffd\IfNoValueTF{#1}{}{^{#1}}#2} \IfNoValueTF{#3}{}{(#3)}}
% }
% \DeclareDocumentCommand\dd{}{\differential} % Shorthand for \differential

% \DeclareDocumentCommand\derivative{ s o m g d() }
% { % Total derivative
%     % s: star for \flatfrac flat derivative
%     % o: optional n for nth derivative
%     % m: mandatory (x in df/dx)
%     % g: optional (f in df/dx)
%     % d: long-form d/dx(...)
%     \IfBooleanTF{#1}
%     {\let\fractype\flatfrac}
%     {\let\fractype\frac}
%     \IfNoValueTF{#4}
%     {
%         \IfNoValueTF{#5}
%         {\fractype{\diffd \IfNoValueTF{#2}{}{^{#2}}}{\diffd #3\IfNoValueTF{#2}{}{^{#2}}}}
%         {\fractype{\diffd \IfNoValueTF{#2}{}{^{#2}}}{\diffd #3\IfNoValueTF{#2}{}{^{#2}}} \argopen(#5\argclose)}
%     }
%     {\fractype{\diffd \IfNoValueTF{#2}{}{^{#2}} #3}{\diffd #4\IfNoValueTF{#2}{}{^{#2}}}}
% }
% \DeclareDocumentCommand\dv{}{\derivative} % Shorthand for \derivative

% \DeclareDocumentCommand\partialderivative{ s o m g g d() }
% { % Partial derivative
%     % s: star for \flatfrac flat derivative
%     % o: optional n for nth derivative
%     % m: mandatory (x in df/dx)
%     % g: optional (f in df/dx)
%     % g: optional (y in d^2f/dxdy)
%     % d: long-form d/dx(...)
%     \IfBooleanTF{#1}
%     {\let\fractype\flatfrac}
%     {\let\fractype\frac}
%     \IfNoValueTF{#4}
%     {
%         \IfNoValueTF{#6}
%         {\fractype{\partial \IfNoValueTF{#2}{}{^{#2}}}{\partial #3\IfNoValueTF{#2}{}{^{#2}}}}
%         {\fractype{\partial \IfNoValueTF{#2}{}{^{#2}}}{\partial #3\IfNoValueTF{#2}{}{^{#2}}} \argopen(#6\argclose)}
%     }
%     {
%         \IfNoValueTF{#5}
%         {\fractype{\partial \IfNoValueTF{#2}{}{^{#2}} #3}{\partial #4\IfNoValueTF{#2}{}{^{#2}}}}
%         {\fractype{\partial^2 #3}{\partial #4 \partial #5}}
%     }
% }
% \DeclareDocumentCommand\pderivative{}{\partialderivative} % Shorthand for \partialderivative
% \DeclareDocumentCommand\pdv{}{\partialderivative} % Shorthand for \partialderivative

\DeclareDocumentCommand\variation{ o g d() }{ % Functional variation
    % o: optional n for nth differential
    % g: optional argument for readability and to control spacing
    % d: long-form as in d(F(g))
    \IfNoValueTF{#2}{
        \IfNoValueTF{#3}
        {\delta \IfNoValueTF{#1}{}{^{#1}}}
        {\mathinner{\delta \IfNoValueTF{#1}{}{^{#1}}\argopen(#3\argclose)}}
    }
    {\mathinner{\delta \IfNoValueTF{#1}{}{^{#1}}#2} \IfNoValueTF{#3}{}{(#3)}}
}
\DeclareDocumentCommand\var{}{\variation} % Shorthand for \variation

\DeclareDocumentCommand\functionalderivative{ s o m g d() }
{ % Functional derivative
    % s: star for \flatfrac flat derivative
    % o: optional n for nth derivative
    % m: mandatory (g in dF/dg)
    % g: optional (F in dF/dg)
    % d: long-form d/dx(...)
    \IfBooleanTF{#1}
    {\let\fractype\flatfrac}
    {\let\fractype\frac}
    \IfNoValueTF{#4}
    {
        \IfNoValueTF{#5}
        {\fractype{\variation \IfNoValueTF{#2}{}{^{#2}}}{\variation #3\IfNoValueTF{#2}{}{^{#2}}}}
        {\fractype{\variation \IfNoValueTF{#2}{}{^{#2}}}{\variation #3\IfNoValueTF{#2}{}{^{#2}}} \argopen(#5\argclose)}
    }
    {\fractype{\variation \IfNoValueTF{#2}{}{^{#2}} #3}{\variation #4\IfNoValueTF{#2}{}{^{#2}}}}
}
\DeclareDocumentCommand\fderivative{}{\functionalderivative} % Shorthand for \functionalderivative
\DeclareDocumentCommand\fdv{}{\functionalderivative} % Shorthand for \functionalderivative


\begin{document}

\section*{Variational Principles \hfill IB Easter}

\section{Multi-variable function}
\begin{itemize}
    \item Stationary point $\implies \grad f = \vec{0}$
    \item Max, min, saddle: check eigenvalues of Hessian matrix
\end{itemize}

\section{Convexity, Extremising functions $\R^n \to \R$}
\begin{itemize}
    \item Convex set; convex, strictly convex, concave function etc.
    \item (Chord connecting two points of convex set lies entirely in the set, chord of graph of convex function lies above graph of function)
    \item Stationary point of convex function is global minimum (may not be unique point)
          \subsection*{Alternative characterisation of convexity}
    \item Once differentiable function: \\
          $f$ is convex
          \[\iff f(y) \geq f(x) + (y-x) \cdot \grad f (x)\] (graph lies above tangent plane)
          \[\iff (y-x) \cdot (\grad f(y) - \grad f (x)) \geq 0\] (in case $n = 1$, condition means $f'$ monotonic increasing)
    \item Twice differentiable: convex iff Hessian matrix non-negative definite \\
          (no equivalence for strictly convex, $H$ positive definite $\implies$ strict convexity but no reverse implication)
    \item Extremising function under constraints: \\
          Extremise $f(\vec{x})$ while fixing $g(\vec{x})=L$:\\
          By Lagrange multiplier, extremise $h(\vec{x}, \lambda) = f(\vec{x}) -\lambda (L-g(\vec{x}))$
\end{itemize}

\section{Legendre Transform}
Given function $f:\R^n \to \R$, the Legendre Transform of $f$ is \[f^*(\vec{p}) = \sup_{\vec{x}} \, (\vec{p}\cdot \vec{x} - f(\vec{x})), \] over the domain $D\subseteq \R^n$ on which the $\sup$ is finite.
\begin{itemize}
    \item $f^*$ is always convex
    \item If $f$ is convex and differentiable, then $f(\vec{x}) - \vec{p}\cdot \vec{x} $ is convex. \\
          So any stationary point of $\vec{p}\cdot \vec{x} - f(\vec{x})$ is global max (unique if $f$ is strictly convex), which occurs at $\vec{x}(\vec{p})$ that satisfies \[\grad f (\vec{x}) = \vec{p}\]
          In this case $f^*(p) = \vec{p}\cdot \vec{x}(\vec{p}) - f(\vec{x}(\vec{p}))$
    \item $f^{**} = f$ if $f$ is convex
\end{itemize}

\subsection*{Application to Thermodynamics}
\[\dd U = T \dd S - P \dd V\]
Legendre transform changes independent variables: \\
$U(S,V)$ becomes $F(T,V)$, and $H(S,P)$
\begin{itemize}
    \item Helmholtz Free Energy:
          \[F(T,V) = \min_S \, (U(S,V)-TS) = - U^*(S,V)\]
          Legendre transform of $U$ with respect to $S$, fixing $V$\\
          The independent variable $T$ is temperature since the min is attained when $T = \left. \pdv{U}{S} \right|_V$\\
          Get \[\dd F = -S \dd T - P \dd V\]
    \item Enthalpy:
          \[H(S,P) = \min_V \, (U(S,V) + PV) = - U^*(-P,S),\] Legendre transform of $U$ with respect to $V$, with $S$ fixed\\
          The independent variable $P$ is indeed pressure since min attained when $P = -  \left. \pdv{U}{V} \right|_S$ \\
          \\
          Get \[\dd H = T \dd S + V \dd P\]
    \item Gibbs Free Energy: \[G(T,P) = H(S,P) - TS, \] where $S$ satisfies $T = \left. \pdv{H}{S} \right|_P$ , Legendre transform w.r.t. $S$

\end{itemize}

\section{Extremising functional}
\begin{itemize}
    \item Finding "stationary" points of functional (max, min, saddle): \[L[y] = \int_\alpha^\beta f(x,y,y') dx\] subject to boundary condition that cause boundary term to vanish
    \item Taylor expand to first order, integrate by parts:
          \[L[y+ \varepsilon \eta] - L[y] = \varepsilon \int_\alpha^\beta \eta \left(\pdv{f}{y} - \odv{}{x} \pdv{f}{y'} \right) \, \dd x + O(\varepsilon^2)\]
    \item Functional derivative/Euler-Lagrange equation: \[\fdv{L[y]}{y} = \pdv{f}{y} - \odv{}{x} \pdv{f}{y'} =0\]
    \item First integrals if $f$ does not depend explicitly on $y$, or $x$
\end{itemize}


\subsection*{Multiple dependent variables}
Extremise \[ L[\vec{y}] = \int_\alpha^\beta f(x,y_1,\dots,y_n,y_1',\dots,y_n') \, \dd x,\] with appropriate boundary conditions

\begin{itemize}
    \item Generalisation: Use Euler-Lagrange on each dependent variable
    \item Modification for first integrals:
          \begin{itemize}
              \item $f$ no explicit dependence on some $y_i$: get first integral $\pdv{f}{y_i'} =$ constant
              \item $f$ no explicit dependence on $x$: get first integral $f - \sum y_i' \pdv{f}{y_i'} =$ constant
          \end{itemize}
\end{itemize}

\subsection*{Multiple independent variables}
Find function $\vec{\Phi}: D \subseteq \R^m \to \R^n$ fixed at boundary that extremises
\[F[\vec{\Phi}] = \int \dots \int_D f(x_1, \dots x_m, \vec{\Phi}, \grad\vec{\Phi} ) \, \dd x_1 \dots \dd x_m, \] where $\grad \vec{\Phi} = \left( \pdv{\vec{\Phi}_i}{x_j}\right)_{ij}$

\begin{itemize}
    \item Taylor expand $f$, write integrand as a divergence + $\eta \cdot (\fdv{F}{\Phi})$
    \item Integral of divergence vanishes
    \item Euler-Lagrange give PDEs
\end{itemize}

\subsection*{Euler-Lagrange for higher derivatives}
For example: Extremise \[F[x] = \int_\alpha^\beta f(t,\dot{x}, \ddot{x}) \, \dd t, \] with $x, \dot{x}$ fixed at boundary \\

Taylor Expand integrand, integration by parts: get \[ \delta F[x] = \int_\alpha^\beta \delta x \underbrace{\left( \pdv{f}{x} - \odv{}{x} \pdv{f}{\dot{x}} + \odv[2]{}{x}  \pdv{f}{\ddot{x}}  \right)}_{\begin{displaystyle}
            \fdv{F}{x}
        \end{displaystyle}} \, \dd t\]

\subsection*{Extremising with constraints}
Extremise \[F[y] = \int_\alpha^\beta f(x,y,y') \, \dd x\] under functional constraint \[G[y] = \int_\alpha^\beta g(x,y,y') \, \dd x = K \]

\begin{itemize}
    \item Use Lagrange multiplier: Extremise \[L[y] = \int_\alpha^\beta f(x,y,y') - \lambda g(x,y,y') \, \dd x = F[y]- \lambda G[y] \]
\end{itemize}

Extremise \[F[\vec{x}] = \int_\alpha^\beta f(t,\vec{x},\vec{x}') \, \dd t\] under function constraint $g(\vec{x}) = 0$ for all $t$
\begin{itemize}
    \item Need Lagrange multiplier $\lambda(t)$ that changes with $t$: \[\Phi[\vec{x},\lambda] = \int_\alpha^\beta \left( f(t,\vec{x},\vec{x}') - \lambda(t) g(\vec{x}) \right) \, \dd t\]
\end{itemize}


\section{Example, variational principles}
\begin{itemize}
    \item Geodesic (minimise distance): Euclidean plane (straight line), sphere (great circle), general surface
    \item Minimal surface (minimise surface area): e.g. catenoid of soap film
    \item Brachistochrone: shortest travel time (cycloid)
    \item Fermat's Principle: least time $\implies$ Snell's Law
    \item Minimise potential energy (e.g. catenary)
    \item Dido problem (isoperimetric problem, maximise area)
    \item Sturm-Liouville problem
    \item Lagrangian (Old) principle of least action: Minimises $m \int v \, \dd l$
    \item Hamilton's principle: action $= \int L \, \dd t$
\end{itemize}


\subsubsection*{Sturm-Liouville Problem}
Given $\rho(x), w(x) \geq 0$, extremise \[F[y] = \int_\alpha^\beta \rho(x) (y')^2 + \sigma(x) y^2 \dd x,\] subject to \[G[y] = \int_\alpha^\beta w(x)y^2 \dd{x} = 1\] and fixed boundary of $y$

\begin{itemize}
    \item Use Lagrange multiplier, Euler Lagrange gives \[ - \odv{}{x} (\rho(x) y') + \sigma(x) y = \lambda w(x) y\]
    \item Extremal are eigenfunctions of Sturm-Liouville operator \[\mathcal{L} = - \odv{}{x} \left( \rho(x) \odv{}{x} \right) + \sigma(x),\] extremal $y$ satisfy \[\mathcal{L}y = 2wy\]
    \item If also $\sigma(x) > 0$, then $F[y] \geq 0$, so minimum of $F[y]/G[y]$ is smallest eigenvalue of $\mathcal{L}$ (Check notes again for boundary term)
\end{itemize}

\section{Formulation of Mechanics}
\subsection*{Lagrangian Mechanics}
\begin{itemize}
    \item Lagrangian $L = T-V$ in generalised coordinates
    \item Hamilton's Principle: Trajectory is stationary point of action (not actually minimiser) with fixed endpoints \[A[\vec{x}] = \int_{t_1}^{t_2} L \, \dd t \]
    \item Lagrange's Equation \[\pdv{L}{x_i} = \odv*{\left(\pdv{L}{\dot{x}_i}\right)}{t}\]
    \item Implies Newton's 2nd Law
    \item If potential energy $V$ is independent of time, get conservation of energy
\end{itemize}

\subsection*{Noether's Theorem}
(More details in Townsend's notes p.35)

If $\vec{y} = (y_1,\dots y_n)$ extremises functional $F[\vec{y}] = \int_\alpha^\beta f(x, \vec{y} , \vec{\dot{y}}) \, \dd x,$
and there is 1-parameter continuous symmetry
$y_i(x) \mapsto Y_i(s,x)$ such that \[\odv*{f(x, \vec{Y}(s,x) , \vec{\dot{Y}}(s,x))}{s} = 0, \text{ and } Y_i(0,x) = y_i(x)\]
Get first integral: \[ \left. \sum_{i=1}^n \pdv{f}{y_i'} \pdv{Y_i}{s} \right|_{s=0} = \text{ constant}\]

(Invariance under e.g. translation, rotation give conservation laws)

\subsection*{Hamiltonian Mechanics}
Hamiltonian $=$ Legendre transform of Lagrangian $L(\vec{q}, \vec{\dot{q}},t)$ with respect to velocity $\vec{\dot{q}}$ (in generalised coordinates):
\begin{align*}
    H(\vec{q}, \vec{p}, t) & = \sup_{\vec{\dot{q}}} [\vec{p} \cdot \vec{\dot{q}} - L(\vec{q}, \vec{\dot{q}},t) ]                                                     \\
                           & = \vec{p} \cdot \vec{\dot{q}}(\vec{p}) - L(\vec{q}, \vec{\dot{q}}(\vec{p}),t), \text{ where } \vec{\dot{q}}(\vec{p}) \text{ satisfies }
\end{align*}
\[ \vec{p} = \pdv{L}{\vec{\dot{q}}}, \text{ i.e. } p_i = \pdv{L}{\dot{q}_i} \]

\begin{itemize}
    \item New independent variable $\vec{p}$ is generalised momentum
    \item In one particle scenario, Hamiltonian is total energy $(T+V)$
    \item Hamilton's equations:
          \begin{align*}
              \dd H & = \pdv{H}{q_i} \dd q_i + \pdv{H}{p_i} \dd p_i +  \pdv{H}{t} \dd t                                                                       \\
                    & = \pdv{L}{q_i}  \dd q_i + \left[\dot{q}_i + \left(p_j - \pdv{L}{\dot{q}_j}\right) \pdv{\dot{q}_j}{p_i}\right] \dd p_i - \pdv{L}{t}\dd t \\
                    & = \left(\dot{p}_i \right) \dd q_i + \left( \dot{q}_i \right) \dd p_i - \pdv{L}{t}\dd t.
          \end{align*}
          where $p_j = \pdv{L}{\dot{q}_j}$ by above, and
          $\pdv{L}{q_i} = \odv*{\pdv{L}{\dot{q}_i}}{t} = \dot{p}_i$ by Lagrange's equation

          Matching terms give
          \[\pdv{H}{\vec{q}} = - \vec{\dot{p}},  \qquad \pdv{H}{\vec{p}} = \vec{\dot{q}}, \qquad \pdv{H}{t} = - \pdv{L}{t}\]
          (Be careful of which variables held fixed in partial differentiation)
    \item Hamilton's equations can also be obtained from E-L equations of a variational principle, extremising \[S[\vec{q}, \vec{p}] = \int \left\{\vec{p} \cdot \vec{q} - H(\vec{q}, \vec{p}, t)\right\} \, \dd t\]
\end{itemize}

\section{Second variation}
To extremise $F[y] = \int_\alpha^\beta f(x,y,y') \, \dd x$, with fixed ends.
\begin{itemize}
    \item Taylor expand $F[y+ \varepsilon \eta] - F[y]$ to second order, about stationary point $y$:
          \[F[y+ \varepsilon \eta] - F[y] = \varepsilon^2 \underbrace{ \frac{1}{2} \int_\alpha^\beta \left( \eta^2 \pdv[2]{f}{y} + (\eta')^2 \pdv[2]{f}{y'} + 2\eta \eta' \pdv{f}{ y, y'} \right)\, \dd x}_{\begin{displaystyle}\delta^2 F[y, \eta] \end{displaystyle} } + O(\varepsilon^3)\]
    \item Using by parts on mixed term:
          \[\delta^2 F[y , \eta ] =  \frac{1}{2} \int_\alpha^\beta \left\{ \underbrace{\left(  \pdv[2]{f}{y} - \odv*{\left( \pdv{f}{ y, y'}\right)}{x}  \right)}_{Q} \eta^2 + \underbrace{\pdv[2]{f}{y'}}_{P} (\eta')^2 \right\} \, \dd x\]
    \item If $y$ satisfies E-L, and $\delta^2 F[y , \eta ] \geq 0$ for any $\eta$ that vanishes at endpoints, then $y$ is a local minimiser of $F[y]$
\end{itemize}
\subsection*{Conditions for local minimum}
\begin{itemize}
    \item Legendre condition: $y$ is a local minimum $\implies P = \pdv[2]{f}{y'} \geq 0$; \\
          otherwise, can find $\eta$ with small magnitude but large $(\eta')^2$ to give contradiction

          (necessary, not sufficient)

    \item $P>0$ and $Q\geq 0$ sufficient for local minimum
    \item Rewrite \[Q \eta^2 + P (\eta')^2 = Q \eta^2 + (P\eta \eta')' - \eta (P\eta')' \]
          In terms of  Sturm-Liouville operator \[ \delta^2 F[y , \eta ] = \frac{1}{2} \int_\alpha^\beta \eta \underbrace{(-(P\eta')' + Q \eta)}_{\mathcal{L}(\eta)} \, \dd x\]
          If $\mathcal{L}$ has a negative eigenvalue, with eigenfunction $\eta$ that vanishes at endpoints, i.e. \[\mathcal{L}(\eta) = - \omega^2 \eta, \quad \eta(\alpha) = \eta(\beta) = 0,\] then $y$ is not a local minimiser
\end{itemize}

\subsubsection*{Jacobi condition}
It is known that $P>0$ is not sufficient for local minimum, but when does it fail?
\begin{itemize}
    \item For any differentiable $\phi$, have \[ 0 = \int_\alpha^\beta (\phi \eta^2)' \, \dd x = \int_\alpha^\beta [2 \phi \eta \eta' + \phi' \eta^2] \, \dd x, \]
    \item Add it to $\delta^2 F[y , \eta]$, complete the square (assuming $P>0$):
          \[ \delta^2 F[y , \eta ] = \frac{1}{2} \int_\alpha^\beta \left[ P\left( \eta' + \frac{\phi\eta}{P} \right)^2 + \left(Q+\phi' - \frac{\phi^2}{P} \right) \eta^2 \right] \dd x\]

    \item If can choose $\phi$ to make the second term vanish, then $\delta^2 F[y , \eta] >0 $. This sufficient condition is given by a Ricatti equation
          \[\phi^2 = P(Q+\phi')\]
    \item Transform the equation, by letting $\phi = -P \frac{u'}{u}$, get Jacobi accessory equation \[ -(Pu')' + Qu = 0 \]
          Need to find solution $u$ to this Strum-Liouville equation s.t. $u(x) \neq 0$ for $\alpha < x< \beta$, which might not exist for  interval too large

\end{itemize}


\end{document}