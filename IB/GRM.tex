\documentclass[12pt, a4paper]{article}
\usepackage[T1]{fontenc}
\usepackage{amsmath, amssymb, amsthm}
\usepackage{babel}
\usepackage{siunitx}
\usepackage{tikz}
\usepackage{centernot}
\usepackage{tcolorbox}
\usepackage{cancel}
\usepackage{enumitem}
\usepackage{xparse}

\usetikzlibrary{arrows}

\setlength{\parindent}{0pt}


% Matrix groups
\newcommand{\GL}{\mathrm{GL}}
\newcommand{\M}{\mathrm{M}}
\newcommand{\Or}{\mathrm{O}}
\newcommand{\PGL}{\mathrm{PGL}}
\newcommand{\PSL}{\mathrm{PSL}}
\newcommand{\PSO}{\mathrm{PSO}}
\newcommand{\PSU}{\mathrm{PSU}}
\newcommand{\SL}{\mathrm{SL}}
\newcommand{\SO}{\mathrm{SO}}
\newcommand{\Spin}{\mathrm{Spin}}
\newcommand{\Sp}{\mathrm{Sp}}
\newcommand{\SU}{\mathrm{SU}}
\newcommand{\U}{\mathrm{U}}
\newcommand{\Mat}{\mathrm{Mat}}


% Special sets
\newcommand{\C}{\mathbb{C}}
\newcommand{\CP}{\mathbb{CP}}
\newcommand{\F}{\mathbb{F}}
\newcommand{\GG}{\mathbb{G}}
\newcommand{\N}{\mathbb{N}}
% \newcommand{\P}{\mathbb{P}}
\newcommand{\Q}{\mathbb{Q}}
\newcommand{\R}{\mathbb{R}}
\newcommand{\RP}{\mathbb{RP}}
\newcommand{\T}{\mathbb{T}}
\newcommand{\Z}{\mathbb{Z}}
\renewcommand{\H}{\mathbb{H}}

% Brackets
\renewcommand{\vec}[1]{\boldsymbol{\mathbf{#1}}}
\newcommand{\cis}[1]{ \cos\left( #1 \right) + i \sin \left( #1 \right)}

% Algebra
\DeclareMathOperator{\adj}{adj}
\DeclareMathOperator{\Ann}{Ann}
\DeclareMathOperator{\Aut}{Aut}
\DeclareMathOperator{\Char}{char}
\DeclareMathOperator{\disc}{disc}
\DeclareMathOperator{\dom}{dom}
\DeclareMathOperator{\fix}{fix}
\DeclareMathOperator{\Hom}{Hom}
\DeclareMathOperator{\id}{id}
\DeclareMathOperator{\image}{image}
\DeclareMathOperator{\im}{im}
\DeclareMathOperator{\tr}{tr}
\newcommand{\Bilin}{\mathrm{Bilin}}
\newcommand{\Frob}{\mathrm{Frob}}



\let\Im\relax
\let\Re\relax


\DeclareMathOperator{\hcf}{hcf}
\DeclareMathOperator{\Isom}{Isom}
\DeclareMathOperator{\lcm}{lcm}
\DeclareMathOperator{\sgn}{sgn}
\DeclareMathOperator{\supp}{supp}
\DeclareMathOperator{\Sym}{Sym}
\DeclareMathOperator{\Syl}{Syl}
\DeclareMathOperator{\Im}{Im}
\DeclareMathOperator{\Re}{Re}
\DeclareMathOperator{\Ker}{Ker}


% Theorems
\theoremstyle{definition}
\newtheorem*{aim}{Aim}
\newtheorem*{axiom}{Axiom}
\newtheorem*{claim}{Claim}
\newtheorem*{cor}{Corollary}
\newtheorem*{conjecture}{Conjecture}
\newtheorem*{defi}{Definition}
\newtheorem*{eg}{Example}
\newtheorem*{ex}{Exercise}
\newtheorem*{fact}{Fact}
\newtheorem*{law}{Law}
\newtheorem*{lemma}{Lemma}
\newtheorem*{notation}{Notation}
\newtheorem*{prop}{Proposition}
\newtheorem*{question}{Question}
\newtheorem*{rrule}{Rule}
\newtheorem*{thm}{Theorem}
\newtheorem*{assumption}{Assumption}

\newtheorem*{remark}{Remark}
\newtheorem*{warning}{Warning}
\newtheorem*{exercise}{Exercise}

\newtheorem{nthm}{Theorem}[section]
\newtheorem{nlemma}[nthm]{Lemma}
\newtheorem{nprop}[nthm]{Proposition}
\newtheorem{ncor}[nthm]{Corollary}


\newcommand{\abs}[1]{\left| #1 \right|} % for absolute value
\newcommand{\grad}[1]{\mathbf{\nabla} #1} % for gradient
\let\divsymb=\div % rename builtin command \div to \divsymb
\renewcommand{\div}[1]{\mathbf{\nabla} \cdot #1} % for divergence
\newcommand{\curl}[1]{\mathbf{\nabla} \times #1} % for curl




% Derivatives

\newcommand{\dd}[1][]{\mathrm{d} #1}
\newcommand{\odiff}[1]{\frac{\dd}{\dd{#1}}}
\newcommand{\odv}[2]{\frac{\dd{#1}}{\dd{#2}}}
\newcommand{\pdiff}[1]{\frac{\partial}{\partial{#1}}}
\newcommand{\pdv}[2]{\frac{\partial{#1}}{\partial{#2}}}


% \def\diffd{\mathrm{d}}
% \DeclareDocumentCommand\differential{ o g d() }{ % Differential 'd'
%     % o: optional n for nth differential
%     % g: optional argument for readability and to control spacing
%     % d: long-form as in d(cos x)
%     \IfNoValueTF{#2}{
%         \IfNoValueTF{#3}
%         {\diffd\IfNoValueTF{#1}{}{^{#1}}}
%         {\mathinner{\diffd\IfNoValueTF{#1}{}{^{#1}}\argopen(#3\argclose)}}
%     }
%     {\mathinner{\diffd\IfNoValueTF{#1}{}{^{#1}}#2} \IfNoValueTF{#3}{}{(#3)}}
% }
% \DeclareDocumentCommand\dd{}{\differential} % Shorthand for \differential

% \DeclareDocumentCommand\derivative{ s o m g d() }
% { % Total derivative
%     % s: star for \flatfrac flat derivative
%     % o: optional n for nth derivative
%     % m: mandatory (x in df/dx)
%     % g: optional (f in df/dx)
%     % d: long-form d/dx(...)
%     \IfBooleanTF{#1}
%     {\let\fractype\flatfrac}
%     {\let\fractype\frac}
%     \IfNoValueTF{#4}
%     {
%         \IfNoValueTF{#5}
%         {\fractype{\diffd \IfNoValueTF{#2}{}{^{#2}}}{\diffd #3\IfNoValueTF{#2}{}{^{#2}}}}
%         {\fractype{\diffd \IfNoValueTF{#2}{}{^{#2}}}{\diffd #3\IfNoValueTF{#2}{}{^{#2}}} \argopen(#5\argclose)}
%     }
%     {\fractype{\diffd \IfNoValueTF{#2}{}{^{#2}} #3}{\diffd #4\IfNoValueTF{#2}{}{^{#2}}}}
% }
% \DeclareDocumentCommand\dv{}{\derivative} % Shorthand for \derivative

% \DeclareDocumentCommand\partialderivative{ s o m g g d() }
% { % Partial derivative
%     % s: star for \flatfrac flat derivative
%     % o: optional n for nth derivative
%     % m: mandatory (x in df/dx)
%     % g: optional (f in df/dx)
%     % g: optional (y in d^2f/dxdy)
%     % d: long-form d/dx(...)
%     \IfBooleanTF{#1}
%     {\let\fractype\flatfrac}
%     {\let\fractype\frac}
%     \IfNoValueTF{#4}
%     {
%         \IfNoValueTF{#6}
%         {\fractype{\partial \IfNoValueTF{#2}{}{^{#2}}}{\partial #3\IfNoValueTF{#2}{}{^{#2}}}}
%         {\fractype{\partial \IfNoValueTF{#2}{}{^{#2}}}{\partial #3\IfNoValueTF{#2}{}{^{#2}}} \argopen(#6\argclose)}
%     }
%     {
%         \IfNoValueTF{#5}
%         {\fractype{\partial \IfNoValueTF{#2}{}{^{#2}} #3}{\partial #4\IfNoValueTF{#2}{}{^{#2}}}}
%         {\fractype{\partial^2 #3}{\partial #4 \partial #5}}
%     }
% }
% \DeclareDocumentCommand\pderivative{}{\partialderivative} % Shorthand for \partialderivative
% \DeclareDocumentCommand\pdv{}{\partialderivative} % Shorthand for \partialderivative

\DeclareDocumentCommand\variation{ o g d() }{ % Functional variation
    % o: optional n for nth differential
    % g: optional argument for readability and to control spacing
    % d: long-form as in d(F(g))
    \IfNoValueTF{#2}{
        \IfNoValueTF{#3}
        {\delta \IfNoValueTF{#1}{}{^{#1}}}
        {\mathinner{\delta \IfNoValueTF{#1}{}{^{#1}}\argopen(#3\argclose)}}
    }
    {\mathinner{\delta \IfNoValueTF{#1}{}{^{#1}}#2} \IfNoValueTF{#3}{}{(#3)}}
}
\DeclareDocumentCommand\var{}{\variation} % Shorthand for \variation

\DeclareDocumentCommand\functionalderivative{ s o m g d() }
{ % Functional derivative
    % s: star for \flatfrac flat derivative
    % o: optional n for nth derivative
    % m: mandatory (g in dF/dg)
    % g: optional (F in dF/dg)
    % d: long-form d/dx(...)
    \IfBooleanTF{#1}
    {\let\fractype\flatfrac}
    {\let\fractype\frac}
    \IfNoValueTF{#4}
    {
        \IfNoValueTF{#5}
        {\fractype{\variation \IfNoValueTF{#2}{}{^{#2}}}{\variation #3\IfNoValueTF{#2}{}{^{#2}}}}
        {\fractype{\variation \IfNoValueTF{#2}{}{^{#2}}}{\variation #3\IfNoValueTF{#2}{}{^{#2}}} \argopen(#5\argclose)}
    }
    {\fractype{\variation \IfNoValueTF{#2}{}{^{#2}} #3}{\variation #4\IfNoValueTF{#2}{}{^{#2}}}}
}
\DeclareDocumentCommand\fderivative{}{\functionalderivative} % Shorthand for \functionalderivative
\DeclareDocumentCommand\fdv{}{\functionalderivative} % Shorthand for \functionalderivative


\begin{document}

Up to Ch 10 in notes

\section*{Groups, Rings and Modules \hfill IB Lent}

\section{Group Theory}

\begin{itemize}
      \item Automorphism: Isomorphism from a group to itself
      \item A group $G$ is a permutation group of degree $n$ if $G \leq \Sym(X)$ for some set $X$ with $\abs{X} = n$
\end{itemize}


\subsection*{Isomorphism Theorems}
\begin{itemize}
      \item First Isomorphism Theorem: \\ If $H$ and $G$ are groups and $\phi: H \to G$ a homomorphism, then \[H/ \Ker \phi \cong \Im \phi\]
      \item Second Isomorphism Theorem: \\ If $H \leq G, K \triangleleft G$, then $HK \leq G$, and $H \cap K \triangleleft H$, and  \[ \frac{H}{H \cap K} \cong \frac{HK}{K}\]
            (Use homomorphism $H \to HK/K$ by $h \mapsto hK$)
      \item Correspondence Theorem: \\
            If $K \triangleleft G$, then there exists bijection between
            \begin{align*}
                  \{\text{subgroups of } G \text{ containing } K \} & \leftrightarrow \{\text{subgroups of } G/K \} \\
                  \text{via } \qquad  H                             & \mapsto H/K                                   \\
                  \{g \in G : gK \in S\}                            & \mapsfrom S
            \end{align*}
            This restricts to normal subgroups

      \item Third Isomorphism Theorem: \\
            If $K \leq H \leq G$ and $K \triangleleft G, H \triangleleft G$, then \[\frac{G/K}{H/K} \cong G/H \]
            (Use homomorphism $G/K \to G/H$ by $gK \mapsto gL$)
\end{itemize}

\subsection*{Simple Groups}
A group $G$ is simple iff its only normal subgroups are $\{e\}$ and $G$
\begin{itemize}
      \item An abelian group is simple iff it is isomorphic to $C_p$, some prime $p$

            (Only if: any non-trivial $g$ generates the whole group, so cyclic; not prime order $\implies$ have proper subgroup)
      \item If $G$ is a finite group, then it has a composition series: \[1 \triangleleft G_0 \triangleleft G_1 \triangleleft \dots \triangleleft G_n = G, \] with each quotient group $G_n/G_{n-1}$ simple.

            (Use correspondence theorem and induction)
\end{itemize}

\subsection*{Group Action}
\begin{itemize}
      \item An action of $G$ on set $X$ gives a permutation representation $\phi: G \to \Sym(X)$
      \item Examples:
            \begin{itemize}
                  \item $G$ acts on itself/collection of cosets $G/H$ by multiplication
                  \item Acts on itself/any normal subgroup by conjugation
                  \item $G$ act on $\Sub(G)$, the set of its subgroups by conjugation
                        \[ g* H = gHg^{-1} \]
                        Stabiliser $= N_G(H) = \{g\in G : gHg^{-1} = H\}$, the normaliser of $H$ in $G$\\
                        $N_G(H)$ is the largest subgroup of $G$ containing $H$ as a normal subgroup
            \end{itemize}

      \item If $G$ is a non-abelian simple group, and $H \leq G$ of index $n >1$. Then $n \geq 5$ and $G$ is isomorphic to a subgroup of $A_n$

            ($G$ acts on $G/H$ by left mult, get injective hom $\implies G \leq S_n$; \\
            $G \cap A_n \triangleleft G$, but $G \cap A_n = \{e\} \implies \abs{G} = 2$, abelian)
\end{itemize}

\subsection*{Alternating Groups}
Conjugacy class of $g$ splits from $S_n$ to $A_n$ iff $\exists$ odd permutation that commutes with $g$
\begin{itemize}
      \item $A_n$ is simple for $n \geq 5$
            \begin{itemize}
                  \item $A_n$ is generated by 3-cycles \\
                        (3-cycles generate double transpositions, which generate $A_n$)
                  \item All 3-cycles are conjugate in $A_n$
                  \item Any non-trivial $N \triangleleft A_n$ contains a 3-cycle \\ (consider cases)
            \end{itemize}
\end{itemize}

\subsection*{$p$-groups}
For a prime $p$, a finite group $G$ is a $p$-group if $\abs{G} = p^n$, some $n \in \N$
\begin{itemize}
      \item $p$-groups have $Z \neq \{e\}$ (proof by counting)
      \item only simple $p$-group is $C_p$
      \item If $G$ is a $p$-group of order $p^n$, then $G$ has a subgroup of order $p^r$ for $r=0,1,\dots n$ (composition series)
      \item If $G/Z(G)$ is cyclic, then $G$ is abelian (all elements have the form $g^i z$)
\end{itemize}

\subsection*{Sylow Theorems}
Let $G$ be a finite group of order $p^am$, where $p$ is a prime and $p \nmid m$. Then
\begin{enumerate}
      \item $\Syl_p(G) = \{ P \leq G: \abs{P} = p^a\}$ is non-empty, i.e. there exists a Sylow $p$-subgroup
      \item All elements of $\Syl_p(G)$ are conjugate
      \item The number of Sylow $p$-subgroups $n_p = \abs{\Syl_p(G)}$ satisfies \[n_p \equiv 1  \quad (\text{mod } p) \text{, and } n_p \mid \abs{G} \implies n_p \mid m\]
\end{enumerate}


\subsection*{Matrix Groups}
Over a field $F$, e.g. $\C, \Z/p\Z$, can have $GL_n(F)$, $SL_n(F)$, $PSL_n(F) = SL_n(F)/Z \cap SL_n(F)$, where $Z =$ center of $GL_n(F) = $ scalar matrices

\subsection*{Abelian Groups}
\begin{itemize}
      \item Decomposition: Every finite abelian group is isomorphic to product of cyclic groups (proof later in course)
      \item If $m$ and $n$ are coprime, then $C_n \times C_m \cong C_{mn}$
      \item If $G$ is a finite abelian group, then $G \cong C_{p_1^{\alpha_1}} \times C_{p_2^{\alpha_2}} \times \dots C_{p_k^{\alpha_k}}$, where $p_i$ are prime, not necessarily distinct
            OR $G \cong C_{d_1} \times C_{d_2} \times \dots C_{d_t}$, with $d_1 \mid d_2 \mid \dots \mid d_t$
\end{itemize}


\section{Ring Theory}
\begin{itemize}
      \item Definition:\\
            A ring is a triple $(R,+,\cdot)$, two binary operations (need to check closure), with axioms:
            \begin{itemize}
                  \item $(R,+)$ is an abelian group with identity $0$
                  \item Multiplication is associative, and has identity $1$
                  \item Distributivity: $(x+y)\cdot z = x\cdot z + y \cdot z, \quad x \cdot(y+z) = x \cdot y + x \cdot z$
            \end{itemize}
      \item $R$ is commutative if multiplication is commutative (addition automatically is)\\ (All rings in GRM are commutative)
      \item Subring: $S \subseteq R$ is a subring (written $S \leq R$) if $(S, +, \cdot)$ is a ring, (thus must have same identity elements as $R$)
      \item An element $r\in R$ is a unit if it has a multiplicative inverse, units form a group under multiplication
      \item A field is a ring with $0\neq 1$ and all non-zero elements are units
\end{itemize}

\subsection*{New rings from old}
\begin{itemize}
      \item Take product of rings, elementwise add/mult
      \item If $R$ is a ring, $X$ is a set, take the set of all functions $X \to R$ with pointwise operations:
            \[(f+g)(x) = f(x) + g(x) ,\quad (f \cdot g)(x) = f(x) \cdot g(x) \]
      \item $R[X] = \{(a_0,a_1,\dots ) \mid a_i \in R, \text{ finitely many non-zero}\}$ \\
            (Ring of polynomials with coeffs in $R$)\\

            Operations defined as polynomials: $(a_0,a_1,\dots, a_m ,0 ,\dots )=a_0 + a_1X + \dots + a_m X^m$

            \begin{itemize}
                  \item Polys are different from functions (esp. in rings like $\Z/p\Z$)
                  \item Degree of polynomial $(a_0,a_1,\dots )$: largest $m$ s.t. $a_m \neq 0$
                  \item Monic polynomial: degree $m$ and $a_m=1$
                  \item Division algorithm, induction
            \end{itemize}
\end{itemize}

Examples:
\begin{itemize}
      \item $\Z \leq \Q \leq \R \leq \C$
      \item Gaussian Integers = $\Z[i] = \{ a + bi \mid a,b, \in \Z\} $
      \item $R[X_1,\dots X_n] = \{\text{polys in } X_1,...,X_n \text{ with coefficients in } R\}$
      \item Power series $R[[X]]$ (convergence not an issue)
      \item Laurent polynomials $R[X,X^{-1}]$
      \item Zero ring: $\{0\}$, the only ring where $0=1$
\end{itemize}


\subsection*{Ideals, Quotients}

\begin{itemize}
      \item Ring homomorphism: $\phi: R \to S$ is a ring homomorphism if $\forall x,y \in R$
            \begin{itemize}
                  \item $\phi(x + y) = \phi(x) + \phi(y)$
                  \item $\phi(xy) = \phi(x)\phi(y)$
                  \item $\phi(1_R) = 1_S$
            \end{itemize} (preserves structure of both $+$ and $\cdot$, need to specify image of $1$ since elements need not have multiplicative inverse)
      \item Isomorphism = bijective homomorphism
      \item $\ker \phi = \{ r\in R \mid \phi(r) = 0\}$
      \item $I \subseteq R$ is an ideal ($I \triangleleft R$) if
            \begin{itemize}
                  \item (Additive closure) $I$ is a (normal, as addition must be commutative) subgroup of $(R,+)$
                  \item (Strong closure) If $r \in R$ and $x \in I$, then $rx \in I$
            \end{itemize}
      \item An ideal $I$ is a proper ideal if $I \neq R$  ($I$ must not contain any unit) \\
            (In particular does not contain $1$, so proper ideals are not subrings)
      \item If $\phi: R \to S$ is a homomorphism, then $\ker \phi \triangleleft R$
      \item Ideal generated by $a_1,a_2,\dots a_n$: \[(a_1,a_2,\dots, a_n) = \{a_1r_1+a_2r_2 + \dots + a_nr_n \mid r_i\in R\}\]
            In particular, $(a) = aR = \{ar \mid r \in R \}$
      \item An ideal $I$ is principal if $I = (a)$ for some $a$
      \item Quotient ring:\\
            If $I \triangleleft R$, then the set $R/I$ of additive cosets of $I$ form the quotient ring with operations:
            \begin{align*}
                  (r_1+I)+(r_2+I)       & = r_1+r_2+I \\
                  (r_1+I) \cdot (r_2+I) & = r_1r_2+I
            \end{align*}
      \item $I$ is an ideal $\iff I$ is a kernel of some homomorphism/quotient map
      \item There exist unique ring homomorphism  $\phi : \Z \to R$: \[1 \mapsto 1_R, \quad \pm n \mapsto \pm (\underbrace{1_R + \dots + 1_R}_{n})\]
            $\ker \phi = n\Z$ for some $n = \Char R \geq 0$, the characteristic of $R$ \\
            (If $\Char R > 0$, it is the order of $1_R$ in $(R,+)$; otherwise $1_R$ has infinite order)

\end{itemize}

\subsection*{Isomorphism Theorems}

\begin{itemize}
      \item First Isomorphism Theorem: \\
            If $\phi: R \to S$ is a ring homomorphism, then \[R/ \Ker \phi \cong \Im \phi \leq S\]

      \item Second Isomorphism Theorem: \\
            Let $R \leq S, J \triangleleft S$, then $ R\cap J \triangleleft R$, and $R+J \leq S$, and  \[ \frac{R}{R \cap J} \cong \frac{R+J}{J}\]

      \item Correspondence Theorem: \\
            If $I \triangleleft R$, then there exists bijection between
            \begin{align*}
                  \{\text{ideals in } R \text{ containing } I \} & \leftrightarrow \{\text{ideals in } R/I \} \\
                  \text{via } \qquad  J                          & \mapsto J/I                                \\
                  \{r \in R : r+I \in K\}                        & \mapsfrom K
            \end{align*}

      \item Third Isomorphism Theorem: \\
            If $I \triangleleft R, J \triangleleft R$ and $I \subseteq J$, then $J/I \triangleleft R/I$ and \[\frac{R/I}{J/I} \cong R/J \]
\end{itemize}

\subsection*{Integral domain, maximal, prime ideals}

\begin{itemize}
      \item $R$ (non-zero) is an integral domain if $\forall a,b \in R: ab=0 \implies a=0 \text{ or } b =0$ \\
            i.e. no zero divisors ($a \neq 0$ is a zerodivisor if $\exists \, b \neq 0$ s.t. $ab=0$)
      \item Finite integral domains are fields; all fields are integral domains
      \item $R$ integral domain $\implies R[X]$ integral domain (polys with coeff in $R$)
      \item At most deg(f) many roots
      \item Any finite subgroup of the multiplicative group of a field is cyclic
      \item If $R$ integral domain, there exists $F$ field of fractions s.t. \begin{enumerate}
                  \item $R \leq F$
                  \item Every element of $F$ may be written as $ab^{-1}$, for $a,b \in R$, $b \neq 0$ \\
                        ($b^{-1}$ is multiplicative inverse in $F$)
            \end{enumerate} (via equivalence classes)
\end{itemize}

Look at rings only through its ideals:

\begin{itemize}
      \item $R$ a field $\iff$ only ideals are $\{0\}$ and $R$
      \item Maximal Ideal: $I \triangleleft R$ is maximal if for all ideal $J$ with $I \leq J \leq R$, then $J = I$ or $J=R$
            (no proper ideal strictly bigger $R$)
      \item $I \triangleleft R$ maximal $\iff R/I$ a field
      \item Prime Ideal: $I \triangleleft R$ is prime if $I \neq R$ and \[\forall a,b, \in R: ab \in I \implies a\in I \text{ or } b \in I\]
      \item $I \triangleleft R$ prime $\iff R/I$ an integral domain
      \item Maximal ideal $\implies$ prime
      \item $R$ integral domain, then $\Char R = 0$ or prime number
\end{itemize}

\subsection*{Factorisation in rings}
\begin{itemize}
      \item Unit, divides, associates, irreducible (factorisation must contain a unit), prime ($p \mid ab \implies p \mid a$ or $ p \mid b$) elements (also non-zero and not unit)
      \item $(r)$ prime ideal $\iff$ $r$ prime or $r=0$
      \item Prime $\implies$ irreducible, converse false
      \item Principal Ideal Domain: every ideal is principal
      \item $(r)$ maximal $\implies$ $r \in R$ is irreducible, converse holds if $R$ is a PID
      \item Euclidean Domain (ED): there exists Euclidean function:
            \[\phi: R \backslash \{0\} \to \Z_{\geq 0}\] s.t.
            \begin{enumerate}[label = (\roman*)]
                  \item $a \mid b \implies \phi(a) \leq \phi(b)$
                  \item If $a,b \in R$ and $b \neq 0$, then $\exists \, q,r \in R$ with $a=qb+r$ and either $r=0$ or $\phi(r) < \phi(b)$
            \end{enumerate}
\end{itemize}


\end{document}