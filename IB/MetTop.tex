\documentclass[12pt, a4paper]{article}
\usepackage[T1]{fontenc}
\usepackage{amsmath, amssymb, amsthm}
\usepackage{babel}
\usepackage{siunitx}
\usepackage{tikz}
\usepackage{centernot}
\usepackage{tcolorbox}
\usepackage{cancel}
\usepackage{enumitem}
\usepackage{xparse}

\usetikzlibrary{arrows}

\setlength{\parindent}{0pt}


% Matrix groups
\newcommand{\GL}{\mathrm{GL}}
\newcommand{\M}{\mathrm{M}}
\newcommand{\Or}{\mathrm{O}}
\newcommand{\PGL}{\mathrm{PGL}}
\newcommand{\PSL}{\mathrm{PSL}}
\newcommand{\PSO}{\mathrm{PSO}}
\newcommand{\PSU}{\mathrm{PSU}}
\newcommand{\SL}{\mathrm{SL}}
\newcommand{\SO}{\mathrm{SO}}
\newcommand{\Spin}{\mathrm{Spin}}
\newcommand{\Sp}{\mathrm{Sp}}
\newcommand{\SU}{\mathrm{SU}}
\newcommand{\U}{\mathrm{U}}
\newcommand{\Mat}{\mathrm{Mat}}


% Special sets
\newcommand{\C}{\mathbb{C}}
\newcommand{\CP}{\mathbb{CP}}
\newcommand{\F}{\mathbb{F}}
\newcommand{\GG}{\mathbb{G}}
\newcommand{\N}{\mathbb{N}}
% \newcommand{\P}{\mathbb{P}}
\newcommand{\Q}{\mathbb{Q}}
\newcommand{\R}{\mathbb{R}}
\newcommand{\RP}{\mathbb{RP}}
\newcommand{\T}{\mathbb{T}}
\newcommand{\Z}{\mathbb{Z}}
\renewcommand{\H}{\mathbb{H}}

% Brackets
\renewcommand{\vec}[1]{\boldsymbol{\mathbf{#1}}}
\newcommand{\cis}[1]{ \cos\left( #1 \right) + i \sin \left( #1 \right)}

% Algebra
\DeclareMathOperator{\adj}{adj}
\DeclareMathOperator{\Ann}{Ann}
\DeclareMathOperator{\Aut}{Aut}
\DeclareMathOperator{\Char}{char}
\DeclareMathOperator{\disc}{disc}
\DeclareMathOperator{\dom}{dom}
\DeclareMathOperator{\fix}{fix}
\DeclareMathOperator{\Hom}{Hom}
\DeclareMathOperator{\id}{id}
\DeclareMathOperator{\image}{image}
\DeclareMathOperator{\im}{im}
\DeclareMathOperator{\tr}{tr}
\newcommand{\Bilin}{\mathrm{Bilin}}
\newcommand{\Frob}{\mathrm{Frob}}



\let\Im\relax
\let\Re\relax


\DeclareMathOperator{\hcf}{hcf}
\DeclareMathOperator{\Isom}{Isom}
\DeclareMathOperator{\lcm}{lcm}
\DeclareMathOperator{\sgn}{sgn}
\DeclareMathOperator{\supp}{supp}
\DeclareMathOperator{\Sym}{Sym}
\DeclareMathOperator{\Syl}{Syl}
\DeclareMathOperator{\Im}{Im}
\DeclareMathOperator{\Re}{Re}
\DeclareMathOperator{\Ker}{Ker}


% Theorems
\theoremstyle{definition}
\newtheorem*{aim}{Aim}
\newtheorem*{axiom}{Axiom}
\newtheorem*{claim}{Claim}
\newtheorem*{cor}{Corollary}
\newtheorem*{conjecture}{Conjecture}
\newtheorem*{defi}{Definition}
\newtheorem*{eg}{Example}
\newtheorem*{ex}{Exercise}
\newtheorem*{fact}{Fact}
\newtheorem*{law}{Law}
\newtheorem*{lemma}{Lemma}
\newtheorem*{notation}{Notation}
\newtheorem*{prop}{Proposition}
\newtheorem*{question}{Question}
\newtheorem*{rrule}{Rule}
\newtheorem*{thm}{Theorem}
\newtheorem*{assumption}{Assumption}

\newtheorem*{remark}{Remark}
\newtheorem*{warning}{Warning}
\newtheorem*{exercise}{Exercise}

\newtheorem{nthm}{Theorem}[section]
\newtheorem{nlemma}[nthm]{Lemma}
\newtheorem{nprop}[nthm]{Proposition}
\newtheorem{ncor}[nthm]{Corollary}


\newcommand{\abs}[1]{\left| #1 \right|} % for absolute value
\newcommand{\grad}[1]{\mathbf{\nabla} #1} % for gradient
\let\divsymb=\div % rename builtin command \div to \divsymb
\renewcommand{\div}[1]{\mathbf{\nabla} \cdot #1} % for divergence
\newcommand{\curl}[1]{\mathbf{\nabla} \times #1} % for curl




% Derivatives

\newcommand{\dd}[1][]{\mathrm{d} #1}
\newcommand{\odiff}[1]{\frac{\dd}{\dd{#1}}}
\newcommand{\odv}[2]{\frac{\dd{#1}}{\dd{#2}}}
\newcommand{\pdiff}[1]{\frac{\partial}{\partial{#1}}}
\newcommand{\pdv}[2]{\frac{\partial{#1}}{\partial{#2}}}


% \def\diffd{\mathrm{d}}
% \DeclareDocumentCommand\differential{ o g d() }{ % Differential 'd'
%     % o: optional n for nth differential
%     % g: optional argument for readability and to control spacing
%     % d: long-form as in d(cos x)
%     \IfNoValueTF{#2}{
%         \IfNoValueTF{#3}
%         {\diffd\IfNoValueTF{#1}{}{^{#1}}}
%         {\mathinner{\diffd\IfNoValueTF{#1}{}{^{#1}}\argopen(#3\argclose)}}
%     }
%     {\mathinner{\diffd\IfNoValueTF{#1}{}{^{#1}}#2} \IfNoValueTF{#3}{}{(#3)}}
% }
% \DeclareDocumentCommand\dd{}{\differential} % Shorthand for \differential

% \DeclareDocumentCommand\derivative{ s o m g d() }
% { % Total derivative
%     % s: star for \flatfrac flat derivative
%     % o: optional n for nth derivative
%     % m: mandatory (x in df/dx)
%     % g: optional (f in df/dx)
%     % d: long-form d/dx(...)
%     \IfBooleanTF{#1}
%     {\let\fractype\flatfrac}
%     {\let\fractype\frac}
%     \IfNoValueTF{#4}
%     {
%         \IfNoValueTF{#5}
%         {\fractype{\diffd \IfNoValueTF{#2}{}{^{#2}}}{\diffd #3\IfNoValueTF{#2}{}{^{#2}}}}
%         {\fractype{\diffd \IfNoValueTF{#2}{}{^{#2}}}{\diffd #3\IfNoValueTF{#2}{}{^{#2}}} \argopen(#5\argclose)}
%     }
%     {\fractype{\diffd \IfNoValueTF{#2}{}{^{#2}} #3}{\diffd #4\IfNoValueTF{#2}{}{^{#2}}}}
% }
% \DeclareDocumentCommand\dv{}{\derivative} % Shorthand for \derivative

% \DeclareDocumentCommand\partialderivative{ s o m g g d() }
% { % Partial derivative
%     % s: star for \flatfrac flat derivative
%     % o: optional n for nth derivative
%     % m: mandatory (x in df/dx)
%     % g: optional (f in df/dx)
%     % g: optional (y in d^2f/dxdy)
%     % d: long-form d/dx(...)
%     \IfBooleanTF{#1}
%     {\let\fractype\flatfrac}
%     {\let\fractype\frac}
%     \IfNoValueTF{#4}
%     {
%         \IfNoValueTF{#6}
%         {\fractype{\partial \IfNoValueTF{#2}{}{^{#2}}}{\partial #3\IfNoValueTF{#2}{}{^{#2}}}}
%         {\fractype{\partial \IfNoValueTF{#2}{}{^{#2}}}{\partial #3\IfNoValueTF{#2}{}{^{#2}}} \argopen(#6\argclose)}
%     }
%     {
%         \IfNoValueTF{#5}
%         {\fractype{\partial \IfNoValueTF{#2}{}{^{#2}} #3}{\partial #4\IfNoValueTF{#2}{}{^{#2}}}}
%         {\fractype{\partial^2 #3}{\partial #4 \partial #5}}
%     }
% }
% \DeclareDocumentCommand\pderivative{}{\partialderivative} % Shorthand for \partialderivative
% \DeclareDocumentCommand\pdv{}{\partialderivative} % Shorthand for \partialderivative

\DeclareDocumentCommand\variation{ o g d() }{ % Functional variation
    % o: optional n for nth differential
    % g: optional argument for readability and to control spacing
    % d: long-form as in d(F(g))
    \IfNoValueTF{#2}{
        \IfNoValueTF{#3}
        {\delta \IfNoValueTF{#1}{}{^{#1}}}
        {\mathinner{\delta \IfNoValueTF{#1}{}{^{#1}}\argopen(#3\argclose)}}
    }
    {\mathinner{\delta \IfNoValueTF{#1}{}{^{#1}}#2} \IfNoValueTF{#3}{}{(#3)}}
}
\DeclareDocumentCommand\var{}{\variation} % Shorthand for \variation

\DeclareDocumentCommand\functionalderivative{ s o m g d() }
{ % Functional derivative
    % s: star for \flatfrac flat derivative
    % o: optional n for nth derivative
    % m: mandatory (g in dF/dg)
    % g: optional (F in dF/dg)
    % d: long-form d/dx(...)
    \IfBooleanTF{#1}
    {\let\fractype\flatfrac}
    {\let\fractype\frac}
    \IfNoValueTF{#4}
    {
        \IfNoValueTF{#5}
        {\fractype{\variation \IfNoValueTF{#2}{}{^{#2}}}{\variation #3\IfNoValueTF{#2}{}{^{#2}}}}
        {\fractype{\variation \IfNoValueTF{#2}{}{^{#2}}}{\variation #3\IfNoValueTF{#2}{}{^{#2}}} \argopen(#5\argclose)}
    }
    {\fractype{\variation \IfNoValueTF{#2}{}{^{#2}} #3}{\variation #4\IfNoValueTF{#2}{}{^{#2}}}}
}
\DeclareDocumentCommand\fderivative{}{\functionalderivative} % Shorthand for \functionalderivative
\DeclareDocumentCommand\fdv{}{\functionalderivative} % Shorthand for \functionalderivative


\begin{document}

\section*{Metric and Topological Spaces \hfill IB Mich}

\section{Metric Spaces}
\begin{itemize}
    \item Metric Space $= (X,d)$, a set $X$, a metric $d$, satisfying the axioms:
          \begin{enumerate}
              \item $d(a,b) \geq 0 $, with equality iff $a=b$ (Positive definite)
              \item $d(a,b) = d(b,a) $ (Symmetric)
              \item $d(a,c) \leq d(a,b) + d(b,c)$ (Triangle inequality)
          \end{enumerate}
    \item Examples: \begin{itemize}
              \item On $\R^n$: discrete metric (1 if equal, 0 otherwise), Euclidean metric, Manhattan metric, British railway metric
              \item On $\Z$: $p$-adic metric
              \item On $C[0,1]$: uniform metric ($\, d(f,g) = \max_{x\in[0,1]} \abs{f(x)-g(x)}\,$)
              \item Counting (Hamming metric)
          \end{itemize}
    \item Metric subspace: subset with same metric
    \item Convergence and continuity can be defined via metric: \\
          (see below for alternative definitions of continuity)
    \item Notions in vector space naturally give metric space:
          \begin{itemize}
              \item Norm: (positive definite, triangle inequality, scalable), e.g. $L^1, L^2,$ uniform norm $L^\infty$
              \item Inner product: (positive definite, symmetric, linear), e.g. dot product, integral etc. Cauchy-Schwarz give triangle inequality
          \end{itemize}
\end{itemize}

\subsection*{Open and closeness}
\begin{itemize}
    \item Open ball:  $B_r(x) = \{ y \in X: d(x,y) < r\}$

          Closed ball: $\bar{B}_r(x) = \{ y \in X: d(x,y) \leq r\}$
    \item Open subset:
          $U \subseteq X$ is open if $\forall x \in U$, $\exists \delta >0$ s.t. $B_\delta(x) \subseteq U$ \\
          (Every point is interior point)

          Closed subset: $C$ is closed in $X$ if $X\backslash C$ is open \\
          (Open or close is a property of the \emph{subset}, depends on the parent set as well)
    \item Open neighbourhood of $x$ in $X$: an open subset in $X$ containing $x$
    \item Limit point: any point $x$ s.t. there exists sequence $x_n \to x$ \\ (can be approached using a sequence)
    \item $C \subseteq X$ is closed iff every limit point of $C$ belongs to $C$
    \item Properties:
          \begin{itemize}
              \item $\emptyset$ and $X$ are open subsets of $X$
              \item Union (finite or infinite, both countable and uncountable) of open sets is an open set
              \item Finite intersection of open sets is open
          \end{itemize}

\end{itemize}

\subsection*{Alternate characterisation of continuity}

Using open/close sets, or $\varepsilon-\delta$ definition:\\
Given metric spaces $(X,d_x)$ and $(Y,d_y)$, $f: X \to Y$, $f$ is continuous
\begin{itemize}
    \item $\forall x_n \to x$, $f(x_n) \to f(x)$
    \item $U\subseteq Y$ open $\implies f^{-1}(U) \subseteq X$ open
    \item $ C\subseteq Y$ open $\implies f^{-1}(C) \subseteq X$ closed
    \item $ \forall x\in X, \, \forall \varepsilon >0, \, \exists \, \delta>0$ such that $f(B_\delta(x)) \subseteq B_\varepsilon(f(x))$

\end{itemize}
Composition of continuous functions is continuous

\begin{itemize}
    \item A sequence $x_n$ is Cauchy if for all $\varepsilon > 0, \exists N(\varepsilon)$ s.t. \[d(x_m, x_n) < \varepsilon \text{ whenever } n, m \geq N(\varepsilon) \]
          (same as IA Analysis)
    \item A metric space is complete if every Cauchy sequences converges \\ (not a topological property)

\end{itemize}

\section{Topological Spaces}

\begin{itemize}
    \item Topological Space: a set $X$ (the space) and $\tau \subseteq \mathbb{P}(X)$, (the topology) s.t.
          \begin{itemize}
              \item $\emptyset, X \in \tau$
              \item $V_\alpha \in \tau$ for all $\alpha \in A \implies \bigcup_{\alpha \in A} V_\alpha \in \tau $ \quad (finite or ctbly/unctbly infinite)
              \item $V_1, V_2, \dots V_n \in \tau \implies \bigcap_{i=1}^n V_i \in \tau$ \qquad (must be finite)
          \end{itemize}
          ($X$ is the collection of points, and $\tau$ is all subsets we \emph{designate} to be open)
    \item Induced topology: \\ From a given metric $d$, set $\tau$ to be the open subsets of $X$ under $d$
    \item Examples:
          \begin{itemize}
              \item Coarse/indiscrete Topology: $\tau = \{\emptyset, X\}$
              \item Discrete Topology: all subsets $\tau =\mathbb{P}(X)$ (from discrete metric)
              \item Cofinite Topology: $\tau = \{A \subseteq X : X \backslash A \text{ is is finite or } A = \emptyset \}$
              \item Right Order Topology on $\R$: $\tau = \{(a,\infty) : a \in \R \} \cup \{\R = (-\infty, \infty), \emptyset\}$
          \end{itemize}
    \item Continuity, close subsets can be defined only using open sets given by topology
\end{itemize}

\subsection*{Interior and Closure}
Given a topological space $(X,\tau)$, and $A \subseteq X$,

\begin{itemize}
    \item The interior of $A$ is the union of all open sets, i.e. the largest open set contained in $A$: \[\text{Int } (A) = \bigcup \{ U \in \tau : U \subseteq A\}\]
    \item The closure of $A$ is the intersection of all closed sets, i.e. the smallest closed set containing $A$: \[ \text{Cl } (A) = \bigcap \{ F \text{ closed} : F \supseteq A \}\]
    \item \[(\text{Cl } (A^c))^c = \text{Int } (A), \quad (\text{Int } (A^c))^c = \text{Cl } (A)\]
    \item In a metric space $(X,d)$, the closure of $A \subseteq X$ adds in all the limit points of $A$ in $X$
    \item A subset $A \subseteq X$ is a dense in $F \subseteq X$ if $\text{Cl } (A) = F$
\end{itemize}

\subsection*{Homeomorphism}

A function $f : X \to Y$ is a homeomorphism ($X \simeq Y$, homeomorphic, regarded as the same) if
\begin{itemize}
    \item $f$ is a bijection
    \item both $f$ and $f^{-1}$ are continuous
\end{itemize}

(Must require inverse to also be continuous, since continuous function might not continuous inverse)

\begin{itemize}
    \item Homeomorphism is an equivalence relation
    \item Topological properties are preserved by homeomorphisms (things defined only using open sets)
\end{itemize}

\subsection*{Sequences}
\begin{itemize}
    \item Open neighbourhood can now be defined using the open sets given by the topology: an open subset $U \subseteq X$ s.t. $x \in U$
    \item Convergent sequence: $x_n \to x$ if for any open neighbourhood $U$ of $x$, $\exists N$ s.t. $x_n \in U$ for all $n > N$
    \item A limit point $x \in A \subseteq X$ has sequence $x_n \to x$ where $x_n \in A$ for all $n$ (i.e. for all open neighbourhood $U$ of $x$, $U \cap A \neq \emptyset$)\\ (Alternative defn's exist)
    \item If $U$ is closed, it contains all its limit points (converse not true for topological space)
\end{itemize}

In this definition, limit of sequence (if exists) may not be unique, but a special class, Hausdorff space:
\begin{itemize}
    \item $X$ is a Hausdorff space if for any $x_1, x_2 \in X$, there exist open neighbourhoods $U_1$ of $x_1$, and $U_2$ of $x_2$, s.t. $U_1 \cap U_2 = \emptyset$\\
          (Can separate any two points by open neighbourhoods)
    \item Any sequence in a Hausdorff space has at most one limit
    \item Is a topological property
\end{itemize}

\section{New Topological Spaces From Old}
\subsection*{Subspace topology}
If $(X,\tau_X)$ is a topology, $Y \subseteq X$, then the subspace topology on $(Y,\tau_Y)$ is given by $\tau_Y = \{ Y \cap U \mid U \in \tau_X \}$ \\ (open sets in Y are given by intersecting $Y$ with open sets in $X$)
\begin{itemize}
    \item If $Y \subseteq X$, with inclusion $\iota: Y \to X$, then if $f: Z \to Y$ is continuous iff $\iota \circ f: Z \to X$ is continuous
\end{itemize}
(Defining property: Topology on $Y$ is the smallest topology on Y for which the inclusion $\iota Y \to X$ is continuous)

\subsection*{Product topology}
\begin{itemize}
    \item Basis: for a topological space $(X,\tau)$, a subset $\mathcal{B} \subset \tau$ is a basis for the topology if every $U \in \tau_X$ is a union of elements in $\mathcal{B}$

    \item If $(X,\tau_X), (Y,\tau_Y)$ are topological spaces, define product topology on $X \times Y$ using basis
          \[\mathcal{B} = \{U_X \times U_Y \mid U_X \in \tau_X, U_Y \in \tau_y \} \]

          Alternatively, define by:
          $V_X \times V_Y \subseteq X \times Y $ is open if for all $(x,y) \in V_X \times V_Y$, there exist open neighbourhoods $U_X$ of $x$, $U_Y$ of $y$ s.t. $(x,y) \in U_X \times U_Y \subseteq V_X \times V_Y$

          Projection maps are continuous

\end{itemize}
(Defining property: $f$ is continuous iff $\pi_i \circ f$ are continuous)


\subsection*{Quotient topology}
Given $\sim$ an equivalence relation on $X$, quotient map $q: X \to X/\sim $ by $q(x) = [x]$,
define quotient topology by: $U$ is open in $X/\sim $ if $q^{-1}(U)$ is open in $X$

Quotient maps are continuous by construction of $X/\sim$

(Defining property: $f: X\sim \to Y$ is continuous if and only if $f \circ q : X \to Y$ is continuous)

\section{Connectivity}

Topological space $X$ is disconnected if $X = A \cup B$, union of two non-empty disjoint open sets \\ (topological property of the space)

\begin{itemize}
    \item $X$ is disconnected iff $\exists$ homeomorphism $X \to \{0,1\}$ with discrete topology \\ (alternative characterisation)
    \item $f: X \to Y$ cts, $X$ connected, then $\im f$ connected.
    \item Path from $x_0$ to $x_1$: continuous $\gamma: [0,1] \to X$ s.t. $\gamma(0) = x_0$ and $\gamma(1) = x_1$
    \item Path connected: there is a path between any 2 points
    \item Path connected $\implies$ connected
    \item $f: X \to Y$ homeomorphism, then restricting to any subset $A \subset X$, $f|_A: A \to f(A)$ is also homeomorphism \\ Homeomorphic spaces stay homeomorphic after taking away points
    \item (n-connectedness)
\end{itemize}

Components: cut up disconnected space into components

\begin{itemize}
    \item Path components: $x \sim y$ if there exists path from $x$ to $y$, take $X/\sim$ the equivalence classes
    \item Connected components (regular connectivity): \[C(x) = \bigcup \{ \text{connected subsets of } X \text{ containing } x \} \], also connected, equivalence classes
          \\ (read notes again)
          \\ maximal connected subspaces
\end{itemize}

\section{Compactness}
Open cover of $X$: a family of open sets $\{U_\alpha: \alpha \in A \}$ such that $\bigcup_{\alpha \in A} U_\alpha = X$\\
Compactness: every open cover of $X$ has a finite subcover

\begin{itemize}
    \item Finite subspace is compact, $[0,1]$ is compact (read proof again)
    \item Closed subset of compact space is compact (as subspace topology)
    \item If $X$ Hausdorff, $C$ compact $\implies$ closed in $X$
    \item Boundedness is not a topological property
    \item Compact metric space is bounded (everything is $R$ away from me, then everything is $2R$ away from each other)
    \item Heine-Borel: $C \subset \R$ is compact iff $C$ is closed and bounded
    \item Image of compact set under continuous map is compact
    \item Maximum value theorem: continuous map on compact domain
    \item If $X$ and $Y$ are compact, then $X \times Y$ is compact
    \item Quotients
    \item Compact metric space is complete
\end{itemize}


Sequential Compactness: \\
$X$ is sequentially compact if any sequence $(x_n)$ in $X$ has convergent subsequence\\
Equivalent to compactness for metric spaces


\end{document}